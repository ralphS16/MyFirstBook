\documentclass[main.tex]{subfiles}
\begin{document}
\chapter{Natural Transformations}\label{chap:natural}
%TODO:knowledge for arrows. \rightsquigarrow and \Rightarrow
%TODO: define "natural in ..."
%TODO: translate commutativity of diagram with natural transformations to commutativity of the components.
\section{Natural Transformations}
"Natural transformations" are admittedly what made mathematicians want to study category theory in the first place. In short, they are "morphisms" between "functors", i.e.: transformations that preserve the structure of "functors".

The abstract structure of a "category" is very familiar because it resembles what is found in algebraic structures such as "groups", "rings" or "vector spaces". That is to say, it consists of the data of one or more sets with one or more operations satisfying one or more properties. In contrast, the definition of a functor is more opaque and by itself, the structure of a functor is not obvious. A functor is effectively a morphism between categories, hence a natural transformation will be a \textit{morphism between morphisms}. Before moving on, one might find it enlightening to look for a satisfying definition of morphism between two group homomorphisms $f,g: G \rightarrow H$ and then observe its meaning when $f$ and $g$ are seen as functors $\deloop{G} \rightsquigarrow \deloop{H}$.

%TODO: better intro ! - explain that functors are slightly more opaque but not that much. Similarly to categories where we only bothered to preserve composition and units and the rest followed, Here we only look at the action on morphisms and the rest follows. For composition, you need the square to commute.

For the general case, let $F,G: \mathbf{C}\rightsquigarrow \mathbf{D}$ be "functors". Morally, the structure of $F$ and $G$ is encapsulated in the following diagrams for every arrow, $f \in \Hom_{\mathbf{C}}(A,B)$.
\begin{center}%TODO: better diagrams.
	
	\begin{minipage}{0.38\textwidth}
		\begin{equation}\label{diag:funcF}
		\begin{tikzcd}
		A \arrow[d, "f"'] \arrow[r, "F_0"] & F(A) \arrow[d, "F_1(f)"] \\
		B \arrow[r, "F_0"']                & F(B)                 
		\end{tikzcd}
		\end{equation}
	\end{minipage}
	\begin{minipage}{0.38\textwidth}
		\begin{equation}\label{diag:funcG}
		\begin{tikzcd}
		A \arrow[d, "f"'] \arrow[r, "G_0"] & G(A) \arrow[d, "G_1(f)"] \\
		B \arrow[r, "G_0"']                & G(B)                 
		\end{tikzcd}
		\end{equation}
	\end{minipage}
\end{center}
Thus, a "morphism" between $F$ and $G$ should fit in this picture by sending diagram \eqref{diag:funcF} to diagram \eqref{diag:funcG} in a "commutative" way.
\begin{defn}[Natural transformation]\label{defnattran}
	\AP Let $F,G : \mathbf{C} \rightsquigarrow \mathbf{D}$ be two ("covariant") "functors", a "natural transformation" $\phi: F \Rightarrow G$ is a map $\phi: \obj{\mathbf{C}} \rightarrow \mor{\mathbf{D}}$ that satisfies $\phi(A) \in \Hom_{\mathbf{D}}(F(A), G(A))$ for all $A \in \obj{\mathbf{C}}$ and makes \eqref{diag:nattrans} "commute" for any $f \in \Hom_{\mathbf{C}}(A,B)$:\footnote{When doing proofs relying on "naturality" (i.e.: the property of being "natural"), we will use \eqref{diag:nattrans} where we instantiate $\phi$, $F$, $G$, $A$, $B$ and $f$ with the "natural transformation", "functors", "objects" and "morphism" that is needed in the proof. In order to make this instantiation less painful, we will use the shorthand $\intro*\NAT(\phi,A,B,f)$ and instantiate the parameters (we can omit $F$ and $G$ because they should be known from the context).}
	\begin{equation}\label{diag:nattrans}
	\begin{tikzcd}
	F(A) \arrow[d, "F(f)"'] \arrow[r, "\phi(A)"] & G(A) \arrow[d, "G(f)"] \\
	F(B) \arrow[r, "\phi(B)"'] & G(B)
	\end{tikzcd}
	\end{equation}
    \AP Each $\phi(A)$ will be called a ""component"" of $\phi$ and may also be denoted $\phi_A$.
\end{defn}
\AP As usual, there are trivial examples of "natural transformations" such as the ""identity transformation"" $\one_F:F \Rightarrow F$ that sends every "object" $A$ to the "identity" map $\id_{F(A)}$, but let us go back to the group case. Although very specific to "single object" "categories", it is simple enough to quickly digest.
\begin{exmp}\label{exmp:grouphom}
	Let $f,g: \deloop{G} \rightsquigarrow \deloop{H}$ be "functors" (i.e.: "group homomorphisms"), both send the unique "object" $\deloopobject$ in $\deloop{G}$ to $\deloopobject$ in $\deloop{H}$. Thus, a "natural transformation" $\phi : f\Rightarrow g$ has a single "component" $\phi(\deloopobject):\deloopobject \rightarrow \deloopobject$ in $H$, which is simply an element $\phi \in H$. The "commutativity" condition is then exhibited by diagram \eqref{diag:homnattrans} (which lives in $\deloop{H}$) for any $x \in G$.
	\begin{equation}\label{diag:homnattrans}
	\begin{tikzcd}
	* \arrow[d, "f(x)"'] \arrow[r, "\phi"] & * \arrow[d, "g(x)"] \\
	* \arrow[r, "\phi"'] & *
	\end{tikzcd}
	\end{equation}
	Recall that composition in $\deloop{H}$ is just multiplication in $H$, so "naturality" of $\phi$ says that for any $x \in G$, $\phi \cdot f(x) = g(x) \cdot \phi$. Equivalently, $\phi f(x) \phi^{-1} = g(x)$. Therefore, $g = c_{\phi} \circ f$ where $c_{\phi}$ denotes "conjugation" by $\phi$.\footnote{\AP In a "group" $(H, \cdot)$, ""conjugation"" by an element $h \in H$ is the "homomorphism@@GRP" $c_h$ defined $x \mapsto hxh^{-1}$.} In short, "natural transformations" between "group homomorphisms" correspond to factorizations through "conjugations".
\end{exmp}
Next, an example closer to the general idea of a "natural transformation".
\begin{exmp}
	Fix some $n \in \N$ and define the "functor" $\gln:\catCRing \rightsquigarrow \catGrp$ by\footnote{The map $\gln(f)$ is just the extension of $f$ on $\gln(R)$ by applying $f$ to every element of the matrices.}
	\begin{align*}
	R &\mapsto \gln(R) \mbox{ for any "commutative ring" $R$ and} \\
	f &\mapsto \gln(f) \mbox{ for any "ring homomorphism" $f$.}
	\end{align*}
	The second "functor" is $\units{(-)}:\catCRing \rightsquigarrow \catGrp$ which sends a "commutative ring" $R$ to its "group" of "units@@RING" $\units{R}$ and a "ring homomorphism" $f$ to $\units{f}$, its restriction on $\units{R}$. Checking these mappings define two "covariant" "functors" is left as an (simple) exercise, but one might expect these to be "functors" as they play nicely with the structure of the "objects" involved.%TODO:detail.
	
	A "natural transformation" between these two "functors" is $\det:\gln \Rightarrow \units{(-)}$ which maps a "commutative ring" $R$ to $\det_R$, the function calculating the "determinant" of a "matrix" in $\gln(R)$. The first thing to check is that $\det_R \in \Hom_{\catGrp}(\gln(R), \units{R})$ which is clear because the "determinant" of an "invertible" "matrix" is always a "unit@@RING", $\det_R(I_n) = 1$ and $\det_R$ is a multiplicative map.\footnote{i.e.: $\det_R(AB)= \det_R(A)\det_R(B)$.} The second thing is to verify that diagram \eqref{diag:detnat} "commutes" for any $f\in \Hom_{\catCRing}(R,S)$:
	\begin{equation}\label{diag:detnat}
	\begin{tikzcd}
	\text{GL}_n(R) \arrow[r, "\det_R"] \arrow[d, "\text{GL}_n(f)"'] & \units{R} \arrow[d, "\units{f} = f\mid_{\units{R}}"] \\
	\text{GL}_n(S) \arrow[r, "\det_S"'] & \units{S}
	\end{tikzcd}
	\end{equation}
	We will check the claim for $n=2$, but the general proof should only involve more notation to write the bigger expressions, no novel idea. Let $a,b,c,d \in R$, we have 
	\begin{align*}
	({\det}_S \circ \text{GL}_2(f))\left( \begin{bmatrix}a&b\\c&d\end{bmatrix} \right)&= 
	{\det}_S\left(\begin{bmatrix}f(a)&f(b)\\f(c)&f(d)\end{bmatrix}\right)\\
	&= f(a)f(d)-f(b)f(c)\\
	&= f(ad-bc)\\
	&= \units{f}(ad-bc)\\
	&= (\units{f}\circ {\det}_R)\left( \begin{bmatrix}a&b\\c&d\end{bmatrix}\right).
	\end{align*}
	We conclude that the diagram "commutes" and that $\det$ is indeed a "natural transformation".\footnote{Modulo the cases $n>2$.}
\end{exmp}
\begin{exer}\label{exer:natural:componentwise}\marginnote{\hyperref[soln:natural:componentwise]{See solution.}}
	Let $F, G: \mathbf{C} \cattimes \mathbf{C}' \rightsquigarrow \mathbf{D}$ be two "functors". Show that a family \[\left\{ \phi_{X,Y}: F(X,Y) \rightarrow G(X,Y)\mid X \in \obj{\mathbf{C}}, Y \in \obj{\mathbf{C}'} \right\}\] is a "natural transformation" if and only if for any $X \in \obj{\mathbf{C}}$ and $Y \in \obj{\mathbf{C}'}$, both \[\phi_{X,\placeholder}: F(X,\placeholder) \Rightarrow G(X,\placeholder) \text{ and } \phi_{\placeholder,Y}: F(\placeholder,Y) \Rightarrow G(\placeholder,Y)\] are "natural".
\end{exer}
Now, in order to talk about a "category" of "functors", it remains to describe the "composition" of "natural transformations".
\begin{defn}[Vertical composition]
	Let $F,G,H: \mathbf{C}\rightsquigarrow \mathbf{D}$ be "parallel" "functors" and $\phi:F\Rightarrow G$ and $\eta:G\Rightarrow H$ be two "natural transformations". \AP Then, the ""vertical composition"" of $\phi$ and $\eta$, denoted $\eta\vertcomp \phi:F\Rightarrow H$ is defined by $(\eta \vertcomp \phi)(A) = \eta(A) \circ \phi(A)$ for all $A \in \obj{\mathbf{C}}$. If $f: A\rightarrow B$ is a "morphism" in $\mathbf{C}$, then diagram \eqref{diag:vertcomp} "commutes" by "naturality" of $\phi$ and $\eta$, showing that $\eta \vertcomp \phi$ is a "natural transformation" from $F$ to $H$.\marginnote{The notation $\vertcomp$ is not widespread, most authors use $\circ$ because "vertical composition" is the "composition" in a "functor category". We believe the distinction is helpful as you learn this material.}
	\begin{equation}\label{diag:vertcomp}
	\begin{tikzcd}
	F(A) \arrow[r, "\phi(A)"] \arrow[d, "F(f)"'] & G(A) \arrow[r, "\eta(A)"] \arrow[d, "G(f)"'] & H(A) \arrow[d, "H(f)"'] \\
	F(B) \arrow[r, "\phi(B)"'] & G(B) \arrow[r, "\eta(B)"'] & H(B)
	\end{tikzcd}
	\end{equation}
	
	The meaning of \textit{vertical} will come to light when "horizontal composition" is introduced in a bit.
\end{defn}
\begin{defn}[Functor categories]
	For any two "categories" $\mathbf{C}$ and $\mathbf{D}$, there is a "functor category" denoted $\catFunc{\mathbf{C}}{\mathbf{D}}$.\footnote{Some authors denote it $\mathbf{D}^{\mathbf{C}}$, analogously to the exponential of sets.} Its "objects" are "functors" from $\mathbf{C}$ to $\mathbf{D}$, its "morphisms" are "natural transformations" between such "functors" and the "composition" is the "vertical composition" defined above. One can check that "associativity" of $\vertcomp$ follows from "associativity" of "composition" in $\mathbf{D}$ and that the "identity" "morphism" for a "functor" $F$ is $\one_F$.%TODO: check?
\end{defn}
\begin{exmp}\label{exmp:grpactionfunctor}
	Recall that a "left action" of a "group" $G$ on a set $S$ is just a functor $\deloop{G} \rightsquigarrow \catSet$. Now, between two such "functors" $F,F' \in \catFunc{\deloop{G}}{\catSet}$, a "natural transformation" is a single map $\sigma: F(\deloopobject) \rightarrow F'(\deloopobject)$ such that $\sigma \circ F(g) = F'(g) \circ \sigma$ for any $g \in G$. In other words, denoting $\cdot$ for both "group actions" on $F(\deloopobject)$ and on $F'(\deloopobject)$, $\sigma$ satisfies $\sigma(g\cdot x) = g\cdot(\sigma(x))$ for any $g \in G$ and $x \in F(\deloopobject)$. \AP In group theory, such a map is called $G$--""equivariant"".
	
	Therefore, the "category" $\catFunc{\deloop{G}}{\catSet}$ can be identified as the category of $G$--"sets@gset" (sets equipped with an "action" of $G$) with $A$--"equivariant" maps as the "morphisms".
\end{exmp}
\begin{exer}[\NOW]\label{exer:natural:natiso}\marginnote{\hyperref[soln:natural:natiso]{See solution.}}
	\AP "Isomorphisms@@CAT" in a "functor category" are called ""natural isomorphisms"". Show that they are precisely the "natural transformations" whose "components" are all "isomorphisms".\marginnote{"Functors" that are "naturally isomorphic" are essentially the same "functor"; they send the same "object" to "isomorphic@@CAT" "objects" and the same "morphism" to "morphisms" that are well-behaved under "composition" with "isomorphisms@@CAT" between the "source" and "targets".}
\end{exer}

\begin{exmps}\label{exmp:simplefunccat}
	We can recover constructions we have seen before by studying "categories" of "functors" with a simple domain.
	\begin{enumerate}
		\item The "terminal" "category" $\termcat$ has a single "object" $\bullet$ and no "morphism" other than the "identity". Notice that for any "category" $\mathbf{C}$, a "functor" $F: \termcat \rightsquigarrow \mathbf{C}$ is a simply a choice of "object" $F(\bullet) \in \obj{\mathbf{C}}$ because $F(\id_{\bullet}) = \id_{F(\bullet)}$. If $F, G\in \catFunc{\termcat}{\mathbf{C}}$, then a "natural transformation" $\phi: F \Rightarrow G$ is simply a choice of "morphism" $\phi: F(\bullet) \rightsquigarrow G(\bullet)$ because "naturality" square \eqref{diag:natsquarebullet} for the only "morphism" $\id_{\bullet}$ is trivially "commutative". We conclude that $\catFunc{\termcat}{\mathbf{C}}$ can be identified with the "category" $\mathbf{C}$ itself.\begin{marginfigure}[-3\baselineskip]
			\begin{equation}\label{diag:natsquarebullet}
				\begin{tikzcd}
					{F(\bullet)} & {F(\bullet)} \\
					{G(\bullet)} & {G(\bullet)}
					\arrow["{F(\id_{\bullet})}", from=1-1, to=1-2]
					\arrow["\phi", from=1-2, to=2-2]
					\arrow["\phi"', from=1-1, to=2-1]
					\arrow["{G(\id_{\bullet})}"', from=2-1, to=2-2]
				\end{tikzcd}
			\end{equation}
		\end{marginfigure}
		\item\label{exmp:isoprodfunccat} Similarly, we can see a "functor" $F: \mathbf{1}\coproduct \mathbf{1} \rightsquigarrow \mathbf{C}$\footnote{Recall $\mathbf{1}\coproduct \mathbf{1}$ is the "category" depicted in \eqref{diag:cat1p1}.} as a choice of two "objects" $F(\bullet_1)$ and $F(\bullet_2)$ and a "natural transformation" $\phi: F \Rightarrow G$ between two such "functors" as a choice of two "morphisms" $\phi_1 : F(\bullet_1) \rightarrow G(\bullet_1)$ and $\phi_2 : F(\bullet_2) \rightarrow G(\bullet_2)$. Therefore, we infer that $\catFunc{\mathbf{1}\coproduct \mathbf{1}}{\mathbf{C}}$ can be identified with $\mathbf{C}\cattimes \mathbf{C}$.
		\item Let us go one level harder. A "functor" $F: \cattwo \rightsquigarrow \mathbf{C}$\footnote{Recall $\cattwo$ is the "category" depicted in \eqref{diag:cat2}.} is a choice of two "objects" $FA$ and $FB$ as well as a morphism $Ff: FA \rightarrow FB$. It can also be seen as a single choice of "morphism" $Ff$ because $FA$ and $FB$ are determined to be the "source" and "target" of $Ff$ respectively. A "natural transformation" $\phi: F \Rightarrow G$ between two such "functors" is \textit{not} simply a choice of two "morphisms" $\phi_A : FA \rightarrow GA$ and $\phi_B: FB \rightarrow GB$ because, while the "naturality" squares for $\id_A$ and $\id_B$ trivially "commute", the "naturality" square \eqref{diag:natsquarearrow} for $f$ is an additional constraint on $\phi$. Namely, it says $(\phi_A,\phi_B)$ makes a "commutative" square with $Ff$ and $Gf$, hence we can identify $\catFunc{\cattwo}{\mathbf{C}}$ with the "arrow category" $\arrowcat{\mathbf{C}}$.
		\begin{marginfigure}[-3\baselineskip]
			\begin{equation}\label{diag:natsquarearrow}
				\begin{tikzcd}
					{FA} & {FB} \\
					{GA} & {GB}
					\arrow["{Ff}", from=1-1, to=1-2]
					\arrow["\phi_B", from=1-2, to=2-2]
					\arrow["\phi_A"', from=1-1, to=2-1]
					\arrow["{Gf}"', from=2-1, to=2-2]
				\end{tikzcd}
			\end{equation}
		\end{marginfigure}
	\end{enumerate}
\end{exmps}
%TODO: maybe comma category as exercise.

%TODO: limits are taken pointwise
%Before giving another example, we present a very nice result using limits. It is essentially saying that constructions inside the functor category $\mathbf{D}^{\mathbf{C}}$ are usually as simple as inside $\mathbf{C}$.
%\begin{prop}
%	Let $\mathbf{C}$, $\mathbf{D}$ and $J$ be categories. If all limits of shape $J$ exist in $\mathbf{C}$, then all such limits also exist in $\mathbf{D}^{\mathbf{C}}$.
%\end{prop}
%\begin{proof}
%	
%\end{proof}

It is now time to build intuition for the "horizontal composition" of "natural transformations" which will ultimately lead to the notion of a \kl[2cat]{$2$--category}.
\begin{defn}[The left action of functors]\label{defn:leftaction}
	Let $F,F':\mathbf{C}\rightsquigarrow \mathbf{D}$, $G:\mathbf{D}\rightsquigarrow \mathbf{D}'$ be "functors" and $\phi:F\Rightarrow F'$ a "natural transformation" as summarized in \eqref{diag:leftaction}.\footnote{Using squiggly arrows for "functors" in diagrams is very non-standard, but I believe it helps remember what kind of objects we are dealing with. Moreover, since these diagrams are not "commutative", it makes a good contrast with the plain arrow notation which was mostly used for "commutative" diagrams.}
	\begin{equation}\label{diag:leftaction}
	% https://q.uiver.app/?q=WzAsMyxbMCwwLCJcXG1hdGhiZntDfSJdLFszLDAsIlxcbWF0aGJme0R9Il0sWzQsMCwiXFxtYXRoYmZ7RH0nIl0sWzAsMSwiRiIsMCx7ImN1cnZlIjotMywic3R5bGUiOnsiYm9keSI6eyJuYW1lIjoic3F1aWdnbHkifX19XSxbMSwyLCJHIl0sWzAsMSwiRiciLDIseyJjdXJ2ZSI6Mywic3R5bGUiOnsiYm9keSI6eyJuYW1lIjoic3F1aWdnbHkifX19XSxbMyw1LCJcXHBoaSIsMCx7InNob3J0ZW4iOnsic291cmNlIjoyMCwidGFyZ2V0IjoyMH19XV0=
    \begin{tikzcd}
	{\mathbf{C}} &&& {\mathbf{D}} & {\mathbf{D}'}
	\arrow[""{name=0, anchor=center, inner sep=0}, "F", curve={height=-18pt}, squiggly, from=1-1, to=1-4]
	\arrow["G", from=1-4, to=1-5]
	\arrow[""{name=1, anchor=center, inner sep=0}, "{F'}"', curve={height=18pt}, squiggly, from=1-1, to=1-4]
	\arrow["\phi", shorten <=5pt, shorten >=5pt, Rightarrow, from=0, to=1]
    \end{tikzcd}
	\end{equation}	
	The "functor" $G$ acts on $\phi$ by sending it to $G\phi := A \mapsto G(\phi(A)) : \obj{\mathbf{C}} \rightarrow \mor{\mathbf{D}'}$. Showing that \eqref{diag:commleftaction} "commutes" for any $f \in \Hom_{\mathbf{C}}(A,B)$ will imply that $G\phi$ is a "natural transformation" from $G\circ F$ to $G\circ F'$ .
	\begin{equation}\label{diag:commleftaction}
		\begin{tikzcd}
		(G\circ F)(A) \arrow[d, "(G\circ F)(f)"'] \arrow[r, "G\phi(A)"] & (G\circ F')(A) \arrow[d, "(G\circ F')(f)"] \\
		(G\circ F)(B) \arrow[r, "G\phi(B)"'] & (G\circ F')(B)
		\end{tikzcd}
	\end{equation}
	Consider this diagram after removing all applications of $G$, by "naturality" of $\phi$, it is "commutative". Since "functors" "preserve" "commutativity", the diagram still "commutes" after applying $G$, hence $G\phi: G\circ F \Rightarrow G \circ F'$ is indeed "natural".\footnote{More concisely, we apply $G$ to $\NAT(\phi,A,B,f)$ to obtain \eqref{diag:commleftaction}.}
	
	We leave you to check this constitutes a left action, namely, for any $G:\mathbf{D}\rightsquigarrow \mathbf{D}'$, $G':\mathbf{D}' \rightsquigarrow \mathbf{D}''$ and $\phi:F\Rightarrow F'$, \[\id_{\mathbf{D}}\phi = \phi \text{ and } G'(G\phi)= (G' \circ G)\phi.\]
\end{defn}

\begin{defn}[The right action of functors]\label{defn:rightaction}
	Let $F,F':\mathbf{C}\rightsquigarrow \mathbf{D}$, $H:\mathbf{C}'\rightsquigarrow \mathbf{C}$ be "functors" and $\phi:F\Rightarrow F'$ a "natural transformation" as summarized in \eqref{diag:rightaction}.
	\begin{equation}\label{diag:rightaction}
	% https://q.uiver.app/?q=WzAsMyxbMSwwLCJcXG1hdGhiZntDfSJdLFs0LDAsIlxcbWF0aGJme0R9Il0sWzAsMCwiXFxtYXRoYmZ7Q30nIl0sWzAsMSwiRiIsMCx7ImN1cnZlIjotMywic3R5bGUiOnsiYm9keSI6eyJuYW1lIjoic3F1aWdnbHkifX19XSxbMCwxLCJGJyIsMix7ImN1cnZlIjozLCJzdHlsZSI6eyJib2R5Ijp7Im5hbWUiOiJzcXVpZ2dseSJ9fX1dLFsyLDAsIkgiLDAseyJzdHlsZSI6eyJib2R5Ijp7Im5hbWUiOiJzcXVpZ2dseSJ9fX1dLFszLDQsIlxccGhpIiwwLHsic2hvcnRlbiI6eyJzb3VyY2UiOjIwLCJ0YXJnZXQiOjIwfX1dXQ==
    \begin{tikzcd}
        {\mathbf{C}'} & {\mathbf{C}} &&& {\mathbf{D}}
        \arrow[""{name=0, anchor=center, inner sep=0}, "F", curve={height=-18pt}, squiggly, from=1-2, to=1-5]
        \arrow[""{name=1, anchor=center, inner sep=0}, "{F'}"', curve={height=18pt}, squiggly, from=1-2, to=1-5]
        \arrow["H", squiggly, from=1-1, to=1-2]
        \arrow["\phi", shorten <=5pt, shorten >=5pt, Rightarrow, from=0, to=1]
    \end{tikzcd}
	\end{equation}
	
	The "functor" $H$ acts on $\phi$ by sending it to $\phi H := A \mapsto \phi(H(A)) : \obj{\mathbf{C}'} \rightarrow \mor{\mathbf{D}}$. Showing that \eqref{diag:commrightaction} "commutes" for any $f \in \Hom_{\mathbf{C}'}(A,B)$ will imply that $\phi H$ is a "natural transformation" from $F\circ H$ to $F'\circ H$.
	\begin{equation}\label{diag:commrightaction}
	\begin{tikzcd}
	(F\circ H)(A) \arrow[d, "(F\circ H)(f)"'] \arrow[r, "\phi H(A)"] & (F'\circ H)(A) \arrow[d, "(F'\circ H)(f)"] \\
	(F\circ H)(B) \arrow[r, "\phi H(B)"'] & (F'\circ H)(B)
	\end{tikzcd}
	\end{equation}
	"Commutativity" of \eqref{diag:commrightaction} follows by "naturality" of $\phi$: change $f$ in diagram \eqref{diag:nattrans} with the "morphism" $H(f):H(A) \rightarrow H(B)$, i.e.: \eqref{diag:commrightaction} is $\NAT(\phi, HA,HB,Hf)$.
	
	We leave you to check this constitutes a right action, namely, for any $H:\mathbf{C}'\rightsquigarrow \mathbf{C}$, $H':\mathbf{C}''\rightsquigarrow \mathbf{C}'$ and $\phi:F\Rightarrow F'$,
	\[\phi \id_{\mathbf{C}} = \phi \text{ and } (\phi H)H' = \phi(H \circ H').\]
\end{defn}

\begin{prop}
	The two actions commute, i.e.: in the setting of \eqref{diag:commleftright}, $G(\phi H) = (G\phi) H$.\footnote{For this reason and the associativity of the two actions, we will drop all the parentheses from such expressions. We will also drop the $\circ$ for "composition" of "functors". All in all, expect to find expressions like $G'G\phi HH'$ and infer the "natural transformation" $A \mapsto G'(G(\phi(H(H'(A)))))$.}%TODO: associativity ?
	\begin{equation}\label{diag:commleftright}
	% https://q.uiver.app/?q=WzAsNCxbMSwwLCJcXG1hdGhiZntDfSJdLFs0LDAsIlxcbWF0aGJme0R9Il0sWzUsMCwiXFxtYXRoYmZ7RH0nIl0sWzAsMCwiXFxtYXRoYmZ7Q30nIl0sWzAsMSwiRiIsMCx7ImN1cnZlIjotMywic3R5bGUiOnsiYm9keSI6eyJuYW1lIjoic3F1aWdnbHkifX19XSxbMSwyLCJHIl0sWzAsMSwiRiciLDIseyJjdXJ2ZSI6Mywic3R5bGUiOnsiYm9keSI6eyJuYW1lIjoic3F1aWdnbHkifX19XSxbMywwLCJIIiwwLHsic3R5bGUiOnsiYm9keSI6eyJuYW1lIjoic3F1aWdnbHkifX19XSxbNCw2LCJcXHBoaSIsMCx7InNob3J0ZW4iOnsic291cmNlIjoyMCwidGFyZ2V0IjoyMH19XV0=
    \begin{tikzcd}
        {\mathbf{C}'} & {\mathbf{C}} &&& {\mathbf{D}} & {\mathbf{D}'}
        \arrow[""{name=0, anchor=center, inner sep=0}, "F", curve={height=-18pt}, squiggly, from=1-2, to=1-5]
        \arrow["G", from=1-5, to=1-6]
        \arrow[""{name=1, anchor=center, inner sep=0}, "{F'}"', curve={height=18pt}, squiggly, from=1-2, to=1-5]
        \arrow["H", squiggly, from=1-1, to=1-2]
        \arrow["\phi", shorten <=5pt, shorten >=5pt, Rightarrow, from=0, to=1]
    \end{tikzcd}
	\end{equation}
\end{prop}
\begin{proof}
In both the L.H.S. and the R.H.S., an object $A \in \obj{\mathbf{C}}$ is sent to $G(\phi(H(A)))$.
\end{proof}
A very useful result following from the properties of these two actions is that for any "commutative" "diagram" in $\catFunc{\mathbf{C}}{\mathbf{D}}$, we can "pre-compose" and "post-compose" with any "functors" and still obtain a "commutative" "diagram". For instance, if \eqref{diag:commuteinfunc} "commutes" in $\catFunc{\mathbf{C}}{\mathbf{D}}$, then for any "functors" $F: \mathbf{C'} \rightsquigarrow \mathbf{C}$ and $G: \mathbf{D} \rightsquigarrow \mathbf{D'}$, then \eqref{diag:commuteinfunccomposed} "commutes".\footnote{We will often use this property by writing things like ``apply $F(\placeholder)G$ to \eqref{diag:commuteinfunc}'' to use the "commutativity" of \eqref{diag:commuteinfunccomposed} in a proof.}\\
\begin{minipage}{0.49\textwidth}
	\begin{equation}\label{diag:commuteinfunc}
		\begin{tikzcd}
			X & Y \\
			{X'} & {Y'}
			\arrow["\phi"', from=1-1, to=2-1]
			\arrow["\eta", from=1-1, to=1-2]
			\arrow["{\phi'}", from=1-2, to=2-2]
			\arrow["{\eta'}"', from=2-1, to=2-2]
		\end{tikzcd}
	\end{equation}
\end{minipage}\begin{minipage}{0.49\textwidth}
	\begin{equation}\label{diag:commuteinfunccomposed}
		\begin{tikzcd}
			{F\circ X\circ G} & {F\circ Y\circ G} \\
			{F\circ X'\circ G} & {F\circ Y'\circ G}
			\arrow["{F\phi G}"', from=1-1, to=2-1]
			\arrow["{F\phi' G}", from=1-2, to=2-2]
			\arrow["{F\eta'G}"', from=2-1, to=2-2]
			\arrow["{F\eta G}", from=1-1, to=1-2]
		\end{tikzcd}
	\end{equation}
\end{minipage}\\

\AP We will refer to these two actions as the ""biaction"" of "functors" on "natural transformations" and they will motivate the definition of another way to "compose" "natural transformations".

Let $\mathbf{C}$, $\mathbf{D}$ and $\mathbf{E}$ be "categories", $H,H': \mathbf{C}\rightsquigarrow \mathbf{D}$ and $G,G':\mathbf{D} \rightsquigarrow \mathbf{E}$ be "functors" and $\phi:H\Rightarrow H'$ and $\eta:G\Rightarrow G'$ be "natural transformations". This is summarized in \eqref{diag:horizcompsetting}.
\begin{equation}\label{diag:horizcompsetting}
% https://q.uiver.app/?q=WzAsMyxbMCwwLCJcXG1hdGhiZntDfSJdLFszLDAsIlxcbWF0aGJme0R9Il0sWzYsMCwiXFxtYXRoYmZ7RX0iXSxbMCwxLCJIIiwwLHsiY3VydmUiOi0zLCJzdHlsZSI6eyJib2R5Ijp7Im5hbWUiOiJzcXVpZ2dseSJ9fX1dLFswLDEsIkgnIiwyLHsiY3VydmUiOjMsInN0eWxlIjp7ImJvZHkiOnsibmFtZSI6InNxdWlnZ2x5In19fV0sWzEsMiwiRyciLDIseyJjdXJ2ZSI6Mywic3R5bGUiOnsiYm9keSI6eyJuYW1lIjoic3F1aWdnbHkifX19XSxbMSwyLCJHIiwwLHsiY3VydmUiOi0zLCJzdHlsZSI6eyJib2R5Ijp7Im5hbWUiOiJzcXVpZ2dseSJ9fX1dLFszLDQsIlxccGhpIiwwLHsic2hvcnRlbiI6eyJzb3VyY2UiOjIwLCJ0YXJnZXQiOjIwfX1dLFs2LDUsIlxcZXRhIiwwLHsic2hvcnRlbiI6eyJzb3VyY2UiOjIwLCJ0YXJnZXQiOjIwfX1dXQ==
\begin{tikzcd}
	{\mathbf{C}} &&& {\mathbf{D}} &&& {\mathbf{E}}
	\arrow[""{name=0, anchor=center, inner sep=0}, "H", curve={height=-18pt}, squiggly, from=1-1, to=1-4]
	\arrow[""{name=1, anchor=center, inner sep=0}, "{H'}"', curve={height=18pt}, squiggly, from=1-1, to=1-4]
	\arrow[""{name=2, anchor=center, inner sep=0}, "{G'}"', curve={height=18pt}, squiggly, from=1-4, to=1-7]
	\arrow[""{name=3, anchor=center, inner sep=0}, "G", curve={height=-18pt}, squiggly, from=1-4, to=1-7]
	\arrow["\phi", shorten <=5pt, shorten >=5pt, Rightarrow, from=0, to=1]
	\arrow["\eta", shorten <=5pt, shorten >=5pt, Rightarrow, from=3, to=2]
\end{tikzcd}
\end{equation}
The ultimate goal is to obtain a new "composition" of $\phi$ and $\eta$ that is a "natural transformation" $G\circ H \Rightarrow G'\circ H'$. Note that the "biaction" defined above yields four other "natural transformations":
\begin{align*}
	G\phi&: G\circ H \Rightarrow G\circ H' &&\eta H: G\circ H \Rightarrow G'\circ H \\
	G'\phi&: G'\circ H \Rightarrow G'\circ H'&&\eta H': G\circ H' \Rightarrow G'\circ H'.
\end{align*}
All of the "functors" involved go from $\mathbf{C}$ to $\mathbf{E}$, so all four "natural transformations" fit in diagram \eqref{diag:etdc} that lives in the "functor category" $\catFunc{\mathbf{C}}{\mathbf{E}}$.
\begin{equation}\label{diag:etdc}
\begin{tikzcd}
G\circ H \arrow[r, "G\phi"] \arrow[d, "\eta H"'] & G\circ H' \arrow[d, "\eta H'"] \\
G'\circ H \arrow[r, "G'\phi"']                   & G'\circ H'                    
\end{tikzcd}
\end{equation}

At first glance, this suggests two different definitions for the "horizontal composition", that is, the "composition@comppaths" of the top "path" $(\eta H' \vertcomp G\phi)$ or the "composition@comppaths" of the bottom "path" $(G'\phi \vertcomp \eta H)$. Surprisingly, both definitions coincide as shown in the next result.
\begin{lem}
	Diagram \eqref{diag:etdc} "commutes", i.e.: $\eta H' \vertcomp G\phi = G'\phi \vertcomp \eta H$.\footnote{Similarly to $\NAT$, we will refer to the "commutativity" of \eqref{diag:etdc} with $\intro*\HOR(\phi,\eta)$. We use $\HOR$ because this lemma is crucial in the definition of "horizontal composition".}
\end{lem}
\begin{proof}
Fix an object $A \in \obj{\mathbf{C}}$. Under $\eta H' \vertcomp G\phi$, it is sent to $\eta(H'(A)) \circ G(\phi(A))$ and under $G'\phi \vertcomp \eta H$, it is sent to $G'(\phi(A)) \circ \eta(H(A))$. Thus, the proposition is equivalent to saying diagram \eqref{diag:proofhorizcomp} is "commutative" (in $\mathbf{E}$) for all $A \in \obj{\mathbf{C}}$.
\begin{equation}\label{diag:proofhorizcomp}
\begin{tikzcd}
(G\circ H)(A) \arrow[r, "G(\phi(A))"] \arrow[d, "\eta(H(A))"'] & (G\circ H')(A) \arrow[d, "\eta(H'(A))"] \\
(G'\circ H)(A) \arrow[r, "G'(\phi(A))"']                       & (G'\circ H')(A)                        
\end{tikzcd}
\end{equation}
This follows from $\NAT(\eta,HA,H'A,\phi(A))$.
\end{proof}

\begin{defn}[Horizontal composition]\label{horizcomp}
	\AP In the setting described in \eqref{diag:horizcompsetting}, we define the ""horizontal composition"" of $\eta$ and $\phi$ by $\eta \horcomp \phi = \eta H' \vertcomp G\phi = G'\phi\vertcomp \eta H$.\footnote{The $\horcomp$ notation is not standard but there are no widespread symbol denoting "horizontal composition". I have mostly seen $\ast$ or plain juxtaposition. Hopefully, you will encounter papers/books clear enough that you can typecheck to find what "composition" is being used.}
\end{defn}
The most important part we expect from a notion of "composition" is "associativity", so let us check $\horcomp$ is "associative".
\begin{prop}
	In the setting of \eqref{diag:assochorizcomp}, $\psi \horcomp (\eta \horcomp \phi)= (\psi \horcomp \eta)\horcomp \phi$.
	\begin{equation}\label{diag:assochorizcomp}
	% https://q.uiver.app/?q=WzAsNCxbMCwwLCJcXG1hdGhiZntDfSJdLFszLDAsIlxcbWF0aGJme0R9Il0sWzYsMCwiXFxtYXRoYmZ7RX0iXSxbOSwwLCJcXG1hdGhiZntGfSJdLFswLDEsIkgiLDAseyJjdXJ2ZSI6LTMsInN0eWxlIjp7ImJvZHkiOnsibmFtZSI6InNxdWlnZ2x5In19fV0sWzAsMSwiSCciLDIseyJjdXJ2ZSI6Mywic3R5bGUiOnsiYm9keSI6eyJuYW1lIjoic3F1aWdnbHkifX19XSxbMSwyLCJHJyIsMix7ImN1cnZlIjozLCJzdHlsZSI6eyJib2R5Ijp7Im5hbWUiOiJzcXVpZ2dseSJ9fX1dLFsxLDIsIkciLDAseyJjdXJ2ZSI6LTMsInN0eWxlIjp7ImJvZHkiOnsibmFtZSI6InNxdWlnZ2x5In19fV0sWzIsMywiSyIsMCx7ImN1cnZlIjotMywic3R5bGUiOnsiYm9keSI6eyJuYW1lIjoic3F1aWdnbHkifX19XSxbMiwzLCJLJyIsMix7ImN1cnZlIjozLCJzdHlsZSI6eyJib2R5Ijp7Im5hbWUiOiJzcXVpZ2dseSJ9fX1dLFs0LDUsIlxccGhpIiwwLHsic2hvcnRlbiI6eyJzb3VyY2UiOjIwLCJ0YXJnZXQiOjIwfX1dLFs3LDYsIlxcZXRhIiwwLHsic2hvcnRlbiI6eyJzb3VyY2UiOjIwLCJ0YXJnZXQiOjIwfX1dLFs4LDksIlxccHNpIiwwLHsic2hvcnRlbiI6eyJzb3VyY2UiOjIwLCJ0YXJnZXQiOjIwfX1dXQ==
    \begin{tikzcd}
        {\mathbf{C}} &&& {\mathbf{D}} &&& {\mathbf{E}} &&& {\mathbf{F}}
        \arrow[""{name=0, anchor=center, inner sep=0}, "H", curve={height=-18pt}, squiggly, from=1-1, to=1-4]
        \arrow[""{name=1, anchor=center, inner sep=0}, "{H'}"', curve={height=18pt}, squiggly, from=1-1, to=1-4]
        \arrow[""{name=2, anchor=center, inner sep=0}, "{G'}"', curve={height=18pt}, squiggly, from=1-4, to=1-7]
        \arrow[""{name=3, anchor=center, inner sep=0}, "G", curve={height=-18pt}, squiggly, from=1-4, to=1-7]
        \arrow[""{name=4, anchor=center, inner sep=0}, "K", curve={height=-18pt}, squiggly, from=1-7, to=1-10]
        \arrow[""{name=5, anchor=center, inner sep=0}, "{K'}"', curve={height=18pt}, squiggly, from=1-7, to=1-10]
        \arrow["\phi", shorten <=5pt, shorten >=5pt, Rightarrow, from=0, to=1]
        \arrow["\eta", shorten <=5pt, shorten >=5pt, Rightarrow, from=3, to=2]
        \arrow["\psi", shorten <=5pt, shorten >=5pt, Rightarrow, from=4, to=5]
    \end{tikzcd}
	\end{equation}
\end{prop}
\begin{proof}
	Similarly to how we constructed diagram \eqref{diag:etdc} in $\catFunc{\mathbf{C}}{\mathbf{E}}$ previously, we can use the "biaction" of "functors" and "composition" of "functors" to obtain the following diagram in $\catFunc{\mathbf{C}}{\mathbf{E}}$.\footnote{All $\circ$'s are left out for simplicity.} \marginnote[2\baselineskip]{Here is how each face "commutes".\begin{itemize}
        \item[\textbf{Top:}]$\HOR(\psi,G\eta)$
        \item[\textbf{Bottom:}]$\HOR(\psi,G'\eta)$
        \item[\textbf{Left:}]$\HOR(\psi,\eta H)$
        \item[\textbf{Right:}]$\HOR(\psi,\eta H')$
        \item[\textbf{Front:}] $\HOR(K\eta,\phi)$
        \item[\textbf{Back:}] $\HOR(K'\eta,\phi)$
    \end{itemize}}
	\begin{equation}
	\begin{tikzcd}
	& K'GH \arrow[dd, "K'\eta H" near start] \arrow[rr, "K'G\phi"] &                                                    & K'GH' \arrow[dd, "K'\eta H'"] \\
	KGH \arrow[rr, crossing over, "KG\phi"' near end] \arrow[dd, "K\eta H"'] \arrow[ru, "\psi GH"] &                                                    & KGH'  \arrow[ru, "\psi GH'"'] &                               \\
	& K'G'H \arrow[rr, "K'G'\phi" near start]                       &                                                    & K'G'H'                        \\
	KG'H \arrow[rr, "KG'\phi"'] \arrow[ru, "\psi G'H"]                    &                                                    & KG'H'\arrow[uu, <-, crossing over, "K\eta H'" near start] \arrow[ru, "\psi G'H'"']                     &                              
	\end{tikzcd}
	\end{equation}
	As detailed in the margin, this "commutes" because each face of the cube corresponds to a variant of diagram \eqref{diag:etdc} (with some substitutions and application of a "functor") and combining "commutative" diagrams yields "commutative" diagrams. Then, it follows that $\horcomp$ is associative because\footnote{We may have drawn only the front and right face, but the cube is cooler.} $\psi \horcomp (\eta \horcomp \phi)$ is the diagonal of the front face followed by the bottom right arrow and $(\psi \horcomp \eta) \horcomp \phi$ is the top front arrow followed by the diagonal of the right face.
	% \begin{itemize}
    %     \item[\textbf{Top:}]$\NAT(\psi,K,K',GHX, GH'X,G\phi_X)$
    %     \item[\textbf{Bottom:}]$\NAT(\psi,K,K',G'HX, G'H'X,G'\phi_X)$
    %     \item[\textbf{Left:}]$\NAT(\psi,K,K',GHX, G'HX,\eta_{HX})$
    %     \item[\textbf{Right:}]$\NAT(\psi,K,K',GH'X, G'H'X,\eta_{H'X})$
    %     \item[\textbf{Front:}] Apply $K$ to $\NAT(\eta, G,G',HX, H'X,\phi_X)$
    %     \item[\textbf{Back:}] Apply $K'$ to $\NAT(\eta, G,G',HX, H'X,\phi_X)$
    % \end{itemize}
\end{proof}

There is one last thing to conclude that $\catCat$ is a \kl[2cat]{$2$--category}, namely, that the "vertical" and "horizontal" "compositions" interact nicely.
\begin{prop}[Interchange identity]\label{prop:interchange}
	\AP In the setting of \eqref{diag:interidsetting}, the ""interchange identity"" holds:
	\begin{equation}\label{eqn:interid}
		(\eta' \vertcomp \eta) \horcomp (\phi' \vertcomp \phi) = (\eta' \horcomp \phi') \vertcomp (\eta \horcomp \phi).
	\end{equation}
    \marginnote{It is in the drawing of \eqref{diag:interidsetting} that the intuition behind the terms "vertical" and "horizontal" is taken.}
	\begin{equation}\label{diag:interidsetting}
	% https://q.uiver.app/?q=WzAsMyxbMCwwLCJcXG1hdGhiZntDfSJdLFszLDAsIlxcbWF0aGJme0R9Il0sWzYsMCwiXFxtYXRoYmZ7RX0iXSxbMCwxLCJIJyIsMCx7ImxhYmVsX3Bvc2l0aW9uIjo4MCwic3R5bGUiOnsiYm9keSI6eyJuYW1lIjoic3F1aWdnbHkifX19XSxbMSwyLCJHJyciLDIseyJjdXJ2ZSI6NSwic3R5bGUiOnsiYm9keSI6eyJuYW1lIjoic3F1aWdnbHkifX19XSxbMSwyLCJHIiwwLHsiY3VydmUiOi01LCJzdHlsZSI6eyJib2R5Ijp7Im5hbWUiOiJzcXVpZ2dseSJ9fX1dLFsxLDIsIkcnIiwwLHsibGFiZWxfcG9zaXRpb24iOjgwLCJzdHlsZSI6eyJib2R5Ijp7Im5hbWUiOiJzcXVpZ2dseSJ9fX1dLFswLDEsIkgnJyIsMix7ImN1cnZlIjo1LCJzdHlsZSI6eyJib2R5Ijp7Im5hbWUiOiJzcXVpZ2dseSJ9fX1dLFswLDEsIkgiLDAseyJjdXJ2ZSI6LTUsInN0eWxlIjp7ImJvZHkiOnsibmFtZSI6InNxdWlnZ2x5In19fV0sWzgsMywiXFxwaGkiLDAseyJzaG9ydGVuIjp7InNvdXJjZSI6MjAsInRhcmdldCI6MjB9fV0sWzMsNywiXFxwaGknIiwwLHsic2hvcnRlbiI6eyJzb3VyY2UiOjIwLCJ0YXJnZXQiOjIwfX1dLFs1LDYsIlxcZXRhIiwwLHsic2hvcnRlbiI6eyJzb3VyY2UiOjIwLCJ0YXJnZXQiOjIwfX1dLFs2LDQsIlxcZXRhJyIsMCx7InNob3J0ZW4iOnsic291cmNlIjoyMCwidGFyZ2V0IjoyMH19XV0=
    \begin{tikzcd}
        {\mathbf{C}} &&& {\mathbf{D}} &&& {\mathbf{E}}
        \arrow[""{name=0, anchor=center, inner sep=0}, "{H'}"{pos=0.8}, squiggly, from=1-1, to=1-4]
        \arrow[""{name=1, anchor=center, inner sep=0}, "{G''}"', curve={height=30pt}, squiggly, from=1-4, to=1-7]
        \arrow[""{name=2, anchor=center, inner sep=0}, "G", curve={height=-30pt}, squiggly, from=1-4, to=1-7]
        \arrow[""{name=3, anchor=center, inner sep=0}, "{G'}"{pos=0.8}, squiggly, from=1-4, to=1-7]
        \arrow[""{name=4, anchor=center, inner sep=0}, "{H''}"', curve={height=30pt}, squiggly, from=1-1, to=1-4]
        \arrow[""{name=5, anchor=center, inner sep=0}, "H", curve={height=-30pt}, squiggly, from=1-1, to=1-4]
        \arrow["\phi", shorten <=4pt, shorten >=4pt, Rightarrow, from=5, to=0]
        \arrow["{\phi'}", shorten <=4pt, shorten >=4pt, Rightarrow, from=0, to=4]
        \arrow["\eta", shorten <=4pt, shorten >=4pt, Rightarrow, from=2, to=3]
        \arrow["{\eta'}", shorten <=4pt, shorten >=4pt, Rightarrow, from=3, to=1]
    \end{tikzcd}
	\end{equation}
    
\end{prop}
\begin{proof}
	Akin to the other proofs, this is a matter of combining the right diagrams. After combining the diagrams in $[\mathbf{C},\mathbf{E}]$ corresponding to $\eta \horcomp \phi$ and $\eta'\horcomp \phi'$, it is easy to see that the R.H.S. of \eqref{eqn:interid} is the "morphism" going from $G\circ H$ to $G''\circ H''$ (see \eqref{diag:rhsinterid}).
	\begin{equation}\label{diag:rhsinterid}
	\begin{tikzcd}
	G\circ H \arrow[r, "G\phi"] \arrow[d, "\eta H"'] & G\circ H' \arrow[d, "\eta H'"]                       &                                  \\
	G'\circ H \arrow[r, "G'\phi"']                   & G'\circ H' \arrow[d, "\eta'H'"'] \arrow[r, "G'\phi'"] & G'\circ H'' \arrow[d, "\eta'H''"] \\
	& G''\circ H' \arrow[r, "G''\phi'"']                   & G''\circ H''                    
	\end{tikzcd}
	\end{equation}

	Moreover, observe that the diagram corresponding to the L.H.S. can be factored with the following equations (it also yields the factored diagram in \eqref{diag:factoredLHS}). \begin{marginfigure}\begin{equation}\label{diag:factoredLHS}
		\begin{tikzcd}
		G\circ H \arrow[r, "G\phi"] \arrow[d, "\eta H"'] & G\circ H' \arrow[r, "G\phi'"]      & G\circ H'' \arrow[d, "\eta H''"]  \\
		G'\circ H \arrow[d, "\eta'H"']                   &                                    & G'\circ H'' \arrow[d, "\eta'H''"] \\
		G''\circ H \arrow[r, "G''\phi"']                 & G''\circ H' \arrow[r, "G''\phi'"'] & G''\circ H''                     
		\end{tikzcd}
	\end{equation}\end{marginfigure}
	\begin{align*}
	(\eta'\vertcomp \eta)H = \eta'H\vertcomp \eta H && (\eta'\vertcomp \eta)H'' = \eta'H''\vertcomp \eta H''\\
	G(\phi'\vertcomp \phi) = G\phi'\vertcomp G\phi && G''(\phi'\vertcomp \phi) = G''\phi'\vertcomp G''\phi
	\end{align*}
	Combining the factored diagram with \eqref{diag:rhsinterid}, we obtain \eqref{diag:interid} from which the "interchange identity" readily follows.\footnote{The top right and bottom left square "commute" by $\HOR(\eta,\phi')$ and $\HOR(\eta',\phi)$ respectively. This implies all of \eqref{diag:interid} "commutes" and we have seen that the "path" from $G \circ H$ to $G'' \circ H''$ can be seen as the R.H.S. of \eqref{eqn:interid} by looking at \eqref{diag:rhsinterid} or the L.H.S. by looking at \eqref{diag:factoredLHS}. Thus, we infer the equality of \eqref{eqn:interid}.}
	\begin{equation}\label{diag:interid}
	\begin{tikzcd}
	G\circ H \arrow[r, "G\phi"] \arrow[d, "\eta H"']    & G\circ H' \arrow[d, "\eta H'"] \arrow[r, "G\phi'"]    & G\circ H'' \arrow[d, "\eta H''"]  \\
	G'\circ H \arrow[r, "G'\phi"'] \arrow[d, "\eta'H"'] & G'\circ H' \arrow[d, "\eta'H'"'] \arrow[r, "G'\phi'"] & G'\circ H'' \arrow[d, "\eta'H''"] \\
	G''\circ H \arrow[r, "G''\phi"']                     & G''\circ H' \arrow[r, "G''\phi'"']                    & G''\circ H''                     
	\end{tikzcd}
	\end{equation}
\end{proof}

\begin{defn}[Strict $2$--cateory]\label{defn:2cat}
	\AP A ""strict $2$--category@2cat"" consists of
	\begin{itemize}
		\item a "category" $\mathbf{C}$,
		\item for every $A,B \in \obj{\mathbf{C}}$ a "category" $\mathbf{C}(A,B)$ with $\Hom_{\mathbf{C}}(A,B)$ as its "objects" ("composition" is denoted $\vertcomp$ and "identities" $\one$) and "morphisms" are called $2$--\textbf{morphisms},
		\item a "category" with $\obj{\mathbf{C}}$ as its "objects", where the "morphisms" are pairs of "parallel" "morphisms" of $\mathbf{C}$ along with a $2$--morphism between them\footnote{A "morphism" in this category is also called a $2$--\textbf{cell}.} and the identity map sends $A \in \obj{\mathbf{C}}$ to the pair $(\id_A, \id_A)$ and the $2$--morphism $\one_{\id_A}$ ("composition" of $2$--cells is denoted $\horcomp$),
	\end{itemize} 
	such that the "interchange identity" \eqref{eqn:interid} holds.
\end{defn}
We will not cover it in this book, but there are notions of "morphisms" between "$2$--categories@2cat" (called $2$--functors), between $3$--categories as well as between $n$--categories for any $n$ (even $n= \infty$!), these objects are more deeply studied in higher category theory.\footnote{Most of higher category theory drops the \textit{strict} part of our definition of $2$--category because this condition is too strong. Very briefly, they allow the properties of "composition", namely "associativity" and "identities", to hold up to "natural isomorphisms".}

\begin{exer}[\NOW]\label{exer:natural:compositionisfunc}\marginnote[1\baselineskip]{\hyperref[soln:natural:equivequiv]{See solution.}}
	Show that there is a "functor" $\catFunc{\mathbf{D}}{\mathbf{E}} \cattimes \catFunc{\mathbf{C}}{\mathbf{D}} \rightsquigarrow \catFunc{\mathbf{C}}{\mathbf{E}}$ whose action on "objects" is $(F,G) \mapsto F\circ G$.
\end{exer}%TODO: explain the implications for commutative diagrams of natural transformations and how you will use it.
%TODO: req exercises: Composing with a constant yields a functor that preserves isomorphisms for use when saying limits are taken pointwise [C x D, E] = [D x C, E]
\section{Equivalences}
As is expected, an isomorphism of "categories" is an "isomorphism@@CAT" in the "category" $\catCat$, namely, a "functor" $F:\mathbf{C}\rightsquigarrow \mathbf{D}$ with an inverse $G:\mathbf{D}\rightsquigarrow \mathbf{C}$ such that $F \circ G = \id_{\mathbf{D}}$ and $G\circ F = \id_{\mathbf{C}}$. As is typical in mathematics, one cannot distinguish between "isomorphic@@CAT" "categories" as they only differ in notations and terminology.%TODO: explain.
\begin{exmps}\label{exmp:isocat}
	\begin{enumerate}
		\item[]\marginnote[2\baselineskip]{Another example for readers who know a bit of advanced algebra. Let $k$ be a "field" and $G$ a finite "group", the "categories" of $k[G]$--\href{https://en.wikipedia.org/wiki/Module_(mathematics)}{modules} ($k[G]$ is the \href{https://en.wikipedia.org/wiki/Group_ring}{group ring} of $k$ over $G$) and of $k$--\href{https://en.wikipedia.org/wiki/Group_representation}{linear representations} of $G$ are "isomorphic@@CAT".}
		\item It was already shown in Example \ref{exmp:grpactionfunctor} (the details were implicit) that for a group $G$, the category $\catFunc{\catSet}{\deloop{G}}$ is "isomorphic@@CAT" to the "category" of $G$--"sets@gset" with $G$--"equivariant" maps as "morphisms".
		\item In Example \ref{exmp:simplefunccat}, three other "isomorphisms@@CAT" were implicitly given:
		\[\catFunc{\termcat}{\mathbf{C}} \isoCAT \mathbf{C} \quad \quad \quad \catFunc{\termcat\coproduct\termcat}{\mathbf{C}} \isoCAT \mathbf{C}\cattimes \mathbf{C} \quad \quad \quad\catFunc{\cattwo}{\mathbf{C}} \isoCAT \arrowcat{\mathbf{C}}.\]
		\item The category $\catRel$ of sets with relations is "isomorphic@@CAT" to $\op{\catRel}$.\footnote{An arbitrary "category" $\mathbf{C}$ is not always "isomorphic@@CAT" to its "opposite". While the "opposite functors" $\op{(\placeholder)}_{\mathbf{C}}: \mathbf{C} \rightsquigarrow \op{\mathbf{C}}$ and $\op{(\placeholder)}_{\op{\mathbf{C}}}: \op{\mathbf{C}} \rightsquigarrow \mathbf{C}$ are inverses of each other, they are "contravariant functors".} The "functor" $\catRel \rightsquigarrow \op{\catRel}$ is the identity on "objects" and sends a relation $R \subseteq X \times Y$ to the opposite relation $\reflectbox{$R$} \subseteq Y \times X$ (which is a "morphism" $X \rightarrow Y$ in $\op{\catRel}$) defined by $(y,x) \in \reflectbox{$R$} \Leftrightarrow (x,y) \in R$. The inverse is defined similarly.
		\item\label{exmp:curryingfunctors} Given three "categories" $\mathbf{C}$, $\mathbf{D}$ and $\mathbf{E}$, there is an "isomorphism@@CAT"\footnote{You might recognize a similarity with "exponentials" which rely on an "isomorphism@@CAT" $\Hom_{\mathbf{C}}(B\times X, A)\isoCAT\Hom_{\mathbf{C}}(B, A^X)$. The example here is more than an instance of "exponentials" of "categories" because the "isomorphism@@CAT" is not only as sets but as "categories".} \[\catFunc{\mathbf{C}\cattimes \mathbf{D}}{\mathbf{E}}\isoCAT\catFunc{\mathbf{C}}{\catFunc{\mathbf{D}}{\mathbf{E}}} .\]%TODO: currying and uncurrying for functors.
		Let $F: \mathbf{C}\cattimes \mathbf{D} \rightsquigarrow \mathbf{E}$, the "currying" of $F$ is $\intro*\Curry{F}: \mathbf{C} \rightsquigarrow \catFunc{\mathbf{D}}{\mathbf{E}}$ defined as follows. For $X \in \obj{\mathbf{C}}$, the 
	\end{enumerate}
\end{exmps}
Although there are other interesting instances of "isomorphic@@CAT" "categories", "natural transformations" lead to a more nuanced (and often more useful) equality between two "categories", that is, "equivalence".
\begin{defn}[Equivalence]
	\AP A "functor" $F:\mathbf{C}\rightsquigarrow \mathbf{D}$ is an ""equivalence"" of "categories" if there exists a "functor" $G:\mathbf{D}\rightsquigarrow \mathbf{C}$ such that $F\circ G\isoCAT \id_{\mathbf{D}}$ and $G\circ F \isoCAT \id_{\mathbf{C}}$.\footnote{Recall that $\isoCAT$ between "functors" stands for "natural isomorphisms".} \AP This is clearly symmetric, so we say two "categories" $\mathbf{C}$ and $\mathbf{D}$ are ""equivalent"", denoted $\mathbf{C} \eqCat \mathbf{D}$, if there is an "equivalence" between them. \AP Moreover, we say that $G$ is a ""quasi-inverse"" of $F$ and vice-versa.
\end{defn}
In order to gain more intuition on how "equivalences" equate two "categories", let us observe what properties this forces on the "functor" $F$. For any "morphism" $f \in \Hom_{\mathbf{C}}(A,B)$, the following square "commutes" where $\phi(A)$ and $\phi(B)$ are "isomorphisms@@CAT".\footnote{"Naturality" of $\phi$ only gives us $GF(f) \circ \phi(A) = \phi(B) \circ f$, but by "composing" with $\phi(A)^{-1}$ or $\phi(B)^{-1}$, we obtain the "commutativity" of all of \eqref{diag:bijfromequiv}. In particular, we have $GF(f) = \phi(B) \circ f \circ \phi(A)^{-1}$.}
\begin{equation}\label{diag:bijfromequiv}
% https://q.uiver.app/?q=WzAsNCxbMCwwLCJBIl0sWzAsMSwiR0YoQSkiXSxbMSwxLCJHRihCKSJdLFsxLDAsIkIiXSxbMCwxLCJcXHBoaShBKSJdLFsxLDIsIkdGKGYpIiwyXSxbMCwzLCJmIl0sWzMsMiwiXFxwaGkoQikiLDJdLFsyLDMsIlxccGhpKEIpXnstMX0iLDIseyJvZmZzZXQiOjJ9XSxbMSwwLCJcXHBoaShBKV57LTF9IiwwLHsib2Zmc2V0IjotMn1dXQ==
\begin{tikzcd}
	A & B \\
	{GF(A)} & {GF(B)}
	\arrow["{\phi(A)}", shift left=1, from=1-1, to=2-1]
	\arrow["{GF(f)}"', from=2-1, to=2-2]
	\arrow["f", from=1-1, to=1-2]
	\arrow["{\phi(B)}"', shift right=1, from=1-2, to=2-2]
	\arrow["{\phi(B)^{-1}}"', shift right=1, from=2-2, to=1-2]
	\arrow["{\phi(A)^{-1}}", shift left=1, from=2-1, to=1-1]
\end{tikzcd}
\end{equation}
This implies that the map $f \mapsto GF(f):\Hom_{\mathbf{C}}(A,B) \rightarrow \Hom_{\mathbf{C}}(GF(A), GF(B))$ is a bijection. Indeed, "pre-composition" by $\phi(A)^{-1}$ and "post-composition" by $\phi(B)$ are both bijections,\footnote{Recall the definitions of "monomorphisms" and "epimorphisms" and the fact that "isomorphisms@@CAT" are "monic" and "epic".} so \[f \mapsto \phi(B) \circ f \circ \phi(A)^{-1} = GF(f)\]is a bijection. Since $A$ and $B$ are arbitrary, $G\circ F$ is a "fully faithful" "functor" and a symmetric argument shows $F\circ G$ is also "fully faithful". Then, it is easy to conclude that $F$ and $G$ must be "fully faithful" as well.%TODO: exercise and ref.

What is more, the existence of an "isomorphism@@CAT" $\eta(A): A \rightarrow FG(A)$ for any object $A$ implies $F$ (symmetrically $G$) has the following property.
\begin{defn}[Essentially surjective]
	\AP A "functor" $F:\mathbf{C}\rightsquigarrow \mathbf{D}$ is ""essentially surjective"" if for any $X \in \obj{\mathbf{D}}$, there exists $Y \in \obj{\mathbf{C}}$ such that $X \isoCAT F(Y)$.
\end{defn}
We will show that these two properties ("full faithfulness" and "essential surjectivity" are necessary and sufficient for $F$ to be an "equivalence".
\begin{thm}
	A "functor" $F:\mathbf{C}\rightsquigarrow \mathbf{D}$ is an "equivalence" of "categories" if and only if $F$ is "fully faithful" and "essentially surjective".
\end{thm}
\begin{proof}
	($\Rightarrow$) Shown above.
	
	($\Leftarrow$) We construct a "functor" $G:\mathbf{D}\rightsquigarrow \mathbf{C}$ such that $G\circ F \isoCAT \id_{\mathbf{C}}$ and $F\circ G \isoCAT \id_{\mathbf{D}}$. Since $F$ is "essentially surjective", for any $A \in \obj{\mathbf{D}}$, there exists an object $G(A) \in \obj{\mathbf{C}}$ and an "isomorphism@@CAT" $\phi(A):F(G(A)) \isoCAT A$. Hence, $A \mapsto G(A)$ is a good candidate to describe the action of $G$ on "objects".
	
	Next, similarly to the converse direction, note that for any $A,B \in \obj{\mathbf{D}}$, the map 
	\[f\mapsto \phi(B) \circ f \circ \phi(A)^{-1}\]
	is a bijection from $\Hom_{\mathbf{D}}(A,B)$ to $\Hom_{\mathbf{D}}(FG(A), FG(B))$. Moreover, since the functor $F$ is "fully faithful", it induces a bijection
    \[F_{A,B}: \Hom_{\mathbf{C}}(G(A), G(B)) \rightarrow \Hom_{\mathbf{D}}(FG(A), FG(B))\] which in turns yields a bijection 
	\[G_{A,B}: \Hom_{\mathbf{D}}(A,B) \rightarrow \Hom_{\mathbf{C}}(G(A), G(B)) = f \mapsto F_1^{-1}(\phi(B) \circ f \circ \phi(A)^{-1}).\]
	This is the action of $G$ on "morphisms". Observe that the construction of $G$ ensures that $F\circ G \isoCAT \id_{\mathbf{D}}$ through the "natural transformation" $\phi$. It remains to show that $G$ is indeed a "functor" and find a "natural isomorphism" $\eta:G\circ F \isoCAT \id_{\mathbf{C}}$.
	
	For any "composable" "morphisms" $(f,g)$, it is easy to verify that 
	\[F(G(f)\circ G(g)) = FG(f) \circ FG(g) = FG(f \circ g),\]
	so "functoriality" of $G$ follows after applying $F_1^{-1}$. To find $\eta$, recall that the definition of $G$ yields "commutativity" of \eqref{diag:findingeta} for any $f\in \Hom_{\mathbf{C}}(A,B)$.
	\begin{equation}\label{diag:findingeta}
		\begin{tikzcd}
	F(A) \arrow[d] \arrow[r, "F(f)"]                    & F(B) \arrow[d]                  \\
	FGF(A) \arrow[u, "\phi(F(A))"] \arrow[r, "FGF(f)"'] & FGF(B) \arrow[u, "\phi(F(B))"']
	\end{tikzcd}
	\end{equation}
	
	Then, because $F$ is "fully faithful", the following square also "commutes" in $\mathbf{C}$ where $\eta = X \mapsto F_1^{-1}(\phi(F(X)))$ and we conclude that $\eta$ is a "natural isomorphism" $\id_{\mathbf{C}} \isoCAT G\circ F$.
	\begin{equation}\label{diag:foundeta}
	\begin{tikzcd}
	A \arrow[d] \arrow[r, "f"]                     & B \arrow[d]                 \\
	GF(A) \arrow[u, "\eta(A)"] \arrow[r, "GF(f)"'] & GF(B) \arrow[u, "\eta(B)"']
	\end{tikzcd}
	\end{equation}
\end{proof}
The insight to extract from this argument is that two categories are "equivalent" if they describe the same "objects" and "morphisms" with the only relaxation that "isomorphic@@CAT" "objects" can appear any number of times in either "category". In contrast, "categories" can only be "isomorphic@@CAT" if they have exactly the same "objects" and "morphisms".

\begin{rem} We used the axiom of choice to construct the "quasi-inverse" of $F$.
\end{rem}%TODO: check

We will detail a couple of \textit{easy} examples of "equivalences" and briefly metion a few \textit{harder} ones.
\begin{exmps}[Easy]
	\begin{enumerate}
		\itemAP Consider the "full@@CAT" "subcategory" of $\catFinSet$ consisting only of the sets $\emptyset, \{1\}, \{1,2\}, \dots, \{1,\dots,n\},\dots$, denote it $\intro*\catFinOrd$.\footnote{The name $\catFinOrd$ is an abbreviation of finite \href{https://en.wikipedia.org/wiki/Ordinal_number}{ordinals}, because we can also define $\catFinOrd$ as the "category" of finite ordinals and functions between them.} The "inclusion functor" is "fully faithful" by definition and we claim it is "essentially surjective". Indeed, any set $X \in \obj{\catFinSet}$ has a finite cardinality $n$, so $X \isoCAT \{1,\dots,n\} \in \obj{\catFinOrd}$.
		\item In a very similar fashion, an early result in linear algebra says that any "finite dimensional" "vector space" over a "field" $k$ is "isomorphic@@VECT" to $k^n$ for some $n\in \N$. \AP Thus, the "category" whose objects are $k^n$ for all $n\in \N$ and "morphisms" are $m\times n$ "matrices" with entries in $k$,\footnote{After making a choice of "basis" for all $k^n$, an $m\times n$ matrix with entries in $k$ corresponds to a "linear map" $k^n \rightarrow k^m$.} which we denote $\intro*\catMat{k}$, is "equivalent" to the "category" of "finite dimensional" "vector spaces".
		\itemAP A ""partial"" function $f: X \pfun Y$ is a function that may not be defined on all of $X$.\footnote{\AP In this context, a \textit{normal} function defined on all of $X$ is called ""total"".} There is "category" $\catPar$ of sets and "partial" functions where "identity morphism" and "composition" are defined straightforwardly.\footnote{You can view $\catPar$ as the "subcategory" of $\catRel$ where you only take the relations $R \subseteq X\times Y$ satisfying for any $x \in X$ (cf. Remark \ref{rem:setsubrel}), \[|\left\{ y \in Y\mid (x,y) \in R \right\}| \leq 1.\]} We can view a "partial" function $f:X \pfun Y$ as a "total" function $f':X \rightarrow Y\coproduct\terminal$ which assigns to every $x$ where $f(x)$ is undefined the value $\ast \in \terminal$. Further extending $f'$ to $[f',\id_{\terminal}]: X\coproduct\terminal \rightarrow Y\coproduct \terminal$, we can see any "partial" function as a function between "pointed" sets where the distinguished element corresponds to being undefined.
		
		We claim that this yields a "fully faithful" "functor" $\catPar \rightsquigarrow \catPtd$ sending $X$ to $(X\coproduct\terminal, \ast)$ and $f: X \pfun Y$ to $[f',\id_{\terminal}]$. %TODO: FINISH.....If we think of a "pointed" set $(X,x)$ as the set $X\setminus x$ can be seen as a .%TODO: One can also see $\catPtd$ as the "category" of sets and ""partial functions"". 
	\end{enumerate}
\end{exmps}
%TODO: skeletons
The first two examples and many other simple examples of "equivalences" are examples of "skeletons". They are morally a "subcategory" where all the "isomorphic@@CAT" copies are removed.
\begin{defn}[Skeleton]
	\AP A "category" is called ""skeletal"" if there it contains no two "isomorphic@@CAT" "objects". A \textbf{"skeleton"} of a "category" is an "equivalent" "skeletal" "category".
\end{defn}
\begin{exmps}
	We have shown that $\catFinOrd \eqCat \catFinSet$ and $\catMat{k} \eqCat\catFDVect{k}$ and we leave to you the easy task to check that these are examples of "skeletons".\footnote{Namely, you should show that no two sets in $\catFinOrd$ are "isomorphic@@CAT" and no two spaces in $\catMat{k}$ are "isomorphic@@CAT".}
\end{exmps}
A "category" always has a "skeleton" if you assume the axiom of choice and the next result justifies say \textit{the} "skeleton" of a "category". 
\begin{exer}\label{exer:natural:skeleton}\marginnote{\hyperref[soln:natural:skeleton]{See solution.}}
	Show that all "skeletons" of a "category" are "isomorphic@@CAT".
\end{exer}
Here are other more interesting examples of "equivalent" "categories".
\begin{exmp}[Medium]
	Let $\mathbf{C}$ be a "category", there is a "functor" $F: \mathbf{C} \rightsquigarrow \arrowcat{\mathbf{C}}$ sending $X$ to $\id_X$ and $f: X \rightarrow Y$ to the "commutative" square in \eqref{diag:commutesquareembedarrow}. This "functor" is an "equivalence" if and only if all "morphisms" in $\mathbf{C}$ are "isomorphisms@@CAT".\footnote{Such a "category" is called a ""groupoid"".} It is clearly "fully faithful", so it is left to show $F$ is "essentially surjective" if and only if $\mathbf{C}$ is a "groupoid".
	\begin{marginfigure}\begin{equation}\label{diag:commutesquareembedarrow}
		\begin{tikzcd}
			X & X \\
			Y & Y
			\arrow["{\id_X}", from=1-1, to=1-2]
			\arrow["f"', from=1-1, to=2-1]
			\arrow["{\id_Y}"', from=2-1, to=2-2]
			\arrow["f", from=1-2, to=2-2]
		\end{tikzcd}
	\end{equation}\end{marginfigure}
	($\Rightarrow$) For any $f: X \rightarrow Y \in \mor{\mathbf{C}}$, by hypothesis, there exists $A \in \obj{\mathbf{C}}$ such that $\id_A \isoCAT f$ in $\arrowcat{C}$. Let $(s: A \rightarrow X ,t: A \rightarrow Y)$ be the "isomorphism@@CAT", its "inverse" must be $(s^{-1},t^{-1})$. Looking at the chain of "commutative" squares in \eqref{diag:chaincommsquare}, we can infer that $s \circ t^{-1}$ is the "inverse" of $f$.\footnote{The "composition" $f \circ s \circ t^{-1}$ is the top path of the combined two leftmost squares, the bottom path is $t \circ t^{-1} \circ \id_Y = \id_Y$. The "composition" $s \circ t^{-1} \circ f$ is the bottom path of the combined two rightmost squares, the top path is $\id_X \circ s \circ s^{-1} =\id_X$.}
	\begin{equation}\label{diag:chaincommsquare}
		% https://q.uiver.app/?q=WzAsMTAsWzEsMCwiQSJdLFsyLDAsIlgiXSxbMSwxLCJBIl0sWzIsMSwiWSJdLFszLDAsIkEiXSxbMywxLCJBIl0sWzQsMCwiWCJdLFs0LDEsIlgiXSxbMCwwLCJZIl0sWzAsMSwiWSJdLFswLDEsInMiXSxbMCwyLCJcXGlkX0EiLDJdLFsyLDMsInQiLDJdLFsxLDMsImYiLDJdLFsxLDQsInNeey0xfSJdLFszLDUsInReey0xfSIsMl0sWzQsNSwiXFxpZF9BIiwyXSxbNCw2LCJzIl0sWzUsNywicyIsMl0sWzYsNywiXFxpZF9YIl0sWzgsMCwidF57LTF9Il0sWzgsOSwiXFxpZF9ZIiwyXSxbOSwyLCJ0XnstMX0iLDJdLFs5LDMsIlxcaWRfWSIsMix7ImN1cnZlIjozfV0sWzEsNiwiXFxpZF9YIiwyLHsiY3VydmUiOi0zfV1d
		\begin{tikzcd}
			Y & A & X & A & X \\
			Y & A & Y & A & X
			\arrow["s", from=1-2, to=1-3]
			\arrow["{\id_A}"', from=1-2, to=2-2]
			\arrow["t"', from=2-2, to=2-3]
			\arrow["f"', from=1-3, to=2-3]
			\arrow["{s^{-1}}", from=1-3, to=1-4]
			\arrow["{t^{-1}}"', from=2-3, to=2-4]
			\arrow["{\id_A}"', from=1-4, to=2-4]
			\arrow["s", from=1-4, to=1-5]
			\arrow["s"', from=2-4, to=2-5]
			\arrow["{\id_X}", from=1-5, to=2-5]
			\arrow["{t^{-1}}", from=1-1, to=1-2]
			\arrow["{\id_Y}"', from=1-1, to=2-1]
			\arrow["{t^{-1}}"', from=2-1, to=2-2]
		\end{tikzcd}
	\end{equation}

	($\Leftarrow$) Let $f: X \rightarrow Y$ be an "object" of $\arrowcat{C}$, the inverse of $f$ satisfies $f \circ f^{-1} = \id_Y$ and $f^{-1} \circ f = \id_X$, so the squares in \eqref{diag:twocommsquares} are "isomorphisms@@CAT" in $\arrowcat{\mathbf{C}}$ (they are inverses of each other). Thus, we find that $f$ is "isomorphic@@CAT" to $\id_X$ which is in the image of $F$.\begin{marginfigure}[-2\baselineskip]\begin{equation}\label{diag:twocommsquares}
		% https://q.uiver.app/?q=WzAsOCxbMCwwLCJYIl0sWzAsMSwiWCJdLFsxLDAsIlgiXSxbMSwxLCJZIl0sWzMsMCwiWCJdLFszLDEsIlgiXSxbNCwwLCJZIl0sWzQsMSwiWCJdLFswLDEsIlxcaWRfWCIsMl0sWzAsMiwiXFxpZF9YIl0sWzIsMywiZiJdLFsxLDMsImYiLDJdLFs0LDUsIlxcaWRfWCIsMl0sWzQsNiwiZiJdLFs2LDcsImZeey0xfSJdLFs1LDcsIlxcaWRfWCIsMl1d
		\begin{tikzcd}
			X & X && X & Y \\
			X & Y && X & X
			\arrow["{\id_X}"', from=1-1, to=2-1]
			\arrow["{\id_X}", from=1-1, to=1-2]
			\arrow["f", from=1-2, to=2-2]
			\arrow["f"', from=2-1, to=2-2]
			\arrow["{\id_X}"', from=1-4, to=2-4]
			\arrow["f", from=1-4, to=1-5]
			\arrow["{f^{-1}}", from=1-5, to=2-5]
			\arrow["{\id_X}"', from=2-4, to=2-5]
		\end{tikzcd}
	\end{equation}\end{marginfigure}
\end{exmp}
\begin{exmps}[Hard] Examples of significant "equivalences" are all over the place in higher mathematics. However, they require a bit of work to describe them, thus let us only say a few words on them.
	\begin{enumerate}
		\item The "equivalence" between the "category" of affine schemes and the "opposite" of the "category" of "commutative rings" is a seminal result scheme theory, a huge part of modern algebraic geometry.%TODO: seminal
		\item The "equivalence" between Boolean lattices and Stone spaces is again seminal in the theory of Stone-type dualities. These can lead to deep connections between topology and logic. One application in particular is the study of the behavior of computer programs through formal semantics.
	\end{enumerate} %TODO: maybe do a guided exercise for the baby stone duality.
\end{exmps}
\begin{exer}\label{exer:natural:equivequiv}
	Show that "equivalence" of "categories" is an equivalence relation.\marginnote{\hyperref[soln:natural:equivequiv]{See solution.}}
\end{exer}
\begin{exer}\label{exer:natural:equivfunccat}
	Show that $\mathbf{C} \eqCat \mathbf{C}'$ and $\mathbf{D} \eqCat \mathbf{D}'$ implies $\catFunc{\mathbf{C}}{\mathbf{D}} \eqCat \catFunc{\mathbf{C}'}{\mathbf{D}'}$.\marginnote{\hyperref[soln:natural:equivfunccat]{See solution.}}
\end{exer}
\end{document}