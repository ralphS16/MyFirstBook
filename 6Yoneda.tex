\documentclass[main.tex]{subfiles}
\begin{document}
%TODO: why I love category theory: % ---- Chap Yoneda: in CT, 99% of the time, you know what to do. Compared to other disciplines where it is harder to find the right method. e.g.: combinatorics: it is practically impossible to find the right way to see your problem. Most of the time (100% of the time in this book) if you don't know what to do, just go back and read the definitions of each part of the problem and try again. This chapter has difficult proofs but a lot of the time, you don't have that many choice. In fact, in the proof of Yoneda, a most important result, stating the result is 99% of the proof done.
\chapter{Yoneda Lemma}\label{chap:yoneda}
%TODO: try to talk about representing elements and rephrase yoneda lemma like nat trans is completely determined by representing element and choice of image and of representing element.
\section{Representable Functors}
Throughout, let $\mathbf{C}$ be a "locally small" "category". Recall that for an "object" $A \in \obj{\mathbf{C}}$, there are two $\Hom$ "functors" from $\mathbf{C}$ to $\catSet$. The "covariant" one, $\Hom_{\mathbf{C}}(A, \placeholder)$, sends an "object" $B \in \obj{\mathbf{C}}$ to $\Hom_{\mathbf{C}}(A,B)$ and a "morphism" $f:B\rightarrow B'$ to $f \circ (\placeholder)$. The "contravariant" one, $\Hom_{\mathbf{C}}(\placeholder,A)$, sends an "object" $B \in \obj{\mathbf{C}}$ to $\Hom_{\mathbf{C}}(B,A)$ and a "morphism" $f: B\rightarrow B'$ to $(\placeholder) \circ f$. In order to lighten the notation, we denote these functors $\Hombis^A$ and $\Hombis_A$ respectively.\footnote{It might seem like this contradicts the notation used so far because $\Hombis^A$ is "covariant" and $\Hombis_A$ "contravariant". However, this is not their \textit{variance} in the parameter $A$, and we will show that in fact, the \textit{variance} in $A$ are opposites.}%TODO: write better.

Although these "functors" are sometimes interesting on their own, their full power is unleashed when they are related to other "functors" through "natural transformations". Before doing that, let us investigate how nice $\Hom$ "functors" are. For instance, many $\Hom$ "functors" can be described in simpler terms.
\begin{exmps}\label{exmp:covrep}
	\begin{enumerate}
		\item[]
		\item Let $\terminal = \{*\}$ be the "terminal" "object" in $\catSet$, then what is the action of $\Hombis^{\terminal}$? For any "object" $B$, \[\Hombis^{\terminal}(B) = \Hom_{\catSet}(\terminal, B)\]
		is easy to describe because for any element $b \in B$, there is a unique function $f: \terminal \rightarrow B = * \mapsto b$. Hence, there is an "isomorphism@@CAT" from $\Hombis^{\terminal}(B)$ to $B$ for any $B \in \obj{\mathbf{C}}$, it sends $f$ to $f(*)$ and its inverse sends $b\in B$ to the map $*\mapsto b$. Moreover, these isomorphisms are natural in $B$ because \eqref{diag:H1Set} clearly "commutes" for any $f:B\rightarrow B'$, yielding a "natural isomorphism" $\Hombis^{\terminal} \isoCAT \id_{\mathbf{C}}$.
		\begin{equation}\label{diag:H1Set}
			\begin{tikzcd}
				\Hombis^{\terminal}(B) \arrow[d] \arrow[r, "f\circ (\placeholder)"] & \Hombis^{\terminal}(B') \arrow[d] \\
				B \arrow[u] \arrow[r, "f"']               & B' \arrow[u]     
			\end{tikzcd}
		\end{equation}
		\item Consider again the "terminal" "object" in the "category" $\catGrp$, namely, the "group" $\terminal$ only containing an "identity@@GRP". Then, for any "group" $G$, the set $\Hombis^{\terminal}(G)$ is a singleton because any "homomorphism@@GRP" $f:\terminal\rightarrow G$ must send the "identity@@GRP" to the "identity@@GRP" and no other choice can be made. Therefore, unlike in $\catSet$, $\Hombis^{\terminal}$ is very uninteresting and acts like the "constant functor" $\terminal:\catGrp \rightsquigarrow \catSet$.
		
		\item A better choice of "object" to mimic the behavior of $\id_{\catGrp}$ is the additive "group" $\Z$. Indeed, for any $g\in G$, there is a unique "homomorphism@@GRP" $f:\Z \rightarrow G$ sending $0$ to the "identity@@GRP" and $1$ to $g$.\footnote{Note that $f$ is completely determined by $f(1)$ because $f(n) = f(1)+\stackrel{n}{\cdots}+f(1)$ and $f(0)$ must be the "identity@@GRP".} A very similar argument as above yields a "natural isomorphism" $\Hombis^{\Z} \isoCAT \id_{\catGrp}$.
		\item The "terminal" "object" in $\catCat$ is the "category" $\termcat$ with a single "object" $\bullet$ and no "morphism" other than the "identity". Observe that for any "category" $\mathbf{C}$, a "functor" $\termcat \rightsquigarrow \mathbf{C}$ is just a choice of "object". Therefore, the same argument will show that $\Hombis^{\termcat} \isoCAT \obj{(\placeholder)}$, where $\obj{(\placeholder)}$ sends a "category" to its set\footnote{Recall that $\catCat$ only contains "small" "categories".} of "objects" and a "functor" to its action restricted on "objects".
		
	 	In order to obtain a similar way to extract "morphisms", consider the category $\cattwo$ with two "objects" and a single "morphism" between them. One obtains a "natural isomorphism" $\Hombis^{\cattwo} \isoCAT (\placeholder)_1$.\footnote{You can prove this as we did for $\Hombis^{\termcat} \isoCAT \obj{(\placeholder)}$ or use Example \ref{exmp:simplefunccat}.\ref{exmp:funct2arrow}.}
	\end{enumerate}
\end{exmps}
These examples suggest that "functors" that are "naturally isomorphic" to $\Hom$ "functors" have nice properties,\footnote{In fact, we already know that the $\Hom$ 
"functors" are "continuous" (Theorem \ref{thm:homcont} and Corollary \ref{cor:cohomcont}).} they are said to be "representable". %TODO: write beter.

\begin{defn}[Representable functor]
	\AP A "covariant" functor $F: \mathbf{C} \rightsquigarrow \catSet$ is ""representable"" if there is an "object" $X \in \obj{\mathbf{C}}$ such that $F$ is "naturally isomorphic" to $\Hom_{\mathbf{C}}(X,\placeholder)$. If $F$ is "contravariant", then it is "representable" if it is "naturally isomorphic" to $\Hom_{\mathbf{C}}(\placeholder,X)$.
\end{defn}
\begin{exmps}\label{exmp:contrarep}
	Let us give examples of the "contravariant" kind.
	\begin{enumerate}%TODO: new notation for contravariant power set 2^X
		\item The "contravariant" "powerset" functor $\mPcontr: \catSet \rightsquigarrow \catSet$ sends a set $X$ to its "powerset" $\mP(X)$ and a function $f:X\rightarrow Y$ to the inverse image $f^{-1}:\mP(Y) \rightarrow \mP(X)$. It is common to identify subsets of a given set with functions from this set into $2 = \{0,1\}$. Formally, this is an "isomorphism@@CAT" $\mPcontr(X) \isoCAT \Hombis_{2}(X) = 2^X$ for any $X$, it maps $S \subseteq X$ to the characteristic function $\chi_S$.\footnote{It sends $x \in X$ to $1$ if $x \in S$ and to $0$ otherwise.} In the reverse direction, it sends a function $g:X\rightarrow \{0,1\}$ to $g^{-1}(1)$. It is easy to check that for any $f:X\rightarrow Y$, the "isomorphisms@@CAT" make \eqref{diag:H2pset} "commute", so $\mPcontr \isoCAT \Hombis_{2}$.
		\begin{equation}\label{diag:H2pset}
			\begin{tikzcd}
				\Hombis_{2}(X) \arrow[d] \arrow[r, "f\circ (\placeholder)"] & \Hombis_{2}(Y) \arrow[d] \\
				\mPcontr(X) \arrow[u] \arrow[r, "f^{-1}"']                & \mPcontr(Y) \arrow[u]           
			\end{tikzcd}
		\end{equation}
		\item In functional programming, it is often useful to transform a function taking multiple arguments so that it ends up taking a single argument but outputs another function. For instance, the multiplication function $\textsf{mult}: \textsf{int} \times \textsf{int} \rightarrow \textsf{int}$ that takes two numbers as inputs and outputs their product can be rewritten as $\textsf{multc} : \textsf{int} \rightarrow (\textsf{int} \rightarrow \textsf{int})$. The function $\textsf{multc}$ takes a number as input and outputs a function that outputs the product of its input and the initial input of $\textsf{multc}$. For example $\textsf{multc}(3)$ is a function that outputs $3\cdot n$ when $n$ is the input. This new function $\textsf{multc}$ is said to be the "curried" version of $\textsf{mult}$ in honor of Haskell Curry. This leads to a more general argument in $\catSet$.
		
		Fix two sets $A$ and $B$. The "functor" $\Hom(\placeholder\product A,B)$ maps a set $X$ to $\Hom(X\product A, B)$ and a function $f:X\rightarrow Y$ to the function $(\placeholder) \circ (f \product \id_A)$.\footnote{You can see it as the composition $\Hombis_B \circ (\placeholder\product A)$.} As suggested by the "currying" process for \textsf{mult}, for any set $X$, there is a bijection $\Hom(X\product A,B) \isoCAT \Hom(X, B^A)$. The image of $f:X\product A \rightarrow B$ is denoted $\curry{f}$ and it satisfies $f(x,a) = \curry{f}(x)(a)$ for any $x \in X$ and $a \in A$. It is easy to check that this is a bijection and also that it is "natural" in $X$ because the \eqref{diag:natcurry} "commutes" for any $f:X\rightarrow Y$, so $\Hom(\placeholder\product A,B) \isoCAT \Hom(\placeholder,B^A)$.%TODO: make this about exponential objects.
		\begin{equation}\label{diag:natcurry}
			\begin{tikzcd}
				{\Hom(X\product A,B)} \arrow[d] \arrow[rr, "(\placeholder)\circ (f\product \id_A)"] & & {\Hom(Y\product A,B)} \arrow[d] \\
				{\Hom(X, B^A)} \arrow[u] \arrow[rr, "(\placeholder)\circ f"']           & & {\Hom(Y,B^A)} \arrow[u]      
			\end{tikzcd}
		\end{equation}
	\end{enumerate}
\end{exmps}
In the first item of Examples \ref{exmp:covrep} and \ref{exmp:contrarep}, we made an arbitrary choice of set. That is, we could have taken any singleton in the first case and any set with two elements in the second. More generally, one can show that if $A \isoCAT B$, then $\Hombis_A \isoCAT \Hombis_B$ and $\Hombis^A \isoCAT \Hombis^B$.
\begin{exer}
    Let $A,B \in \obj{\mathbf{C}}$ be "isomorphic@@CAT" "objects". Show that $\Hombis^A \isoCAT \Hombis^B$. "Dually@@CAT", show that $\Hombis_A\isoCAT \Hombis_B$. 
\end{exer}
Surprisingly, the converse is also true and it will follow from the "Yoneda lemma", but we prove it on its own first as a warm-up for the proof of the "lemma@yoneda".
\begin{prop}
	Let $A,B \in \obj{\mathbf{C}}$ be such that $\Hombis^A \isoCAT \Hombis^B$, then $A \isoCAT B$.
\end{prop}
\begin{proof}
	The "natural isomorphism" gives two "natural transformations" $\phi: \Hombis^A \Rightarrow \Hombis^B$ and $\eta: \Hombis^B \Rightarrow \Hombis^A$ such that for any "object" $X \in \obj{\mathbf{C}}$, \[\eta_X \circ \phi_X:\Hombis^A(X)\rightarrow \Hombis^A(X) \quad \text{ and }\quad  \phi_X \circ \eta_X:\Hombis^B(X) \rightarrow \Hombis^B(X)\] are "identities". In order to show $A \isoCAT B$, we will find two "morphisms" $f:B\rightarrow A$ and $g:A\rightarrow B$ such that $f\circ g = \id_A$ and $g\circ f = \id_B$.
	
	First, note that putting $X$ equal to $A$, we get $\eta_A(\phi_A(\id_A)) = \id_A$ and we claim that \[\eta_A(\phi_A(\id_A)) =\phi_A(\id_A) \circ \eta_B(\id_B).\]
	Since $\phi_A(\id_A)$ is a "morphism" from $B$ to $A$, \eqref{diag:proofYonedaprep} "commutes" by "naturality" of $\eta$. The equality then follows, starting with $\id_B \in \Hombis_B(B)$.
	\begin{equation}\label{diag:proofYonedaprep}
		\begin{tikzcd}
			\Hombis_B(A) \arrow[r, "\eta_A"]                                      & \Hombis_A(A)                                       \\
			\Hombis_B(B) \arrow[r, "\eta_B"'] \arrow[u, "\phi_A(\id_A)\circ (\placeholder)"] & \Hombis_A(B) \arrow[u, "\phi_A(\id_A) \circ (\placeholder)"']
		\end{tikzcd}
	\end{equation}

	A "dual@@CAT" argument shows that \[\id_B = \phi_B(\eta_B(\id_B)) =  \eta_B(\id_B) \circ \phi_A(\id_A),\]
	so we can conclude, letting $f = \phi_A(\id_A)$ and $g= \eta_B(\id_B)$, that $A \isoCAT B$.
\end{proof}
For every $A \in \obj{\mathbf{C}}$, there are two "functors" $\Hombis^A$ and $\Hombis_A$, they are "objects" of $[\mathbf{C}, \catSet]$ and $[\op{\mathbf{C}}, \catSet]$ respectively. It is then reasonable to expect that the assignments $A \mapsto \Hombis^A$ and $A \mapsto \Hombis_A$ are "functorial".

\begin{defn}[""Yoneda embeddings""]
	The "contravariant" "embedding@@YON" $\Hombis^{(\placeholder)}: \op{\mathbf{C}} \rightsquigarrow \catFunc{\mathbf{C}}{\catSet}$ sends $A \in \obj{\mathbf{C}}$ to the $\Hom$ "functor" $\Hombis^A$ and a "morphism" $f: A'\rightarrow A$ to the "natural transformation" $\Hombis^f: \Hombis^{A} \Rightarrow \Hombis^{A'}$ defined by $\Hombis^f_B := \Hom_{\mathbf{C}}(f,B) = (\placeholder) \circ f$ for every $B \in \obj{\mathbf{C}}$. The "naturality" of $\Hombis^f$ follows because \eqref{diag:defYonembed} "commutes" (by "associativity") for any $g: B\rightarrow B'$.%TODO: detail in footnote.
	\begin{equation}\label{diag:defYonembed}
		\begin{tikzcd}
			\Hombis^A(B) \arrow[r, "(\placeholder)\circ f"] \arrow[d, "g \circ (\placeholder)"'] & \Hombis^{A'}(B) \arrow[d, "g\circ (\placeholder)"] \\
			\Hombis^A(B') \arrow[r, "(\placeholder)\circ f"']                         & \Hombis^{A'}(B')                       
		\end{tikzcd}
	\end{equation}
	
	The "covariant" "embedding@@YON" $\Hombis_{(\placeholder)}:C \rightsquigarrow \catFunc{\op{\mathbf{C}}}{\catSet}$ sends $B \in \obj{\mathbf{C}}$ to the $\Hom$ "functor" $\Hombis_B$ and a "morphism" $f:B\rightarrow B'$ to the "natural transformation" $\Hombis_f:\Hombis_B \rightarrow \Hombis_{B'}$ defined by $\Hombis_f^A = \Hom_{\mathbf{C}}(A,f) = f\circ (\placeholder)$ for any $A \in \obj{\mathbf{C}}$. "Naturality" follows from a similar argument.%TODO: put it in footnote.
\end{defn}
"Functoriality" is left for the reader to check. The "embeddings@@YON" are called like that because both "functors" are "fully faithful" as will follow from the "Yoneda lemma".%TODO: do it!
%TODO: recall what is happening with the notation.
\section{Yoneda Lemma}
We have understood how an "object" $A \in \obj{\mathbf{C}}$ sees the "category" $\mathbf{C}$ through "representables", but since a "representable" is an "object" of another "category", it is daring to study what "representables" see and how it relates to the "object" it "represents". More formally, what is the "functor" $\Hom_{\catFunc{\mathbf{C}}{\catSet}}(\Hombis^A, \placeholder)$ describing. \AP For simplicity, we denote it $\intro*\Nat(\Hombis^A,\placeholder)$ because, for a "functor" $F:\mathbf{C} \rightsquigarrow \catSet$, $\Nat(\Hombis^A,F)$ is the "collection"\footnote{Even if $\mathbf{C}$ is "locally small", there is no guarantee that $\catFunc{\mathbf{C}}{\catSet}$ is "locally small". Nevertheless, one consequence of the "Yoneda lemma" is that $\Nat(F,G)$ is a set whenever $F$ is "representable".} of "natural transformations" from $\Hombis^A$ to $F$.

The surprising relation that the "Yoneda lemma" describes is that $\Nat(\Hombis^A,F)$ is "isomorphic@@CAT" to $F(A)$ "naturally" in $F$ and $A$. We first show the "isomorphism@@CAT" and then explain the "naturality".%TODO: give more intuition
%TODO: explain naturally
\begin{lem}[Yoneda lemma I]
	For any $A \in \obj{\mathbf{C}}$ and $F: C\rightsquigarrow \catSet$,
	\[\Nat(\Hombis^A, F) \isoCAT F(A).\]
\end{lem} 
\begin{proof}
	Fix $A$ and $F$, let $\phi_{A,F}: \Nat(\Hombis^A, F) \rightarrow F(A)$ be defined by $\alpha \mapsto \alpha_A(\id_A)$ (check that the types match). Let $\eta_{A,F}: F(A) \rightarrow \Nat(\Hombis^A,F)$ send an element $a \in F(A)$ to the "natural transformation" that has "components" $\eta_{A,F}(a)_B: f \mapsto F(f)(a): \Hom_{\mathbf{C}}(A,B) \rightarrow F(B)$ for any $B \in \obj{\mathbf{C}}$. Checking \eqref{diag:Yoneda1} "commutes" for any $g:B\rightarrow B'$ shows that $\eta_{A,F}(a)$ is a "natural transformation".
	\begin{equation}\label{diag:Yoneda1}
		\begin{tikzcd}
			\Hombis^A(B) \arrow[d, "g\circ (\placeholder)"'] \arrow[r, "F(\placeholder)(a)"] & F(B) \arrow[d, "F(g)"] \\
			\Hombis^A(B') \arrow[r, "F(\placeholder)(a)"']                        & F(B')                 
		\end{tikzcd}
	\end{equation}

	We now check that $\phi_{A,F}$ and $\eta_{A,F}$ are inverses. First,
	$(\eta \circ \phi)_{A,F}$ sends $\alpha\in \Nat(\Hombis^A,F)$ to $\eta_{A,F}(\alpha_A(\id_A))$, and at any $B \in \obj{\mathbf{C}}$, we have 
	\begin{align*}
		\eta_{A,F}(\alpha_A(\id_A))_B(f) &= F(f)(\alpha_A(\id_A)&&\mbox{def of $\eta$}\\
		&= \alpha_B(f \circ \id_A) &&\mbox{"naturality" of $\alpha$}\\
		&= \alpha_B(f),
	\end{align*}
	thus $\alpha = (\eta \circ \phi)_{A,F}(\alpha)$.
	
	Conversely, $(\phi\circ \eta)_{A,F}$ sends $a \in F(A)$ to $\eta_{A,F}(a)_A(\id_A) = F(\id_A)(a) = a$.
	
	We conclude that $\eta_{A,F}$ and $\phi_{A,F}$ are inverses.
\end{proof}
What this results first tells us is that $\Nat(\Hombis^A, F)$ is a set (because it is "isomorphic@@CAT" to $F(A)$ which is a set). This lets us define two new "functors" to understand the second part of the "Yoneda lemma".

The assignment $(A,F) \mapsto \Nat(\Hombis^A,F)$ is a "functor" $\mathbf{C} \cattimes \catFunc{\mathbf{C}}{\catSet} \rightsquigarrow \catSet$. We denote it $\Nat(\Hombis^{(\placeholder)}, \placeholder)$, it sends a "morphism" $(g,\mu): (A,F) \rightarrow (A',F')$ to $\mu \cdot (\placeholder) \cdot \Hombis^g:\Nat(\Hombis^A,F) \rightarrow \Nat(\Hombis^{A'},F')$.\footnote{As $\Nat(\placeholder, \placeholder)$ is the $\Hom$ "bifunctor@hombif" of $\catFunc{\mathbf{C}}{\catSet}$, we can see $\Nat(\Hombis^{(\placeholder)}, \placeholder)$ as the "composition" \[\Nat(\placeholder, \placeholder) \circ (\Hombis^{(\placeholder)} \functimes \id_{\catFunc{\mathbf{C}}{\catSet}}).\]} %TODO: describe the composition. and show it is a funcotr%TODO: define it as Hom(\placeholder,\placeholder) functor.

The assignment $(A,F) \mapsto F(A)$ is another "functor" of the same type. We denote it $\intro*\Ev$ (for evaluation), it sends a "morphism" $(g,\mu): (A,F) \rightarrow (A',F')$ to $F'(g) \circ \mu_A :F(A) \rightarrow F'(A')$.%TODO: show it is a functor

\begin{lem}[Yoneda lemma II]
	There is a "natural isomorphism" $\Nat(\Hombis^{(\placeholder)}, \placeholder) \isoCAT \Ev$.
\end{lem}
\begin{proof}
	The "components" of this "isomorphism@@CAT" are the ones described in the first part of the result. It remains to show that $\phi$ is "natural" in $(A,F)$. For any $(g, \mu): (A,F) \rightarrow (A',F')$, we need to show the following square "commutes".
	\begin{equation}
		\begin{tikzcd}
			{\Nat(\Hombis^A,F)} \arrow[d, "\mu \cdot (\placeholder) \cdot \Hombis^g"'] \arrow[r, "{\phi_{A,F}}"] & F(A) \arrow[d, "F'(g) \circ \mu_A"] \\
			{\Nat(\Hombis^{A'}, F')} \arrow[r, "{\phi_{A',F'}}"']                                  & F'(A')                            
		\end{tikzcd}
	\end{equation}
	Starting with a "natural transformation" $\alpha \in \Nat(\Hombis^A,F)$ the lower path sends it to $(\mu\cdot \alpha \cdot \Hombis^g)_{A'}(\id_{A'})$ and the upper path sends it to $(F'(g) \circ \mu_A)(\alpha_A(\id_A))$. The following derivation shows they are equal.
	\begin{align*}
		(\mu\cdot \alpha \cdot \Hombis^g)_{A'}(\id_{A'}) &= (\mu_{A'}\circ \alpha_{A'})(\Hombis^g_{A'}(\id_{A'}))&&\mbox{def of composition}\\
		&= (\mu_{A'}\circ \alpha_{A'})(g)&&\mbox{def of $\Hombis^g_{A'}$}\\
		&= (\mu_{A'}\circ \alpha_{A'})(\Hombis^A_g(\id_A))&&\mbox{def of $\Hombis^A_g$}\\
		&= (\mu_{A'}\circ \alpha_{A'} \circ \Hombis^A_g)(\id_A)\\
		&= (\mu_{A'} \circ F(g) \circ \alpha_A)(\id_A)&&\mbox{naturality of $\alpha$}\\
		&=(F'(g) \circ \mu_A)(\alpha_A(\id_A)) &&\mbox{naturality of $\mu$}
	\end{align*}
\end{proof}

\begin{cor}
	The "Yoneda embeddings" $\Hombis^{(\placeholder)}$ and $\Hombis_{(\placeholder)}$ are "fully faithful".
\end{cor}
\begin{proof}
	Left as an exercise.%TODO: do it!
\end{proof}

\begin{exmp}[Cayley's theorem with the Yoneda lemma]
	Cayley's theorem states that any "group" is "isomorphic@@GRP" to the "subgroup" of a "permutation group". We will use the "Yoneda lemma" to show that.
	
	Recall the first part of the "Yoneda lemma" which states that for a category $\mathbf{C}$, a "functor" $F:\mathbf{C} \rightsquigarrow \catSet$ and an object $A\in \obj{\mathbf{C}}$, we have \[\Nat(\Hom(A, \placeholder), F) \isoCAT F(A).\]
    Moreover, we know the explicit maps, namely, a "natural transformation" $\phi$ in the L.H.S. is mapped to $\phi_A(\id_A)$ and an element $u \in F(A)$ is mapped to the "natural transformation" $\{\phi_B = f \mapsto F(f)(u) \mid B \in \obj{\mathbf{C}}\}$.%TODO:explain the notation.
	
	Let us apply this to $\mathbf{C}$ being the "delooping" of $G$. Recall that any "functor" $F: \deloop{G} \rightsquigarrow \catSet$ sends $\deloopobject$ to a set $S$ and any $g \in G$ to a "permutation" of $S$, it corresponds to an "action@@GRP" of $G$ on $S$.
	
	To use the "Yoneda lemma", our only choice of "object" for $A$ is $\deloopobject$ and we will choose for $F$ the "functor" it "represents", i.e.: $F=\Hom(\deloopobject, \placeholder)$. The "Yoneda lemma" yields
	\[\Nat(\Hom(\deloopobject, \placeholder), \Hom(\deloopobject,\placeholder)) \isoCAT \Hom(\deloopobject, \deloopobject).\]
	We already know what the R.H.S. is $G$,\footnote{By definition of $\deloop{G}$.} but we have to do a bit of work to understand the L.H.S. First, observe that a "natural transformation" $\phi: \Hom(\deloopobject, \placeholder) \Rightarrow \Hom(\deloopobject, \placeholder)$ is just one "morphism" $\phi_{\deloopobject}: \Hom(\deloopobject, \deloopobject) \rightarrow \Hom(\deloopobject, \deloopobject)$. Namely, it is a map from $G$ to $G$. Second, recalling that $\Hom(\deloopobject, g) = g \circ (\placeholder)$ and that $\deloopobject$ is the only object in $\obj{\mathbf{C}}$, we get that $\phi_{\deloopobject}$ must only make \eqref{diag:cayleyYoneda} "commute".
	\begin{equation}\label{diag:cayleyYoneda}
		\begin{tikzcd}
			G \arrow[d, "g \circ(\placeholder)"'] \arrow[r, "\phi_{\deloopobject}"] & G \arrow[d, "g\circ (\placeholder)"] \\
			G \arrow[r, "\phi_{\deloopobject}"'] & G
		\end{tikzcd}
	\end{equation}
	This is equivalent to $\phi_{\deloopobject}(g \cdot h) = g \cdot \phi_{\deloopobject}(h)$, and we get that each $\phi_{\deloopobject}$ is a $G$--"equivariant" map. Denote the set of $G$--"equivariant" maps $\Hom_G(G,G)$. We obtain that, as sets,
	\[\Hom_G(G,G) \isoCAT G.\]
	Now, we can check that $\Hom_G(G,G)$ is a "subgroup" of $\Perm_G$ (the "group" of "permutations" of the set $G$) and that the bijection is in fact an "group isomorphism". Cayley's theorem follows.
	
	To check that $\Hom_G(G,G) < \Perm_G$, we have to show that $\id_G$ is $G$--"equivariant", that $G$--"equivariant" maps are bijective and that they are stable under composition and taking inverse. First, we have $\id_G(g\cdot h) = g \cdot h = g \cdot \id_G(h)$, so $\id_G \in \Hom_G(G,G)$. Second, let $f$ be a $G$--"equivariant" map. For any $g\in G$, we have $f(g) = f(g\cdot 1) = g \cdot f(1)$. Thus, $f$ is determined only by where it sends the "identity@@GRP". Additionally, sice for any choice of $f(1)$, $g \cdot f(1)$ ranges over $G$ when $g$ ranges over $G$, $f$ is bijective. Therefore, if $f$ and $f'$ are both $G$--"equivariant" map, then \[(f\circ f')(g\cdot h) = f(f'(g\cdot h)) = f(g \cdot f'(h)) = g\cdot (f\circ f')(h),\]
	hence $f\circ f'$ is $G$--"equivariant". Finally, $f^{-1}$ is the $G$--"equivariant" map sending $1$ to $f(1)^{-1}$ and we conclude that $\Hom_G(G,G)$ is a "subgroup" of $\Perm_G$.
	
	The final check is that the "Yoneda" bijection $G\rightarrow \Hom_G(G,G)$ sending $g$ to $(\placeholder)\cdot g$ is a "group homomorphism".\footnote{"isomorphism@@GRP" follows because it is a bijection.} It is clear that it sends the "identity@@GRP" to the "identity@@GRP" and for any $g, h \in G$
	$$(\placeholder)\cdot gh = ((\placeholder) \cdot g)\cdot h = ((\placeholder)\cdot h) \circ ((\placeholder)\cdot g),$$ so this is a "group homomorphism". 
\end{exmp}

\section{Universality as Representability}
"Representability" is one of the two ways to describe "universal" constructions that we hinted at at the end of Chapter \ref{chap:universal}. In this section, we will explore how any "universal property" is equivalent to "representability" of some "functor". Since "(co)@colimits""limits" and "universal morphisms" are "initial" or "terminal" "objects" in some "category", there is a first trivial way to express "universality" as "representability".
\begin{exer}[\NOW]\label{exer:yoneda:initisrepr1}\marginnote{\hyperref[soln:yoneda:initisrepr1]{See solution.}}
	Let $\mathbf{C}$ be a "category", $X \in \obj{\mathbf{C}}$ and $\terminal: \mathbf{C} \rightsquigarrow \catSet$ be the "constant functor" at the singleton $\terminal = \{\star\}$. Show that $\Hom_{\mathbf{C}}(X,\placeholder) \isoCAT \terminal$ if and only if $X$ is "initial". Dually, $\Hom_{\mathbf{C}}(\placeholder,X) \isoCAT \terminal$ if and only if $X$ is "terminal".\footnote{In the dual statement, the domain of $\terminal$ is $\op{\mathbf{C}}$.}
\end{exer}
It turns out this result is not very useful.%TODO: explain why not useful.
% This result is not completely satisfactory because it is a bit too close to a tautological construction. Indeed, let $F$ be a "discrete" "diagram" picking two "objects" $X,Y \in \obj{\mathbf{C}}$. The result above says that $P$ is the "limit" of $F$ (i.e.: the "product@binary product" of $X$ and $Y$) if and only if $\Hom_{\Cone(F)}()$ it says that $X\product Y$ is the "product@binary product" of $X$ and $Y$ using this need to construct the right "category" $\mathbf{C}$ and then 

%TODO: translate definitions to repr
%TODO: examples, terminal, product, exponentials
%TODO: prove that limits are taken pointwise. https://math.stackexchange.com/questions/4050547/limits-in-functor-categories

\begin{prop}
	Let $X,Y\in \obj{\mathbf{C}}$. The "product@binary product" of $X$ and $Y$ exists if and only if there exists $P \in \obj{\mathbf{C}}$ such that $\Hom_{\mathbf{C}\cattimes \mathbf{C}}(\diagFunc_{\mathbf{C}}(\placeholder),(X,Y)) \isoCAT \Hom_{\mathbf{C}}(\placeholder,P)$. The "product@binary" is $P$. %TODO: better way to say this.
\end{prop}
\begin{proof}
	($\Rightarrow$) Let $P = X \product Y$, for any $A \in \obj{\mathbf{C}}$, there is an "isomorphism@@CAT"
	\[\Hom_{\mathbf{C}\cattimes \mathbf{C}}((A,A),(X,Y)) \isoCAT \Hom_{\mathbf{C}}(A,X\product Y)\]
	which sends the pair $(f:A \rightarrow X, g: A \rightarrow Y)$ to $(f,g): A \rightarrow X\product Y$.\footnote{Recall that $(f,g)$ is the unique "morphism" satisfying $\projection_X \circ (f,g) = f$ and $\projection_Y \circ (f,g) = g$. Be careful not to confuse it with a pair of "morphisms".} In the other direction, $p: A \rightarrow X\product Y$ is sent to the pair $(\projection_X \circ p, \projection_Y \circ p)$. Let us show it is "natural" in $A$. For any $m: A' \rightarrow A$, \eqref{diag:productreprnatural} "commutes" because the top path sends the pair $(f,g)$ to the "morphism" $(f,g)$ then to $(f,g) \circ m = (f \circ m, g \circ m)$ and the bottom path sends $(f,g)$ to $(f,g) \circ (m,m) = (f \circ m, g \circ m)$ which is then sent to $(f \circ m, g \circ m)$. %TODO: change the way we display pairs of functions.
	\begin{equation}\label{diag:productreprnatural}%TODO: exercise for composition of product morphism
		% https://q.uiver.app/?q=WzAsNCxbMCwwLCJcXEhvbV97XFxtYXRoYmZ7Q31cXGNhdHRpbWVzIFxcbWF0aGJme0N9fSgoQSxBKSwoWCxZKSkiXSxbMCwxLCJcXEhvbV97XFxtYXRoYmZ7Q31cXGNhdHRpbWVzIFxcbWF0aGJme0N9fSgoQScsQScpLChYLFkpKSJdLFsxLDAsIlxcSG9tX3tcXG1hdGhiZntDfX0oQSxYXFxwcm9kdWN0IFkpIl0sWzEsMSwiXFxIb21fe1xcbWF0aGJme0N9fShBJyxYXFxwcm9kdWN0IFkpIl0sWzAsMiwiXFxzaW0iXSxbMCwxLCJcXHBsYWNlaG9sZGVyIFxcY2lyYyAobSxtKSIsMl0sWzEsMywiXFxzaW0iLDJdLFsyLDMsIlxccGxhY2Vob2xkZXIgXFxjaXJjIG0iXV0=
		\begin{tikzcd}
			{\Hom_{\mathbf{C}\cattimes \mathbf{C}}((A,A),(X,Y))} & {\Hom_{\mathbf{C}}(A,X\product Y)} \\
			{\Hom_{\mathbf{C}\cattimes \mathbf{C}}((A',A'),(X,Y))} & {\Hom_{\mathbf{C}}(A',X\product Y)}
			\arrow["\sim", from=1-1, to=1-2]
			\arrow["{\placeholder \circ (m,m)}"', from=1-1, to=2-1]
			\arrow["\sim"', from=2-1, to=2-2]
			\arrow["{\placeholder \circ m}", from=1-2, to=2-2]
		\end{tikzcd}
	\end{equation}

	($\Leftarrow$) First, we define $\projection_X$ and $\projection_Y$ to be the pair of "morphisms" corresponding to $\id_P$ under the "isomorphism@@CAT" $\Hom_{\mathbf{C}\cattimes \mathbf{C}}((P,P),(X,Y)) \isoCAT \Hom_{\mathbf{C}}(P,P)$. Given two "morphism" $f: A \rightarrow X$ and $g: A \rightarrow Y$, the "isomorpshism@@CAT" 
	\[\Hom_{\mathbf{C}\cattimes \mathbf{C}}((A,A),(X,Y)) \isoCAT \Hom_{\mathbf{C}}(A,P)\]
	yields a unique "morphism" $!: A \rightarrow P$. To see that $\projection_X \circ {!} = f$ and $\projection_Y \circ {!} = g$ we start with $\id_P$ in the top right of \eqref{diag:productreprnaturaltwo} which "commutes" by hypothesis.
	\begin{marginfigure}\begin{equation}\label{diag:productreprnaturaltwo}
		\begin{tikzcd}
			{\Hom_{\mathbf{C}\cattimes \mathbf{C}}((P,P),(X,Y))} & {\Hom_{\mathbf{C}}(P,P)} \\
			{\Hom_{\mathbf{C}\cattimes \mathbf{C}}((A,A),(X,Y))} & {\Hom_{\mathbf{C}}(A,P)}
			\arrow["\sim"', from=1-2, to=1-1]
			\arrow["{\placeholder \circ (!,!)}"', from=1-1, to=2-1]
			\arrow["\sim", from=2-2, to=2-1]
			\arrow["{\placeholder \circ {!}}", from=1-2, to=2-2]
		\end{tikzcd}
	\end{equation}\end{marginfigure}
\end{proof}
\begin{cor}["Dual@@CAT"]
	Let $X,Y\in \obj{\mathbf{C}}$. The "coproduct" of $X$ and $Y$ exists if and only if there exists $S \in \obj{\mathbf{C}}$ such that $\Hom_{\mathbf{C}\cattimes \mathbf{C}}((X,Y), \diagFunc_{\mathbf{C}}(\placeholder)) \isoCAT \Hom_{\mathbf{C}}(S,\placeholder)$. The "coproduct" is $S$.\footnote{We implicitly use the fact that $\op{(\mathbf{C}\cattimes \mathbf{C})} \isoCAT \op{\mathbf{C}} \cattimes \op{\mathbf{C}}$.}
\end{cor}
In order to generalize these two results to arbitrary "(co)@colimits""limits", we defined the generalized version of $\diagFunc_{\mathbf{C}}$.
\begin{defn}[Generalized diagonal functor]
	\AP Let $\mathbf{J}$ and $\mathbf{C}$ be "categories" the ""generalized diagonal functor"" $\gdiagFunc_{\mathbf{C}}^{\mathbf{J}}: \mathbf{C} \rightsquigarrow \catFunc{\mathbf{J}}{\mathbf{C}}$ sends an "object" $X \in \obj{\mathbf{C}}$ to the "constant functor" at $X$ and a "morphism" $f: X \rightarrow Y \in \mor{\mathbf{C}}$ to the "natural transformation" whose components are all $f: X \rightarrow Y$.\begin{marginfigure}[-1\baselineskip]We have $\gdiagFunc_{\mathbf{C}}^{\mathbf{J}}(f) : X \Rightarrow Y$ because for any $j \in \mor{\mathbf{J}}$, we have \[\begin{tikzcd}
		X & X \\
		Y & Y
		\arrow["f"', from=1-1, to=2-1]
		\arrow["{X(j)= \id_X}", from=1-1, to=1-2]
		\arrow["f", from=1-2, to=2-2]
		\arrow["{Y(j)=\id_Y}"', from=2-1, to=2-2]
	\end{tikzcd}\]\end{marginfigure}
\end{defn}
\begin{rem}
	This is a generalization of the "diagonal functor" $\diagFunc_{\mathbf{C}}: \mathbf{C} \rightsquigarrow \mathbf{C} \cattimes \mathbf{C}$ because with the "isomorphism@@CAT" $\catFunc{\termcat\coproduct\termcat}{\mathbf{C}} \isoCAT \mathbf{C} \cattimes \mathbf{C}$ described in Example \ref{exmp:simplefunccat}.\ref{exmp:isoprodfunccat}, we can identify $\diagFunc_{\mathbf{C}}$ with $\gdiagFunc_{\mathbf{C}}^{\termcat\coproduct\termcat}$.
\end{rem}
\begin{prop}\label{prop:limitrepr}
	Let $F: \mathbf{J} \rightsquigarrow \mathbf{C}$ be a "diagram". The "limit" of $F$ exists if and only if there is an object $L \in \obj{\mathbf{C}}$ such that $\Nat(\gdiagFunc_{\mathbf{C}}^{\mathbf{J}}(\placeholder),F) \isoCAT \Hom_{\mathbf{C}}(\placeholder,L)$.\footnote{Recall that 
	\[\Nat(\gdiagFunc_{\mathbf{C}}^{\mathbf{J}}(\placeholder),F) = \Nat(\placeholder,F) \circ \gdiagFunc_{\mathbf{C}}^{\mathbf{J}}.\]} The "tip" of the "limit" "cone" is $L$.
\end{prop}
\begin{proof}
	First, we note that for any $X \in \obj{\mathbf{C}}$, a "natural transformation" $\psi: \gdiagFunc_{\mathbf{C}}^{\mathbf{J}}(X) \Rightarrow F$ is a "cone over" $F$ with "tip" $X$. Indeed, for any $j: A \rightarrow B \in \mor{\mathbf{J}}$, the "naturality" square in \eqref{diag:natiscone} is "commutative".
	\begin{equation}\label{diag:natiscone}
		% https://q.uiver.app/?q=WzAsNCxbMCwwLCJYIl0sWzAsMSwiRkEiXSxbMSwwLCJYIl0sWzEsMSwiRkIiXSxbMCwxLCJcXHBzaV9YIiwyXSxbMCwyLCJYKGopPVxcaWRfWCJdLFsyLDMsIlxccHNpX1giXSxbMSwzLCJGKGopIiwyXV0=
	\begin{tikzcd}
		X & X \\
		FA & FB
		\arrow["{\psi_A}"', from=1-1, to=2-1]
		\arrow["{X(j)=\id_X}", from=1-1, to=1-2]
		\arrow["{\psi_B}", from=1-2, to=2-2]
		\arrow["{F(j)}"', from=2-1, to=2-2]
	\end{tikzcd}
	\end{equation}
	This is equivalent to $\{\psi_A: X \rightarrow FA\}_{A \in \obj{\mathbf{J}}}$ being a "cone over" $F$. Furthermore, a "morphism" of "cones" $\phi \rightarrow \psi$ is a "morphism" $f$ between the "tips" such that $\forall A \in \obj{\mathbf{J}}, \phi_A = \psi_A \circ f$. By looking at \eqref{diag:morphconediagonal}, we see this condition is equivalent to $\phi = \psi \circ \gdiagFunc_{\mathbf{C}}^{\mathbf{J}}(f)$.
	\begin{marginfigure}[-2\baselineskip]
		\begin{equation}\label{diag:morphconediagonal}
		% https://q.uiver.app/?q=WzAsNixbMCwxLCJYIl0sWzEsMiwiRkEiXSxbMiwxLCJYIl0sWzMsMiwiRkIiXSxbMSwwLCJZIl0sWzMsMCwiWSJdLFswLDEsIlxccHNpX0EiLDJdLFsyLDMsIlxccHNpX0IiLDJdLFsxLDMsIkYoaikiLDJdLFs0LDEsIlxccGhpX0EiLDIseyJsYWJlbF9wb3NpdGlvbiI6NzB9XSxbNSwzLCJcXHBoaV9CIl0sWzQsNSwiXFxpZF9ZIiwxXSxbNCwwLCJmIiwxXSxbNSwyLCJmIiwxXSxbMCwyLCJcXGlkX1giLDEseyJsYWJlbF9wb3NpdGlvbiI6NzB9XV0=
		\begin{tikzcd}
			& Y && Y \\
			X && X \\
			& FA && FB
			\arrow["{\psi_A}"', from=2-1, to=3-2]
			\arrow["{\psi_B}"', from=2-3, to=3-4]
			\arrow["{F(j)}"', from=3-2, to=3-4]
			\arrow["{\phi_A}"'{pos=0.7}, from=1-2, to=3-2]
			\arrow["{\phi_B}", from=1-4, to=3-4]
			\arrow["{\id_Y}"{description}, from=1-2, to=1-4]
			\arrow["f"{description}, from=1-2, to=2-1]
			\arrow["f"{description}, from=1-4, to=2-3]
			\arrow["{\id_X}"{description, pos=0.7}, from=2-1, to=2-3]
		\end{tikzcd}
		\end{equation}
	\end{marginfigure}
	($\Rightarrow$) Let $\{\psi_A: L \rightarrow FA\}_{A \in \obj{\mathbf{J}}}$ be the "terminal" "cone over" $F$ and see it as a "natural transformation" $\psi: \gdiagFunc_{\mathbf{C}}^{\mathbf{J}}(L) \Rightarrow F$. We need to define a "natural isomorphism" $\Nat(\gdiagFunc_{\mathbf{C}}^{\mathbf{J}}(\placeholder),F) \isoCAT \Hom_{\mathbf{C}}(\placeholder,L)$. Similarly to the proofs of the previous section, we will see that we only need to see where $\id_L$ is sent to and the rest of the "natural transformation" will \textit{construct itself}. Our only choice for the "cone" corresponding to $\id_L$ is $\psi$ (it is the only "cone" we know exists).
	\begin{marginfigure}[\baselineskip]
		\begin{equation}\label{diag:natdefiso}
			% https://q.uiver.app/?q=WzAsNCxbMCwwLCJcXE5hdChcXGdkaWFnRnVuY197XFxtYXRoYmZ7Q319XntcXG1hdGhiZntKfX0oTCksRikiXSxbMCwxLCJcXE5hdChcXGdkaWFnRnVuY197XFxtYXRoYmZ7Q319XntcXG1hdGhiZntKfX0oWCksRikiXSxbMSwwLCJcXEhvbV97XFxtYXRoYmZ7Q319KEwsTCkiXSxbMSwxLCJcXEhvbV97XFxtYXRoYmZ7Q319KFgsTCkiXSxbMCwxLCJcXHBsYWNlaG9sZGVyXFxjaXJjIFxcZ2RpYWdGdW5jX3tcXG1hdGhiZntDfX1ee1xcbWF0aGJme0p9fShmKSIsMl0sWzAsMiwiIiwyLHsic3R5bGUiOnsidGFpbCI6eyJuYW1lIjoiYXJyb3doZWFkIn19fV0sWzIsMywiXFxwbGFjZWhvbGRlclxcY2lyYyBmIl0sWzEsMywiIiwwLHsic3R5bGUiOnsidGFpbCI6eyJuYW1lIjoiYXJyb3doZWFkIn19fV1d
		\begin{tikzcd}
			{\Nat(\gdiagFunc_{\mathbf{C}}^{\mathbf{J}}(L),F)} & {\Hom_{\mathbf{C}}(L,L)} \\
			{\Nat(\gdiagFunc_{\mathbf{C}}^{\mathbf{J}}(X),F)} & {\Hom_{\mathbf{C}}(X,L)}
			\arrow["{\placeholder\circ \gdiagFunc_{\mathbf{C}}^{\mathbf{J}}(f)}"', from=1-1, to=2-1]
			\arrow[tail reversed, from=1-1, to=1-2]
			\arrow["{\placeholder\circ f}", from=1-2, to=2-2]
			\arrow[tail reversed, from=2-1, to=2-2]
		\end{tikzcd}
		\end{equation}
	\end{marginfigure}
	Indeed, for any $f: X \rightarrow L$ the "naturality" square in \eqref{diag:natdefiso} means the "cone" corresponding to $f: X \rightarrow L$ is $\{\psi_A \circ f: X \rightarrow FA\}_{A \in \obj{\mathbf{J}}}$ by starting with $\id_L$ in the top right. Now, since $\psi$ is the "terminal" "cone", for any "cone" $\{\phi_A: X \rightarrow FA\}_{A \in \obj{\mathbf{J}}}$, there is a unique "morphism" of "cones" $f: X \rightarrow L$ which satisfies $\forall A \in \obj{\mathbf{J}}, \psi_A \circ f = \phi_A$. We conclude that $f \mapsto \psi \circ  \gdiagFunc_{\mathbf{C}}^{\mathbf{J}}(f)$ is a "natural isomorphism".

	($\Leftarrow$) Let $\psi: \gdiagFunc_{\mathbf{C}}^{\mathbf{J}}(L) \Rightarrow F$ be the "cone" corresponding to $\id_L \in \Hom_{\mathbf{C}}(L,L)$ under the "natural isomorphism", we will show it is "terminal". By the "commutativity" of \eqref{diag:natdefiso} and bijectivity of the horizontal arrows, for any "cone" $\phi: \gdiagFunc_{\mathbf{C}}^{\mathbf{J}}(X) \Rightarrow F$, there is a unique "morphism" $f: X \rightarrow L$ such that $\phi = \psi \circ \gdiagFunc_{\mathbf{C}}^{\mathbf{J}}(f)$. By the first paragraph of the proof, this is the unique "morphism" of "cones" showing $\psi$ is "terminal".
\end{proof}
\begin{cor}["Dual@@CAT"]
	Let $F: \mathbf{J} \rightsquigarrow \mathbf{C}$ be a "diagram". The "colimit" of $F$ exists if and only if there is an object $L \in \obj{\mathbf{C}}$ such that $\Nat(F,\gdiagFunc_{\mathbf{C}}^{\mathbf{J}}(\placeholder)) \isoCAT \Hom_{\mathbf{C}}(L,\placeholder)$. The "tip" of the "colimit" "cone" is $L$.
\end{cor}
%TODO: now do the same thing for the universal arrows.
\begin{prop}
	Let $U: \catMon \rightsquigarrow \catSet$ be the "forgetful functor", $A$ be a set and $\freemon{A}$ be the "free monoid" on $A$, we have $\Hom_{\catSet}(A,U\placeholder) \isoCAT \Hom_{\catMon}(\freemon{A},\placeholder)$.
\end{prop}
\begin{proof}
	We have already shown before Definition \ref{defn:freemon} that sending $h: A \rightarrow M$ to $h^*: \freemon{A} \rightarrow M$ is a bijection.\footnote{In the other direction, $h:\freemon{A} \rightarrow M$ is sent to $U(h) \circ i$ where $i: A \hookrightarrow A^*$ is the inclusion.} Now, we need to show it is "natural" in $M$. For any "monoid homomorphism" $f: M \rightarrow N$, \eqref{diag:freemonrepr} "commutes" (we omitted applications of $U$) because starting with $h: A \rightarrow M$, we have $(f \circ h)^* = f \circ h^*$.\footnote{To check this, let $w = a_1\cdots a_n\in A^*$, we have \begin{align*}
		(f \circ h)^*(w) &= fh(a_1)\cdots fh(a_n)\\&= f(h(a_1)\cdots h(a_n))\\&= f(h(w)).
	\end{align*}}
	\begin{equation}\label{diag:freemonrepr}
		% https://q.uiver.app/?q=WzAsNCxbMCwwLCJcXEhvbV97XFxjYXRTZXR9KEEsTSkiXSxbMCwxLCJcXEhvbV97XFxjYXRTZXR9KEEsTikiXSxbMSwwLCJcXEhvbV97XFxjYXRNb259KFxcZnJlZW1vbntBfSxNKSJdLFsxLDEsIlxcSG9tX3tcXGNhdE1vbn0oXFxmcmVlbW9ue0F9LE4pIl0sWzAsMSwiZiBcXGNpcmMgXFxwbGFjZWhvbGRlciIsMl0sWzAsMiwiXFxzaW0iXSxbMiwzLCJmIFxcY2lyYyBcXHBsYWNlaG9sZGVyIl0sWzEsMywiXFxzaW0iLDJdXQ==
		\begin{tikzcd}
			{\Hom_{\catSet}(A,M)} & {\Hom_{\catMon}(\freemon{A},M)} \\
			{\Hom_{\catSet}(A,N)} & {\Hom_{\catMon}(\freemon{A},N)}
			\arrow["{f \circ \placeholder}"', from=1-1, to=2-1]
			\arrow["\sim", from=1-1, to=1-2]
			\arrow["{f \circ \placeholder}", from=1-2, to=2-2]
			\arrow["\sim"', from=2-1, to=2-2]
		\end{tikzcd}
	\end{equation}
\end{proof}
In the next Proposition, we will generalize this result to see how any "universal morphism" corresponds to some kind of "representability" and we will even give a converse direction. The generalizations of the proof is straightforward, so we suggest you try to get familiar with a specific case in the next exercise.
\begin{exer}\label{exer:yoneda:expobjectrepr}\marginnote{\hyperref[soln:yoneda:expobjectrepr]{See solution.}}
	Let $\mathbf{C}$ be a "category" and $X \in \obj{\mathbf{C}}$ be such that $\placeholder \product X$ is a "functor". An "object" $A \in \obj{\mathbf{C}}$ has an "exponential" $A^X \in \obj{\mathbf{C}}$ if and only if $\Hom_{\mathbf{C}}(\placeholder \product X, A) \isoCAT \Hom_{\mathbf{C}}(\placeholder,A^X)$.
\end{exer} %TODO: solve
\begin{prop}\label{prop:universalrepr}
	Let $F: \mathbf{C} \rightsquigarrow \mathbf{D}$ be a "functor" and $X \in \obj{\mathbf{D}}$. There is a "universal morphism" from $X$ to $F$ if and only if there exists $A \in \obj{\mathbf{C}}$ such that $\Hom_{\mathbf{D}}(X,F\placeholder) \isoCAT \Hom_{\mathbf{C}}(A,\placeholder)$.
\end{prop}
\begin{proof}
	($\Rightarrow$) Let $a: X \rightarrow FA$ be a "universal morphism", by definition, for any $b: X \rightarrow FB$, there is a unique "morphism" $\phi_B(b): A \rightarrow B$ such that $F(\phi_B(b)) \circ a = b$. In the other direction, $\phi_B^{-1}$ sending $f: A \rightarrow B$ to $Ff \circ a$ is the inverse of $\phi_B$.\footnote{We check they are inverses:
	\begin{gather*}
		\phi_B^{-1}(\phi_B(b)) = F(\phi_B(b)) \circ a = b\\
		\phi_B(\phi_B^{-1}(f)) = \phi_B(Ff \circ a) = f.
	\end{gather*}} Let us now check that $\phi_B$ is natural. For any $m: B \rightarrow B'$, \eqref{diag:universalrepr} "commutes" because when starting with $f: A \rightarrow B$ in the top right, the top path sends it to $Ff \circ a$ then to $Fm \circ Ff \circ a$ and the bottom path sends it to $m \circ f$ then to $F(m \circ f) \circ a$.
	\begin{equation}\label{diag:universalrepr}
		% https://q.uiver.app/?q=WzAsNCxbMCwwLCJcXEhvbV97XFxtYXRoYmZ7Q319KFgsRkIpIl0sWzAsMSwiXFxIb21fe1xcbWF0aGJme0N9fShYLEZCJykiXSxbMSwwLCJcXEhvbV97XFxtYXRoYmZ7RH19KEEsQikiXSxbMSwxLCJcXEhvbV97XFxtYXRoYmZ7RH19KEEsQicpIl0sWzAsMSwiRm0gXFxjaXJjIFxccGxhY2Vob2xkZXIiLDJdLFswLDIsIlxcc2ltIl0sWzIsMywibSBcXGNpcmMgXFxwbGFjZWhvbGRlciJdLFsxLDMsIlxcc2ltIiwyXV0=
		\begin{tikzcd}
			{\Hom_{\mathbf{C}}(X,FB)} & {\Hom_{\mathbf{D}}(A,B)} \\
			{\Hom_{\mathbf{C}}(X,FB')} & {\Hom_{\mathbf{D}}(A,B')}
			\arrow["{Fm \circ \placeholder}"', from=1-1, to=2-1]
			\arrow["\sim"', from=1-2, to=1-1]
			\arrow["{m \circ \placeholder}", from=1-2, to=2-2]
			\arrow["\sim", from=2-2, to=2-1]
		\end{tikzcd}
	\end{equation}
	($\Leftarrow$) Let $a: X \rightarrow FA$ be the image of $\id_A:A \rightarrow A$ under the "isomorphism@@CAT" $\Hom_{\mathbf{C}}(X,FA) \isoCAT \Hom_{\mathbf{D}}(A,A)$, we claim that $a$ is a "universal morphism" from $X$ to $F$. Given $b: X \rightarrow FB$, let $\phi_B(b)$ be its image under the "isomorphism@@CAT" $\Hom_{\mathbf{C}}(X,FB) \isoCAT \Hom_{\mathbf{D}}(A,B)$, it satisfies $F(\phi_B(b)) \circ a = b$ because \eqref{diag:universalreprconverse} "commutes" (start with $\id_A$ in the top right corner). The "morphism" $\phi_B(b)$ is unique with this property because any other $f: A \rightarrow B$ is the image of some $b'\neq b$ under $\phi_B$ yielding $Ff \circ a = b'\neq b$.\begin{marginfigure}[-2\baselineskip]\begin{equation}\label{diag:universalreprconverse}
		% https://q.uiver.app/?q=WzAsNCxbMCwxLCJcXEhvbV97XFxtYXRoYmZ7Q319KFgsRkIpIl0sWzAsMCwiXFxIb21fe1xcbWF0aGJme0N9fShYLEZBKSJdLFsxLDEsIlxcSG9tX3tcXG1hdGhiZntEfX0oQSxCKSJdLFsxLDAsIlxcSG9tX3tcXG1hdGhiZntEfX0oQSxBKSJdLFsxLDAsIkYoXFxwaGlfQihiKSkgXFxjaXJjIFxccGxhY2Vob2xkZXIiLDJdLFsyLDAsIlxcc2ltIl0sWzMsMiwiXFxwaGlfQihiKSBcXGNpcmMgXFxwbGFjZWhvbGRlciJdLFszLDEsIlxcc2ltIiwyXV0=
		\begin{tikzcd}
			{\Hom_{\mathbf{C}}(X,FA)} & {\Hom_{\mathbf{D}}(A,A)} \\
			{\Hom_{\mathbf{C}}(X,FB)} & {\Hom_{\mathbf{D}}(A,B)}
			\arrow["{F(\phi_B(b)) \circ \placeholder}"', from=1-1, to=2-1]
			\arrow["\sim", from=2-2, to=2-1]
			\arrow["{\phi_B(b) \circ \placeholder}", from=1-2, to=2-2]
			\arrow["\sim"', from=1-2, to=1-1]
		\end{tikzcd}
	\end{equation}\end{marginfigure}
\end{proof}
\begin{cor}["Dual@@CAT"]\label{prop:universalreprdual}
	Let $F: \mathbf{C} \rightsquigarrow \mathbf{D}$ be a "functor" and $X \in \obj{\mathbf{D}}$. There is a "universal morphism" from $F$ to $X$ if and only if there exists $A \in \obj{\mathbf{C}}$ such that $\Hom_{\mathbf{D}}(F\placeholder,X) \isoCAT \Hom_{\mathbf{C}}(\placeholder,A)$.
\end{cor}
Comparing Propositions \ref{prop:limitrepr} and \ref{prop:universalreprdual} and their "duals@@CAT", we infer that "(co)@colimit""limits" satisfy "universal properties".
\begin{thm}\label{thm:limituniversal}
	Let $F\in \obj{\catFunc{\mathbf{J}}{\mathbf{C}}}$ be a "diagram".
	\begin{itemize}
		\item[-] The "limit" of $F$ exists if and only if there is a "universal morphism" from $\gdiagFunc_{\mathbf{C}}^{\mathbf{J}}$ to $F$.
		\item[-] The "colimit" of $F$ exists if and only if there is a "universal morphism" from $F$ to $\gdiagFunc_{\mathbf{C}}^{\mathbf{J}}$.
	\end{itemize}
\end{thm}
%TODO: can we make this about limit preservation, reflection and creation ? Then prove that Hom preserves limits.
In the next chapter, we will lift these correspondence to a more global version. Namely, we will see how to assemble the "universal morphisms" for all "diagrams" of shape $\mathbf{J}$ into a powerful "object".
\end{document}