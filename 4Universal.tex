\documentclass[main.tex]{subfiles}
\begin{document}
\chapter{Universal Properties}\label{chap:universal}
%TODO: section on freeness, cofreeness, currying, ends.
%TODO: natural numbers object
%TODO: general cantor argument? http://www.lix.polytechnique.fr/Labo/Samuel.Mimram/teaching/INF551/course.pdf
\section{Examples}
\subsection{Free Monoid}
The construction of a \textit{free} object is common to different fields of mathematics and the example we will carry out in $\catMon$ can be carried out in many other "categories" like $\catGrp$, $\catAb$, $\catRing$, $\textbf{Mod}_R$ (we will do this one in the next section). In fact, one way to view this construction comes from the "forgetful" "functor" to $\catSet$ that all these "categories" have in common. In Chapter \ref{chap:adjoints}, we will cover "adjoints" and recover the free constructions from $U$.

We choose $\catMon$ because the concrete characterization of a "free monoid" is the simplest.
\begin{defn}[Classical]
    \AP A "monoid" $M$ is said to be ""free@@MON"" if it can be "presented@@MON" by a set of "generators@@MON" without any "relations", i.e. $M = \langle A \mid \emptyset \rangle$. In this case, $M$ is called the \textbf{"free monoid" on $A$} and denoted $\freemon{A}$.
\end{defn}
It is easy to check that $\freemon{A}$ is the set of finite words with symbols in $A$ with the operation being concatenation and identity being \AP the empty word (denoted $\intro*\emptyword$). In order to give a categorical characterization, we need to look at "homomorphisms@@MON" from or into the "free monoid". Notice that any "homomorphism@@MON" $h^*:\freemon{A} \rightarrow M$ is completely determined by where $h^*$ sends elements of $A$. Indeed, in order to satisfy the "homomorphism@@MON" property, we must have for any $a_1, a_2 \in A$, \[h^*(a_1a_2) = h^*(a_1)\cdot h^*(a_2) \text{ and } h^*(\emptyword) = 1_M.\] In general, the unique "homomorphism@@MON" sending $a \in A$ to $h(a)$ can be defined recursively:
\[h^*(w) = \begin{cases}
    h(a)\cdot h^*(w') &a \in A, w \in \freemon{A}, w = aw'\\
    1_M &w = \emptyword\end{cases}.\]
Now, suppose that a "monoid" $N$ contains $A$ and satisfies the same property, that is for any (set-theoretic) function $h:A \rightarrow M$, there is a unique "homomorphism@@MON" $h^*:N \rightarrow M$ with $h^*(a) = h(a)$. 

%TODO: put subscripts to
If we take $M = \freemon{A}$, and $h: A \rightarrow \freemon{A} = a \mapsto a$, then we get a "homomorphism@@MON" $h_N^*: N \rightarrow \freemon{A}$. Moreover, taking $M = N$ and $i: A \hookrightarrow N$ be the inclusion, the property of $\freemon{A}$ means there is a unique "homomorphism@@MON" $i^*: \freemon{A} \rightarrow N$. Note that $h_N^* \circ i^* : \freemon{A} \rightarrow \freemon{A}$ is a "homomorphism@@MON" satisfying $a \mapsto a$, so it must be the identity by uniqueness. We conclude that $N$ and $\freemon{A}$ are "isomorphic@@MON".
\begin{defn}[Categorical]\label{defn:freemon}
    \AP The "free monoid" of a set $A$ is an object $\freemon{A}$ in $\catMon$ along with a \textit{canonical inclusion} $i: A \rightarrow U(\freemon{A})$ that satisfies the following "universal property": for any "monoid" $M$ and function $h:A \rightarrow U(M)$, there exists a unique "homomorphism@@MON" $h^*: \freemon{A} \rightarrow M$ such that $U(h^*) \circ i = h$, namely, $h^*(i(a)) = h(a)$. This is summarized in \eqref{diag:freemon}, where we omit the $U$ as the underlying set of a "monoid" is often denoted with the same symbol as the "monoid".
    \begin{equation}\label{diag:freemon}
        \begin{tikzcd}
            & A & \freemon{A} & {} & \freemon{A} & {} \\
            && M && M
            \arrow["i", from=1-2, to=1-3]
            \arrow[""{name=0, anchor=center, inner sep=0}, "h^*", dashed, from=1-3, to=2-3]
            \arrow["h"', from=1-2, to=2-3]
            \arrow[""{name=1, anchor=center, inner sep=0}, "h^*", dashed, from=1-5, to=2-5]
            \arrow["{\text{in }\catSet}"{description}, shift left=6, draw=none, from=1-2, to=1-3]
            \arrow["{\text{in }\catMon}"{description}, shift left=6, draw=none, from=1-4, to=1-6]
            \arrow["{\text{\kl[forgetful]{forgetful}}}"', shorten <=6pt, shorten >=10pt, from=1, to=0]
        \end{tikzcd}
    \end{equation}
\end{defn}

\subsection{Abelianization}
\begin{defn}[Classical]
    \AP Let $G$ be a group, the ""abelianization"" of $G$, denoted $\ab{G}$, is the "quotient@@GRP" of $G$ with $G' := \{xyx^{-1}y^{-1} \mid x, y \in G\} \leq G$, called the ""commutator subgroup"", that is $\ab{G} := G/G'$.
\end{defn}
Let us get insight into this definition. The "abelianization" is supposed to be the \textit{biggest} "abelian" "quotient@@GRP" of $G$. To see why, note that if $A$ is an "abelian group", any "homomorphism@@GRP" $h:G \rightarrow A$ must satisfy $h(xyx^{-1}y^{-1}) = 1_A$ for any $x,y\in G$. Hence, $G'$ is contained in the "kernel@GRP" of $h$. This yields a factorization $h = G \stackrel{\pi}{\rightarrow} G/G' \stackrel{h^*}{\rightarrow} A$ with $h^*$ unique, where $\pi$ is the canonical "quotient@@GRP" map.%TODO: maybe define factorization

Moreover, since $\catAb$ is a "full@@CAT" "subcategory" of $\catGrp$, $h^*$ is also unique as a "morphism" in $\catAb$. Using the fact that $G/G'$ is "abelian", we conclude the following categorical definition of $\ab{G}$.
\begin{defn}[Categorical]
    Let $G$ be a group, the "abelianization" of $G$ is an "abelian group" $\ab{G}$ with a map $\pi: G \rightarrow \ab{G}$ satisfying the following "universal property": for any "homomorphism@@GRP" $h:G \rightarrow A$ where $A$ is "abelian", there is a unique "homomorphism@@GRP" $h^*: \ab{G} \rightarrow A$ such that $h^* \circ \pi = h$. This is summarized in \eqref{diag:abelianization}.
    \begin{equation}\label{diag:abelianization}
        \begin{tikzcd}
            & G & \ab{G} & {} & \ab{G} & {} \\
            && A && A
            \arrow["\pi", from=1-2, to=1-3]
            \arrow[""{name=0, anchor=center, inner sep=0}, "h^*", dashed, from=1-3, to=2-3]
            \arrow["h"', from=1-2, to=2-3]
            \arrow[""{name=1, anchor=center, inner sep=0}, "h^*", dashed, from=1-5, to=2-5]
            \arrow["{\text{in }\catGrp}"{description}, shift left=6, draw=none, from=1-2, to=1-3]
            \arrow["{\text{in }\catAb}"{description}, shift left=6, draw=none, from=1-4, to=1-6]
            \arrow["{\text{\kl[forgetful]{forgetful}}}"', shorten <=6pt, shorten >=10pt, from=1, to=0]
        \end{tikzcd}
    \end{equation}
\end{defn}
\subsection{Vector Space Basis}
\begin{defn}[Classical]
    \AP Let $V$ be a "vector space" over a "field" $k$, a ""basis"" for $V$ is a subset $S \subseteq V$ that is "linearly independent" and "generates@@VECT" $V$, namely, any $v \in V$ can be expressed as a "linear combination" of elements in $S$ and any $s \in S$ cannot be expressed as a "linear combination" of elements in $S \setminus\{s\}$.
\end{defn}
Once again, we would like to get rid of the content of this definition talking about elements, so we focus on what this means for "linear maps" coming out of $V$. Let $S$ be a "basis" of $V$, $W$ be another "vector space" over $k$ and $T: V \rightarrow W$ be a "linear map". By "linearity", $T$ is completely determined by where it sends the elements of $S$. Indeed, for any $v \in V$, write $v$ as a "linear combination" $\sum_{s \in S} \lambda_s s$ with $\lambda_s \in k$ (only finitely many of the coefficients are non-zero), then $T(v) = \sum_{s \in S} \lambda_s T(s)$. We conclude that any (set-theoretic) function $t: S \rightarrow W$ extends to a unique "linear map" $T: V \rightarrow  W$.

We claim that this property completely characterizes "bases" of $V$. Indeed, let $S \subseteq V$ be such that for any $t: S \rightarrow W$, there is a unique "linear map" $T: V \rightarrow  W$ extending $t$. We will show that $S$ is "generating@@VECT" and "linearly independent".
\begin{enumerate}
    \item Assume towards a contradiction that $S$ is not "generating@@VECT", that is, there exists $v \in V$ that is not a "linear combination" of vectors in $S$. Equivalently, if $U$ is the "subspace" "generated@@VECT" by $S$, then $V/U$ is not $0$. Now, let $t: S \rightarrow V/U$ be the $0$ map, both the quotient map $\pi: V \rightarrow V/U$ and the $0$ map $0: V \rightarrow V/U$ extend $t$, and since $V/U$ is not trivial, they are different maps.
    \item Assume towards a contradiction that $S$ is not "linearly dependent", that is, there exists $v \in S$ is such that $v = \sum_{s \in S-v} \lambda_s s$. Consider the function \[t: S \rightarrow V \oplus V  = \begin{cases}(s,0) & s\neq v\\ (0,v) & s = v\end{cases}.\]
    There cannot exist a "linear map" $T: V \rightarrow V\oplus V$ extending $t$ because by "linearity", we can show
    \[(0,v) = t(v) = T(v) = T(\sum_{s \in S-v} \lambda_s s) = \sum_{s \in S-v} \lambda_s T(s) = \sum_{s \in S-v} \lambda_s (s,0),\]
    which is absurd.
\end{enumerate}
In conclusion, we have the following alternate definition of a "vector space" "basis".
\begin{defn}[Categorical]
    Let $V$ be a "vector space", a "basis" of $V$ is a set $S$ along with an inclusion $i: S \rightarrow V$ satisfying the following "universal property": for any function $t: S \rightarrow W$ where $W$ is a "vector space", there is a unique "linear map" $T: V \rightarrow W$ such that $T \circ i = t$. This is summarized in \eqref{diag:basis}.
    \begin{equation}\label{diag:basis}
        % https://q.uiver.app/?q=WzAsOSxbMiwxLCJWIl0sWzEsMSwiUyJdLFsyLDIsIlciXSxbNCwyLCJXIl0sWzQsMSwiViJdLFswLDBdLFsxLDBdLFs1LDFdLFszLDFdLFsxLDAsImkiXSxbMCwyLCJUIiwwLHsic3R5bGUiOnsiYm9keSI6eyJuYW1lIjoiZGFzaGVkIn19fV0sWzEsMiwidCIsMl0sWzQsMywiVCIsMCx7InN0eWxlIjp7ImJvZHkiOnsibmFtZSI6ImRhc2hlZCJ9fX1dLFsxLDAsIlxcdGV4dHtpbiB9XFxjYXRTZXQiLDEseyJvZmZzZXQiOi01LCJzdHlsZSI6eyJib2R5Ijp7Im5hbWUiOiJub25lIn0sImhlYWQiOnsibmFtZSI6Im5vbmUifX19XSxbOCw3LCJcXHRleHR7aW4gfVxcY2F0VmVjdHtrfSIsMSx7Im9mZnNldCI6LTUsInN0eWxlIjp7ImJvZHkiOnsibmFtZSI6Im5vbmUifSwiaGVhZCI6eyJuYW1lIjoibm9uZSJ9fX1dLFsxMiwxMCwiXFx0ZXh0e1wiZm9yZ2V0ZnVsXCJ9IiwyLHsic2hvcnRlbiI6eyJzb3VyY2UiOjEwLCJ0YXJnZXQiOjEwfSwibGV2ZWwiOjF9XV0=
    \begin{tikzcd}
        & S & V & {} & V & {} \\
        && W && W
        \arrow["i", from=1-2, to=1-3]
        \arrow[""{name=0, anchor=center, inner sep=0}, "T", dashed, from=1-3, to=2-3]
        \arrow["t"', from=1-2, to=2-3]
        \arrow[""{name=1, anchor=center, inner sep=0}, "T", dashed, from=1-5, to=2-5]
        \arrow["{\text{in }\catSet}"{description}, shift left=6, draw=none, from=1-2, to=1-3]
        \arrow["{\text{in }\catVect{k}}"{description}, shift left=6, draw=none, from=1-4, to=1-6]
        \arrow["{\text{\kl[forgetful]{forgetful}}}"', shorten <=6pt, shorten >=10pt, from=1, to=0]
    \end{tikzcd}
    \end{equation}
\end{defn}

\subsection{Exponential Objects}
This section and the following two are motivated by important constructions in $\catSet$ that we want to define categorically. Going further in this direction amounts to doing \href{https://ncatlab.org/nlab/show/topos}{topos theory}, namely, studying "categories" which look a lot like $\catSet$.
\begin{exer}\label{exer:universal:prodXfunc}\marginnote{\hyperref[soln:universal:prodXfunc]{See solution.}}
    Let $\mathbf{C}$ be a "category" and $X \in \obj{\mathbf{C}}$ be such that for any $Y \in \obj{\mathbf{C}}$, $Y \product X$ exists. Show that $\placeholder\product X$ is a "functor" $\mathbf{C} \rightsquigarrow \mathbf{C}$.
\end{exer}
Let $A$ and $X$ be sets, $A^X$ commonly denotes the set of functions $X \rightarrow A$. In hope to generalize this construction to other "categories", let us study "morphisms" into $A^X$. Given a set $B$ and a "morphism" $f: B \rightarrow A^X$, \AP there is a natural operation called ""uncurrying"" that takes $f$ to $\uncurry{f}:B \times X \rightarrow A$ which basically evaluates both $f$ and its output at the same time. Namely, $\uncurry{f}(b,x) = f(b)(x)$.

As a particular case, we consider the identity function $A^X \rightarrow A^X$. \AP "Uncurrying" yields the ""evaluation"" function $\ev: A^X \times X \rightarrow A$ that evaluates the function in the first coordinate at the second coordinate: $\ev(f,x) = f(x)$.

\AP Now, as the name suggests, "uncurrying" has an inverse operation called ""currying"" which takes $g : B\times X \rightarrow A$ to $\curry{g}: B \rightarrow A^X$ defined by $\curry{g}(b) = x \mapsto g(b,x)$. Morally, $\curry{g}$ delays the evaluation of $g$ to later.\footnote{For computer scientists, this is also related to the concept of \textit{continuations}.} Moreover, notice that the "currying" of $g$ satisfies $\ev(\curry{g}(b), x) = g(b,x) \in A$ for any $b \in B$ and $x \in X$. This along with the fact that "currying" and "uncurrying" are bijective operations\footnote{Check that $\curry{\uncurry{g}} = g$ and $\uncurry{\curry{g}} = g$.} leads to a "universal property" that $\ev$ satisfies. It is summarized in \eqref{diag:exponent}. %TODO: check about continuations.

\begin{equation}\label{diag:exponent}
% https://q.uiver.app/?q=WzAsOSxbMiwxLCJBXlhcXHByb2R1Y3QgWCJdLFsxLDEsIkEiXSxbMiwyLCJCXFxwcm9kdWN0IFgiXSxbNCwyLCJCIl0sWzQsMSwiQV5YIl0sWzAsMF0sWzEsMF0sWzMsMV0sWzUsMV0sWzAsMSwiXFxldiIsMl0sWzIsMCwiXFxjdXJyeXtnfVxccHJvZHVjdG0gXFxpZF9YIiwyLHsic3R5bGUiOnsiYm9keSI6eyJuYW1lIjoiZGFzaGVkIn19fV0sWzIsMSwiZyJdLFszLDQsIlxcY3Vycnl7Z30iLDIseyJzdHlsZSI6eyJib2R5Ijp7Im5hbWUiOiJkYXNoZWQifX19XSxbMSwwLCJcXHRleHR7aW4gfVxcY2F0U2V0IiwxLHsib2Zmc2V0IjotNSwic3R5bGUiOnsiYm9keSI6eyJuYW1lIjoibm9uZSJ9LCJoZWFkIjp7Im5hbWUiOiJub25lIn19fV0sWzcsOCwiXFx0ZXh0e2luIH1cXGNhdFNldCIsMSx7Im9mZnNldCI6LTUsInN0eWxlIjp7ImJvZHkiOnsibmFtZSI6Im5vbmUifSwiaGVhZCI6eyJuYW1lIjoibm9uZSJ9fX1dLFsxMiwxMCwiXFxwbGFjZWhvbGRlclxccHJvZHVjdCBYIiwyLHsibGFiZWxfcG9zaXRpb24iOjQwLCJzaG9ydGVuIjp7InNvdXJjZSI6MTAsInRhcmdldCI6NDB9LCJsZXZlbCI6MX1dXQ==
\begin{tikzcd}
	& A & {A^X\product X} & {} & {A^X} & {} \\
	&& {B\product X} && B
	\arrow["\ev"', from=1-3, to=1-2]
	\arrow[""{name=0, anchor=center, inner sep=0}, "{\curry{g}\productm \id_X}"', dashed, from=2-3, to=1-3]
	\arrow["g", from=2-3, to=1-2]
	\arrow[""{name=1, anchor=center, inner sep=0}, "{\curry{g}}"', dashed, from=2-5, to=1-5]
	\arrow["{\text{in }\catSet}"{description}, shift left=6, draw=none, from=1-2, to=1-3]
	\arrow["{\text{in }\catSet}"{description}, shift left=6, draw=none, from=1-4, to=1-6]
	\arrow["{\placeholder\product X}"'{pos=0.4}, shorten <=8pt, shorten >=32pt, from=1, to=0]
\end{tikzcd}
\end{equation}
This is entirely categorical, so we can define "exponential objects" as follows.
\begin{defn}[Exponential]
    Let $\mathbf{C}$ be a "category" and $X \in \obj{\mathbf{C}}$ be such that $\placeholder\product X$ is a "functor".\footnote{i.e.: all "binary products" with $X \in \obj{\mathbf{C}}$ exist.} For $A \in \obj{\mathbf{C}}$, the ""exponential"" $A^X$ (if it exists) is an "object" $A^X$ along with a "morphism" $\ev: A^X \times X \rightarrow A$ such that for all $g: B\times X \rightarrow A$, there is a unique $\curry{g}:B \rightarrow A^X$ making \eqref{diag:exponent} "commute".
\end{defn}

\subsection{Subobject Classifier}
\begin{exer}\label{exer:universal:subobjfunctor}\marginnote{\hyperref[soln:universal:subobjfunctor]{See solution.}}%TODO: define \pull and pullback along
    Let $\mathbf{C}$ be a "well-powered" "category" with all "pullbacks". We define $\Sub_{\mathbf{C}}$ on "morphisms": it sends $f: X \rightarrow Y$ to $\pull{f}{\placeholder}: \Sub_{\mathbf{C}}(Y) \rightarrow \Sub_{\mathbf{C}}(X)$ sending $m: I \rightarrow Y$ to $\pull{f}{m}$ (the "pullback" of $m$ "along" $f$ as depicted in \eqref{diag:pullbackalongf}). Show that this is well-defined and makes $\Sub_{\mathbf{C}}$ into a "functor" $\op{\mathbf{C}} \rightsquigarrow \mathbf{Set}$.\begin{marginfigure}\begin{equation}\label{diag:pullbackalongf}
        % https://q.uiver.app/?q=WzAsNCxbMCwwLCJKIl0sWzAsMSwiWCJdLFsxLDEsIlkiXSxbMSwwLCJJIl0sWzAsMSwiXFxwdWxse2Z9e1xcaW90YX0iLDIseyJzdHlsZSI6eyJ0YWlsIjp7Im5hbWUiOiJob29rIiwic2lkZSI6InRvcCJ9fX1dLFsxLDIsImYiLDJdLFswLDNdLFszLDIsIlxcaW90YSIsMCx7InN0eWxlIjp7InRhaWwiOnsibmFtZSI6Imhvb2siLCJzaWRlIjoidG9wIn19fV0sWzAsMiwiIiwyLHsic3R5bGUiOnsibmFtZSI6ImNvcm5lciJ9fV1d
\begin{tikzcd}
	J & I \\
	X & Y
	\arrow["{\pull{f}{m}}"', hook, from=1-1, to=2-1]
	\arrow["f"', from=2-1, to=2-2]
	\arrow[from=1-1, to=1-2]
	\arrow["m", hook, from=1-2, to=2-2]
	\arrow["\pullbackd"{anchor=center, pos=0.125}, draw=none, from=1-1, to=2-2]
\end{tikzcd}
    \end{equation}\end{marginfigure}
\end{exer}
In $\catSet$, recall that "subobjects" are subsets. Hence, letting $\Omega=\{\bot,\top\}$ there is a correspondence between $\Sub_{\catSet}(X)$ and $\Hom_{\catSet}(X,\Omega)$, it sends $I \subseteq X$ to the "characteristic" function $\charac_I: X \rightarrow \Omega$,\footnote{The "characteristic" function $\charac_I$ is defined by 
\[\charac_I(x) = \begin{cases}
    \top &x \in I\\\bot & x\notin I
\end{cases}.\]} and in the other direction $f: X \rightarrow \Omega$ is sent to $f^{-1}(\top)\subseteq X$. Furthermore, recall that the preimage can be seen as a "pullback", so we can define $\charac_I$ as the unique function making \eqref{diag:pullbackcharac} into a "pullback" square. Uniqueness holds because this "pullback" implies $I = \charac_I^{-1}(\top)$.%TODO: ref to where we said this.
\begin{equation}\label{diag:pullbackcharac}
    % https://q.uiver.app/?q=WzAsNCxbMCwwLCJJIl0sWzAsMSwiWCJdLFsxLDEsIlxcT21lZ2EiXSxbMSwwLCJcXHRlcm1pbmFsIl0sWzAsMSwiIiwyLHsic3R5bGUiOnsidGFpbCI6eyJuYW1lIjoiaG9vayIsInNpZGUiOiJ0b3AifX19XSxbMSwyLCJcXGNoYXJhY19JIiwyXSxbMCwzXSxbMywyLCJcXHRvcCIsMCx7InN0eWxlIjp7InRhaWwiOnsibmFtZSI6Imhvb2siLCJzaWRlIjoidG9wIn19fV0sWzAsMiwiIiwyLHsic3R5bGUiOnsibmFtZSI6ImNvcm5lciJ9fV1d
    \begin{tikzcd}
        I & \terminal \\
        X & \Omega
        \arrow[hook, from=1-1, to=2-1]
        \arrow["{\charac_I}"', from=2-1, to=2-2]
        \arrow[from=1-1, to=1-2]
        \arrow["\top", hook, from=1-2, to=2-2]
        \arrow["\lrcorner"{anchor=center, pos=0.125}, draw=none, from=1-1, to=2-2]
    \end{tikzcd}
\end{equation}
The role played by the two element set $\{\bot,\top\}$ can now be generalized to other "categories".
\begin{defn}[Subobject classifier]
    \AP Let $\mathbf{C}$ be a "category" with a "terminal" object $\terminal$. A ""subobject classifier"" is a "morphism" $\top: \terminal \rightarrow \Omega \in \mor{\mathbf{C}}$ such that for any "monomorphism" $I \hookrightarrow X$ there is a unique "morphism" $\charac_m: X \rightarrow \Omega$ such that \eqref{diag:pullbackcharac} is a "pullback" square. \AP We call $\charac_I$ the ""characteristic map"" of $I \hookrightarrow X$.
\end{defn}\begin{marginfigure}[6\baselineskip]\begin{equation}\label{diag:classifierwelldef}
    % https://q.uiver.app/?q=WzAsNixbMSwwLCJJIl0sWzEsMSwiWCJdLFsyLDEsIlxcT21lZ2EiXSxbMiwwLCJcXHRlcm1pbmFsIl0sWzAsMCwiSSciXSxbMCwxLCJYIl0sWzAsMSwiIiwyLHsic3R5bGUiOnsidGFpbCI6eyJuYW1lIjoiaG9vayIsInNpZGUiOiJ0b3AifX19XSxbMSwyLCJcXGNoYXJhY19JIiwxXSxbMCwzXSxbMywyLCJcXHRvcCIsMCx7InN0eWxlIjp7InRhaWwiOnsibmFtZSI6Imhvb2siLCJzaWRlIjoidG9wIn19fV0sWzAsMiwiIiwyLHsic3R5bGUiOnsibmFtZSI6ImNvcm5lciJ9fV0sWzQsNSwiIiwwLHsic3R5bGUiOnsidGFpbCI6eyJuYW1lIjoiaG9vayIsInNpZGUiOiJ0b3AifX19XSxbNSwxLCJcXGlkX1giLDFdLFs0LDAsIlxcc2ltIiwwLHsic3R5bGUiOnsidGFpbCI6eyJuYW1lIjoiYXJyb3doZWFkIn19fV0sWzUsMiwiXFxjaGFyYWNfe0knfSIsMix7ImN1cnZlIjo1fV0sWzQsMSwiIiwyLHsic3R5bGUiOnsibmFtZSI6ImNvcm5lciJ9fV1d
\begin{tikzcd}
	{I'} & I & \terminal \\
	X & X & \Omega
	\arrow[hook, from=1-2, to=2-2]
	\arrow["{\charac_I}"{description}, from=2-2, to=2-3]
	\arrow[from=1-2, to=1-3]
	\arrow["\top", hook, from=1-3, to=2-3]
	\arrow["\lrcorner"{anchor=center, pos=0.125}, draw=none, from=1-2, to=2-3]
	\arrow[hook, from=1-1, to=2-1]
	\arrow["{\id_X}"{description}, from=2-1, to=2-2]
	\arrow["\sim", tail reversed, from=1-1, to=1-2]
	\arrow["{\charac_{I'}}"', curve={height=20pt}, from=2-1, to=2-3]
	\arrow["\lrcorner"{anchor=center, pos=0.125}, draw=none, from=1-1, to=2-2]
\end{tikzcd}
\end{equation}\end{marginfigure}
Before drawing a diagram like those above to summarize the "universal property" of a "subobject classifier", we need to make sure that the "characteristic maps" of two "monomorphisms" in the same equivalence class in $\Sub_{\mathbf{C}}(X)$ are equal. Looking at \eqref{diag:classifierwelldef}, the right square is a "pullback" by hypothesis and the left square is a "pullback" by Exercise \ref{exer:limits:isopullback}. Therefore, the rectangle is a "pullback" by the "pasting lemma" and we see that $\charac_{I'} = \charac_I \circ \id_X$ by uniqueness of the "characteristic map".

Now, in a "well-powered" "category" $\mathbf{C}$ has a "terminal" object and all "pullbacks",\footnote{The definition of "subobject classifier" does not need the "well-poweredness" and the existence of all "pullbacks", but they are necessary to have a "universal property" because it uses the "functor" $\Sub_{\mathbf{C}}$. In any case, "subobject classifiers" are usually used when these conditions are satisfied.} a "subobject classifier" $\top: \terminal \rightarrow \Omega$ is such that for any "subobject" $m$ of $X$, which we identify as a "morphism" $m: \terminal \rightarrow \Sub_{\mathbf{C}}(X)$, there is a unique "morphism" $\charac_{m}: X \rightarrow \Omega$ such that $\pull{\charac_m}{\top} = m$. This is summarized in \eqref{diag:upsubobjectclassifier}.%TODO: write footnote better.
\marginnote[2\baselineskip]{Notice that the dashed arrow gets reversed because $\Sub_{\mathbf{C}}$ is "contravariant". We could also write ``in $\op{\mathbf{C}}$'' and not reverse the arrow.}
\begin{equation}\label{diag:upsubobjectclassifier}
    % https://q.uiver.app/?q=WzAsNyxbMSwwLCJcXFN1Yl97XFxtYXRoYmZ7Q319KFxcT21lZ2EpIl0sWzAsMCwiXFx0ZXJtaW5hbCJdLFsxLDEsIlxcU3ViX3tcXG1hdGhiZntDfX0oWCkiXSxbMywxLCJYIl0sWzMsMCwiXFxPbWVnYSJdLFsyLDBdLFs0LDBdLFsxLDAsIlxcdG9wIl0sWzEsMiwiXFxpb3RhIiwyXSxbMyw0LCJcXGNoYXJhY197XFxpb3RhfSIsMix7InN0eWxlIjp7ImJvZHkiOnsibmFtZSI6ImRhc2hlZCJ9fX1dLFs1LDYsIlxcdGV4dHtpbiB9XFxtYXRoYmZ7Q30iLDEseyJvZmZzZXQiOi01LCJzdHlsZSI6eyJib2R5Ijp7Im5hbWUiOiJub25lIn0sImhlYWQiOnsibmFtZSI6Im5vbmUifX19XSxbMSwwLCJcXHRleHR7aW4gfVxcY2F0U2V0IiwxLHsib2Zmc2V0IjotNSwic3R5bGUiOnsiYm9keSI6eyJuYW1lIjoibm9uZSJ9LCJoZWFkIjp7Im5hbWUiOiJub25lIn19fV0sWzAsMiwiXFxwdWxse1xcY2hhcmFjX3tcXGlvdGF9fXtcXHBsYWNlaG9sZGVyfSIsMCx7InN0eWxlIjp7ImJvZHkiOnsibmFtZSI6ImRhc2hlZCJ9fX1dLFs5LDEyLCJcXFN1Yl97XFxtYXRoYmZ7Q319IiwyLHsibGFiZWxfcG9zaXRpb24iOjQwLCJzaG9ydGVuIjp7InNvdXJjZSI6MTAsInRhcmdldCI6NDB9LCJsZXZlbCI6MX1dXQ==
    \begin{tikzcd}
        \terminal & {\Sub_{\mathbf{C}}(\Omega)} & {} & \Omega & {} \\
        & {\Sub_{\mathbf{C}}(X)} && X
        \arrow["\top", from=1-1, to=1-2]
        \arrow["m"', from=1-1, to=2-2]
        \arrow[""{name=0, anchor=center, inner sep=0}, "{\charac_{m}}"', dashed, from=2-4, to=1-4]
        \arrow["{\text{in }\mathbf{C}}"{description}, shift left=6, draw=none, from=1-3, to=1-5]
        \arrow["{\text{in }\catSet}"{description}, shift left=6, draw=none, from=1-1, to=1-2]
        \arrow[""{name=1, anchor=center, inner sep=0}, "{\pull{\charac_{m}}{\placeholder}}", dashed, from=1-2, to=2-2]
        \arrow["{\Sub_{\mathbf{C}}}"'{pos=0.4}, shorten <=7pt, shorten >=29pt, from=0, to=1]
    \end{tikzcd}
\end{equation}

\subsection{Power Objects}
Let $X$ be a set, the "powerset" of $X$, $\mPcov X$ is the set of all subsets of $X$. 
\begin{equation}\label{diag:uppowerobject}
    % https://q.uiver.app/?q=WzAsNyxbMSwwLCJcXFN1Yl97XFxtYXRoYmZ7Q319KFxcUG93ZXIgWFxccHJvZHVjdCBYKSJdLFswLDAsIlxcdGVybWluYWwiXSxbMSwxLCJcXFN1Yl97XFxtYXRoYmZ7Q319KFlcXHByb2R1Y3QgWCkiXSxbMywxLCJZIl0sWzMsMCwiXFxQb3dlciBYIl0sWzIsMF0sWzQsMF0sWzEsMCwiXFxuaV9BIl0sWzEsMiwiXFxpb3RhIiwyXSxbMyw0LCJcXGNoYXJhY197XFxpb3RhfSIsMix7InN0eWxlIjp7ImJvZHkiOnsibmFtZSI6ImRhc2hlZCJ9fX1dLFs1LDYsIlxcdGV4dHtpbiB9XFxtYXRoYmZ7Q30iLDEseyJvZmZzZXQiOi01LCJzdHlsZSI6eyJib2R5Ijp7Im5hbWUiOiJub25lIn0sImhlYWQiOnsibmFtZSI6Im5vbmUifX19XSxbMSwwLCJcXHRleHR7aW4gfVxcY2F0U2V0IiwxLHsib2Zmc2V0IjotNSwic3R5bGUiOnsiYm9keSI6eyJuYW1lIjoibm9uZSJ9LCJoZWFkIjp7Im5hbWUiOiJub25lIn19fV0sWzAsMiwiXFxwdWxse1xcY2hhcmFjX3tcXGlvdGF9fXtcXHBsYWNlaG9sZGVyXFxwcm9kdWN0bVxcaWRfWH0iLDAseyJzdHlsZSI6eyJib2R5Ijp7Im5hbWUiOiJkYXNoZWQifX19XSxbOSwxMiwiXFxTdWJfe1xcbWF0aGJme0N9fShcXHBsYWNlaG9sZGVyIFxccHJvZHVjdCBYKSIsMix7ImxhYmVsX3Bvc2l0aW9uIjo0MCwic2hvcnRlbiI6eyJzb3VyY2UiOjEwLCJ0YXJnZXQiOjQwfSwibGV2ZWwiOjF9XV0=
\begin{tikzcd}
	\terminal & {\Sub_{\mathbf{C}}(\Power X\product X)} & {} & {\Power X} & {} \\
	& {\Sub_{\mathbf{C}}(Y\product X)} && Y
	\arrow["{\ni_A}", from=1-1, to=1-2]
	\arrow["m"', from=1-1, to=2-2]
	\arrow[""{name=0, anchor=center, inner sep=0}, "{\charac_{m}}"', dashed, from=2-4, to=1-4]
	\arrow["{\text{in }\mathbf{C}}"{description}, shift left=6, draw=none, from=1-3, to=1-5]
	\arrow["{\text{in }\catSet}"{description}, shift left=6, draw=none, from=1-1, to=1-2]
	\arrow[""{name=1, anchor=center, inner sep=0}, "{\pull{\charac_{m}}{\placeholder\productm\id_X}}", dashed, from=1-2, to=2-2]
	\arrow["{\Sub_{\mathbf{C}}(\placeholder \product X)}"'{pos=0.4}, shorten <=11pt, shorten >=45pt, from=0, to=1]
\end{tikzcd}
\end{equation}
\marginnote{A "finitely complete" "category" where every "object" has a "power object" is called an \textbf{(elementary) topos}. Topos theory is a vast subject concerned with properties and uses of toposes.}%TODO: write better.
%TODO: power object



\section{Generalization}%TODO: exercises asking to compute the opposite categories.
\begin{defn}[Comma category]\label{defn:comma}
    \AP Given two "functors" \begin{tikzcd}\mathbf{D} \arrow[r, squiggly, "F"] & \mathbf{C} & \mathbf{E} \arrow[l, squiggly, "G"']\end{tikzcd}, there is a "category" $\comcat{F}{G}$,\footnote{Some authors denote this "category" $F/G$.} called the ""comma category"", whose "objects" are triples $(X, Y, \alpha)$ with $X \in \obj{\mathbf{D}}$, $Y \in \obj{\mathbf{E}}$ and $\alpha : F(X) \rightarrow G(Y)$ (in $\mor{\mathbf{C}}$) and "morphisms" between $(X_1, Y_1, \alpha)$ and $(X_2, Y_2, \beta)$ are pairs of "morphisms" $(f,g) \in \Hom_{\mathbf{D}}(X_1,X_2) \times \Hom_{\mathbf{E}}(Y_1,Y_2)$ yielding a "commutative" square as in \eqref{diag:morphicomcat}.
    \begin{equation}\label{diag:morphicomcat}
    \begin{tikzcd}
        F(X_1) \arrow[d, "\alpha"'] \arrow[r, "F(f)"] & F(X_2) \arrow[d, "\beta"] \\
        G(Y_1) \arrow[r, "G(g)"'] & G(Y_2)
    \end{tikzcd}
    \end{equation}
\end{defn}
%We mention them here because it is the perfect time, but not related to universal properties.
\begin{defn}[Arrow category]
    In the setting of Definition \ref{defn:comma}, if $F = G = \id_{\mathbf{C}}$, \AP then $\comcat{\id_{\mathbf{C}}}{\id_{\mathbf{C}}}$ is called the ""arrow category"" of $\mathbf{C}$ and denoted $\arrowcat{\mathbf{C}}$. Its "objects" are "morphisms" in $\mathbf{C}$ and its "morphisms" are "commutative" squares in $\mathbf{C}$.\footnote{Less concisely, a "morphism" $\phi: f \rightarrow g$ between "morphisms" $f: X \rightarrow Y$ and $g: X' \rightarrow Y'$ is a pair of "morphisms" $\phi_X: X \rightarrow X'$ and $\phi_Y: Y \rightarrow Y'$ making \eqref{diag:comutesquarearrow} "commute".}
    \begin{marginfigure}\begin{equation}\label{diag:commutesquarearrow}
        % https://q.uiver.app/?q=WzAsNCxbMCwwLCJYIl0sWzAsMSwiWCciXSxbMSwwLCJZIl0sWzEsMSwiWSciXSxbMCwxLCJcXHBoaV9YIiwyXSxbMCwyLCJmIl0sWzIsMywiXFxwaGlfWSJdLFsxLDMsImciLDJdXQ==
        \begin{tikzcd}
            X & Y \\
            {X'} & {Y'}
            \arrow["{\phi_X}"', from=1-1, to=2-1]
            \arrow["f", from=1-1, to=1-2]
            \arrow["{\phi_Y}", from=1-2, to=2-2]
            \arrow["g"', from=2-1, to=2-2]
        \end{tikzcd}
    \end{equation}\end{marginfigure}
\end{defn}%TODO: add example?
\begin{exer}\label{exer:universal:arrowcatfunctors}\marginnote{\hyperref[soln:universal:arrowcatfunctors]{See solution.}}
    Let $\mathbf{C}$ be a "category" (note the change of font to distinguish the "functors" from their action).
    \begin{enumerate}
        \item Show that $\intro*\idarr:\mathbf{C} \rightsquigarrow \arrowcat{\mathbf{C}}$ sending $X \in \obj{\mathbf{C}}$ to $\id_X$ is "functorial".
        \item Show that $\intro*\sourcearr:\arrowcat{\mathbf{C}} \rightsquigarrow \mathbf{C}$ sending $f \in \obj{\arrowcat{\mathbf{C}}}$ to $\source(f)$ is "functorial".
        \item Show that $\intro*\targetarr:\arrowcat{\mathbf{C}} \rightsquigarrow \mathbf{C}$ sending $f \in \obj{\arrowcat{\mathbf{C}}}$ to $\target(f)$ is "functorial".
    \end{enumerate}
\end{exer}
\begin{defn}[Slice category]
    In the setting of Definition \ref{defn:comma}, if $F = \id_{\mathbf{C}}$ and $G= X: \termcat \rightsquigarrow \mathbf{C}$ is a "constant functor" selecting one "object" $G(\bullet) = X \in \obj{\mathbf{C}}$, \AP then $\comcat{\id_{\mathbf{C}}}{X}$ is called the ""slice category"" over $X$ and denoted $\slice{\mathbf{C}}{X}$.\footnote{Some authors call this "category" $\mathbf{C}$ over $X$.} Its "objects" are "morphisms" in $\mathbf{C}$ with "target" $X$ and its "morphisms" are "commutative" triangles with $X$ as a tip as in \eqref{diag:slice}.
    \begin{equation}\label{diag:slice}
        \begin{tikzcd} 
            A \arrow[rr] \arrow[rd] & & B \arrow[ld]\\
            & X &
        \end{tikzcd}
    \end{equation}
\end{defn}
\begin{defn}[Coslice category]
    In the setting of Definition \ref{defn:comma}, if $G = \id_{\mathbf{C}}$ and $F= X: \termcat \rightsquigarrow \mathbf{C}$ is a "constant functor" selecting one "object" $F(\bullet) = X \in \obj{\mathbf{C}}$, then $\comcat{X}{\id_{\mathbf{C}}}$ is called the ""coslice category"" under $X$ and denoted $\coslice{X}{\mathbf{C}}$.\footnote{Some authors call this "category" $\mathbf{C}$ under $X$.} Its "objects" are "morphisms" in $\mathbf{C}$ with "source" $X$ and its "morphisms" are "commutative" triangles with $X$ as a tip as in \eqref{diag:slice}.
    \begin{equation}\label{diag:coslice}
        \begin{tikzcd}
            & X \arrow[ld] \arrow[rd] &\\
            A \arrow[rr]  & & B 
        \end{tikzcd}
    \end{equation}
\end{defn}
\begin{exer}\label{exer:universal:termslice}\marginnote{\hyperref[soln:universal:termslice]{See solution.}}
    Show that for any "category" $\mathbf{C}$ and "object" $X \in \obj{\mathbf{C}}$, the "slice category" $\slice{\mathbf{C}}{X}$ has a "terminal" "object". State and prove the "dual@@CAT" statement.
\end{exer}
\begin{exer}\label{exer:universal:productslice}\marginnote{\hyperref[soln:universal:productslice]{See solution.}}
    Show that the "product" of $f:A \rightarrow X$ and $g: B \rightarrow X$ in $\slice{\mathbf{C}}{X}$ exists if and only if the "pullback" of $\begin{tikzcd}[cramped, sep=small] A \arrow[r, "f"] & X & B \arrow[l, "g"'] \end{tikzcd}$ exists in $\mathbf{C}$. State and prove the "dual@@CAT" statement.
\end{exer}
% \begin{cor}["Dual@@CAT"]
%     The "coproduct" of $f:X \rightarrow A$ and $g: X \rightarrow A$ in $\coslice{\mathbf{C}}{X}$ exists if and only if the "pushout" of $\begin{tikzcd}[cramped, sep=small] A & \arrow[l, "f"'] X \arrow[r, "g"] & B \end{tikzcd}$ exists in $\mathbf{C}$.
% \end{cor}
Back to "universal properties".
\begin{defn}[Universal morphism]
    If $F: \mathbf{D} \rightsquigarrow \mathbf{C}$ is a functor and $X \in \obj{\mathbf{C}}$. \AP A ""universal morphism"" from $X$ to $F$ is an "initial" "object" in $\comcat{X}{F}$. Namely, it is a "morphism" $a : X \rightarrow F(A)$ such that for any other "morphism" $b: X \rightarrow F(B)$, there is unique "commutative" triangle as in \eqref{diag:univmorph}.
    \begin{equation}\label{diag:univmorph}
        \begin{tikzcd}
            & X \arrow[ld, "a"'] \arrow[rd, "b"] &\\
            F(A) \arrow[rr, dashed, "F(f)"']  & & F(B) 
        \end{tikzcd}
    \end{equation}
    Notice that equivalently, one could say that for any $b: X \rightarrow F(B)$, there is a unique "morphism" $f: A \rightarrow B$ in $\mathbf{D}$ such that $F(f) \circ a = b$, which is summarized in \eqref{diag:univmorphalt}. %TODO: Detail the equivalently: uniqueness of f inside D because we defined morphisms in the comma category as coming from D (not simply being the image of F)
    \begin{equation}\label{diag:univmorphalt}
        \begin{tikzcd}
            & X & FA & {} & A & {} \\
            && FB && B
            \arrow["a", from=1-2, to=1-3]
            \arrow[""{name=0, anchor=center, inner sep=0}, "Ff", dashed, from=1-3, to=2-3]
            \arrow["b"', from=1-2, to=2-3]
            \arrow[""{name=1, anchor=center, inner sep=0}, "f", dashed, from=1-5, to=2-5]
            \arrow["{\text{in }\mathbf{C}}"{description}, shift left=6, draw=none, from=1-2, to=1-3]
            \arrow["{\text{in }\mathbf{D}}"{description}, shift left=6, draw=none, from=1-4, to=1-6]
            \arrow["{F}"', shorten <=6pt, shorten >=12pt, from=1, to=0]
        \end{tikzcd}
    \end{equation}
    
    The "dual@@CAT" notion is a "universal morphism" from $F$ to $X$, it is a "terminal" "object" in $\comcat{F}{X}$. The "dual@@CAT" of \eqref{diag:univmorphalt} is depicted below.
    \begin{equation}\label{diag:univmorphaltdual}
        % https://q.uiver.app/?q=WzAsNyxbMSwwLCJGQSJdLFswLDAsIkEiXSxbMSwxLCJGQiJdLFszLDEsIkIiXSxbMywwLCJBIl0sWzIsMF0sWzQsMF0sWzAsMSwiYSIsMl0sWzIsMCwiRmYiLDIseyJzdHlsZSI6eyJib2R5Ijp7Im5hbWUiOiJkYXNoZWQifX19XSxbMiwxLCJiIl0sWzMsNCwiZiIsMix7InN0eWxlIjp7ImJvZHkiOnsibmFtZSI6ImRhc2hlZCJ9fX1dLFs1LDYsIlxcdGV4dHtpbiB9XFxtYXRoYmZ7RH0iLDEseyJvZmZzZXQiOi01LCJzdHlsZSI6eyJib2R5Ijp7Im5hbWUiOiJub25lIn0sImhlYWQiOnsibmFtZSI6Im5vbmUifX19XSxbMSwwLCJcXHRleHR7aW4gfVxcbWF0aGJme0N9IiwxLHsib2Zmc2V0IjotNSwic3R5bGUiOnsiYm9keSI6eyJuYW1lIjoibm9uZSJ9LCJoZWFkIjp7Im5hbWUiOiJub25lIn19fV0sWzEwLDgsIkYiLDIseyJsYWJlbF9wb3NpdGlvbiI6NDAsInNob3J0ZW4iOnsic291cmNlIjoxMCwidGFyZ2V0Ijo0MH0sImxldmVsIjoxfV1d
\begin{tikzcd}
	A & FA & {} & A & {} \\
	& FB && B
	\arrow["a"', from=1-2, to=1-1]
	\arrow[""{name=0, anchor=center, inner sep=0}, "Ff"', dashed, from=2-2, to=1-2]
	\arrow["b", from=2-2, to=1-1]
	\arrow[""{name=1, anchor=center, inner sep=0}, "f"', dashed, from=2-4, to=1-4]
	\arrow["{\text{in }\mathbf{D}}"{description}, shift left=6, draw=none, from=1-3, to=1-5]
	\arrow["{\text{in }\mathbf{C}}"{description}, shift left=6, draw=none, from=1-1, to=1-2]
	\arrow["F"', shorten <=6pt, shorten >=12pt, from=1, to=0]
\end{tikzcd}
    \end{equation} 
\end{defn}
\begin{defn}[Universal property]
    \AP A ""universal property"" is the property of being a "universal morphism". %TODO: meh
\end{defn}
\begin{exmps}\label{exmps:allup}
    Here we translate all the examples of this chapter into the general language.
    \begin{enumerate}
        \item The "free monoid" on a set $A$ is the "universal morphism" from $A$ to the "forgetful functor" $\catMon \rightsquigarrow \catSet$.
        \item The "abelianization" of a "group" $G$ is the "universal morphism" from $G$ to the "forgetful functor" $\catAb \rightsquigarrow \catGrp$.
        \item The set $S \subseteq V$ is a "basis" for the "vector space" $V$ when the inclusion $S \hookrightarrow V$ is the "universal morphism" from $S$ to the "forgetful functor" $\catVect{k} \rightsquigarrow \catSet$.
        \item An "exponential object" is an "object" $A^X$ along with a "universal morphism" from the "functor" $\placeholder\product X$ to $A$.
        \item A "subobject classifier" is a "morphism" $\top: \terminal \hookrightarrow \Omega$ such that the corresponding function $\top: \terminal \rightarrow \Sub_{\mathbf{C}}(\Omega)$ is a "universal morphism" from $\terminal$ to the "functor" $\Sub_{\mathbf{C}}$.
        \item A "power object" of $X$ is an object $\Power X$ along with a "universal morphism" $\ni_X$ from $\terminal$ to $\Sub_{\mathbf{C}}(\placeholder\product X)$.
    \end{enumerate}
\end{exmps}
We will not bother applying this general definition anymore because the formalism is not crucial to the study of "universal properties".

We have to postpone to Chapter \ref{chap:yoneda} showing that, as we have claimed, any "(co)@colimit""limit" satisfies a "universal property". Still, you might have noticed that our definition of "universal property" also uses a special case of "(co)@colimit""limits", that is, "initial" and "terminal" "objects". What is more, in the following chapters, we will introduce a couple more concepts which often coincide\footnote{By \textit{coincide}, we mean that one is a special case of the other or vice-versa or both directions.} with the concepts of "(co)@colimit""limits" and "universal properties".

%TODO: maybe do diagram chasing part 2
\end{document}