\documentclass[main.tex]{subfiles}
\begin{document}
\chapter{Solutions to Exercises}\label{chap:solutions}
\section{Solutions to Chapter \ref{chap:catfunc}}
\begin{proof}[Solution to Exercise \ref{exer:catfunc:countersubcat}]\label{soln:catfunc:countersubcat}
    Take any "monoid" $M$ with an idempotent element $x\neq 1_M$ (it satisfies $x\cdot x = x$). Letting $\mathbf{C}$ be $\deloop{M}$ and $\mathbf{C}'$ contain the "object" $\deloopobject$ and only the "morphism" $x$ yields a suitable example because the "identity" in $\mathbf{C}'$ is $x$.
\end{proof}
\begin{proof}[Solution to Exercise \ref{exer:catfunc:diagfunc}]\label{soln:catfunc:diagfunc}
    On "morphisms", we define $\diagFunc_{\mathbf{C}}(f) = (f,f)$. The "functoriality" properties hold because everything in $\mathbf{C}\cattimes \mathbf{C}$ is done componentwise.
    \begin{enumerate}[i.]
        \item For $f: X \rightarrow Y$, we have $(f,f): (X,X) \rightarrow (Y,Y)$.
        \item For $f: X \rightarrow Y$ and $g: Y \rightarrow Z$, we have $(g,g) \circ (f,f) = (g \circ f, g \circ f)$.
        \item For any $X \in \obj{\mathbf{C}}$, we have $\diagFunc_{\mathbf{C}}(\id_X) = (\id_X,\id_X) = \id_{(X,X)}$.
    \end{enumerate}
\end{proof}
\begin{proof}[Solution to Exercise \ref{exer:catfunc:placeholder}]\label{soln:catfunc:placeholder}
    A quick way to show $F(X,\placeholder)$ is a "functor" is to recognize it as the "composition" of $F$ with $X \functimes \id_{\mathbf{C}'}$, where $X$ is the "constant functor" at $X$. Similarly, $F(\placeholder, Y) := F \circ (\id_{\mathbf{C}} \functimes Y)$.
\end{proof}
\begin{proof}[Solution to Exercise \ref{exer:catfunc:funccomponent}]\label{soln:catfunc:funccomponent}
    Let us show the three properties of "functoriality".
    \begin{enumerate}[i.]
        \item For any $(f,g): (X,X') \rightarrow (Y,Y')$, by hypothesis, we have the following "commutative" square showing $F(f,g)$ has the right "source" and "target".
        \begin{equation*}
            % https://q.uiver.app/?q=WzAsNCxbMCwwLCJGKFgsWSkiXSxbMCwxLCJGKFgnLFkpIl0sWzEsMSwiRihYJyxZJykiXSxbMSwwLCJGKFgsWScpIl0sWzAsMSwiRihmLFxcaWRfWSkiLDJdLFsxLDIsIkYoXFxpZF97WCd9LGcpIiwyXSxbMywyLCJGKGYsXFxpZF97WSd9KSJdLFswLDMsIkYoXFxpZF9YLGcpIl0sWzAsMiwiRihmLGcpIiwxXV0=
        \begin{tikzcd}
        {F(X,X')} & {F(X,Y')} \\
        {F(Y,X')} & {F(Y,Y')}
        \arrow["{F(f,\id_{X'})}"', from=1-1, to=2-1]
        \arrow["{F(\id_{Y},g)}"', from=2-1, to=2-2]
        \arrow["{F(f,\id_{Y'})}", from=1-2, to=2-2]
        \arrow["{F(\id_X,g)}", from=1-1, to=1-2]
        \arrow["{F(f,g)}"{description}, from=1-1, to=2-2]
        \end{tikzcd}
        \end{equation*}
        \item Let us have two "morphisms" $(f,g): (X,X') \rightarrow (Y,Y')$ and $(f',g'): (Y,Y') \rightarrow (Z,Z')$ in $\mathbf{C} \cattimes \mathbf{C}'$. The hypothesis on $F(\placeholder, \placeholder)$ gives the four "commutative" squares below and the "functoriality" of $F$ in each component gives the "commutativity" of the parts denoted by $*$.
        \begin{equation*}
            % https://q.uiver.app/?q=WzAsOSxbMCw0LCJGKFosWCcpIl0sWzIsNCwiRihaLFknKSJdLFs0LDAsIkYoWCxaJykiXSxbNCwyLCJGKFksWicpIl0sWzQsNCwiRihaLFonKSJdLFswLDAsIkYoWCxYJykiXSxbMCwyLCJGKFksWCcpIl0sWzIsMCwiRihYLFknKSJdLFsyLDIsIkYoWSxZJykiXSxbMCwxLCJGKFxcaWRfWixnKSIsMV0sWzIsMywiRihmLFxcaWRfe1onfSkiLDFdLFszLDQsIkYoZicsXFxpZF97Wid9KSIsMV0sWzEsNCwiRihcXGlkX1osZycpIiwxXSxbNSw2LCJGKGYsXFxpZF97WCd9KSIsMV0sWzUsNywiRihcXGlkX1gsZykiLDFdLFs3LDgsIkYoZixcXGlkX3tZJ30pIiwxXSxbNiw4LCJGKFxcaWRfWSxnKSIsMV0sWzgsMSwiRihmJyxcXGlkX3tZJ30pIiwxXSxbNiwwLCJGKGYnLFxcaWRfe1gnfSkiLDFdLFs4LDMsIkYoXFxpZF9ZLGcnKSIsMV0sWzcsMiwiRihcXGlkX1gsZycpIiwxXSxbNSw4LCJGKGYsZykiLDFdLFs3LDMsIkYoZixnJykiLDFdLFs2LDEsIkYoZicsZykiLDFdLFs4LDQsIkYoZicsZycpIiwxXSxbNSwwLCJGKGYnXFxjaXJjIGYsXFxpZF97WCd9KSIsMix7Im9mZnNldCI6MSwiY3VydmUiOjV9XSxbNSwyLCJGKFxcaWRfWCxnJ1xcY2lyYyBnKSIsMCx7ImN1cnZlIjotNX1dLFsyLDQsIkYoZidcXGNpcmMgZixcXGlkX3taJ30pIiwwLHsib2Zmc2V0IjotMSwiY3VydmUiOi01fV0sWzAsNCwiRihcXGlkX1osZydcXGNpcmMgZykiLDIseyJjdXJ2ZSI6NX1dLFsyNiw3LCIqIiwxLHsibGFiZWxfcG9zaXRpb24iOjcwLCJzaG9ydGVuIjp7InNvdXJjZSI6MjB9LCJzdHlsZSI6eyJib2R5Ijp7Im5hbWUiOiJub25lIn0sImhlYWQiOnsibmFtZSI6Im5vbmUifX19XSxbMSwyOCwiKiIsMSx7ImxhYmVsX3Bvc2l0aW9uIjozMCwic2hvcnRlbiI6eyJ0YXJnZXQiOjIwfSwic3R5bGUiOnsiYm9keSI6eyJuYW1lIjoibm9uZSJ9LCJoZWFkIjp7Im5hbWUiOiJub25lIn19fV0sWzMsMjcsIioiLDEseyJsYWJlbF9wb3NpdGlvbiI6MzAsInNob3J0ZW4iOnsidGFyZ2V0IjoyMH0sInN0eWxlIjp7ImJvZHkiOnsibmFtZSI6Im5vbmUifSwiaGVhZCI6eyJuYW1lIjoibm9uZSJ9fX1dLFsyNSw2LCIqIiwxLHsibGFiZWxfcG9zaXRpb24iOjcwLCJzaG9ydGVuIjp7InNvdXJjZSI6MjB9LCJzdHlsZSI6eyJib2R5Ijp7Im5hbWUiOiJub25lIn0sImhlYWQiOnsibmFtZSI6Im5vbmUifX19XV0=
\begin{tikzcd}
	{F(X,X')} && {F(X,Y')} && {F(X,Z')} \\
	\\
	{F(Y,X')} && {F(Y,Y')} && {F(Y,Z')} \\
	\\
	{F(Z,X')} && {F(Z,Y')} && {F(Z,Z')}
	\arrow["{F(\id_Z,g)}"{description}, from=5-1, to=5-3]
	\arrow["{F(f,\id_{Z'})}"{description}, from=1-5, to=3-5]
	\arrow["{F(f',\id_{Z'})}"{description}, from=3-5, to=5-5]
	\arrow["{F(\id_Z,g')}"{description}, from=5-3, to=5-5]
	\arrow["{F(f,\id_{X'})}"{description}, from=1-1, to=3-1]
	\arrow["{F(\id_X,g)}"{description}, from=1-1, to=1-3]
	\arrow["{F(f,\id_{Y'})}"{description}, from=1-3, to=3-3]
	\arrow["{F(\id_Y,g)}"{description}, from=3-1, to=3-3]
	\arrow["{F(f',\id_{Y'})}"{description}, from=3-3, to=5-3]
	\arrow["{F(f',\id_{X'})}"{description}, from=3-1, to=5-1]
	\arrow["{F(\id_Y,g')}"{description}, from=3-3, to=3-5]
	\arrow["{F(\id_X,g')}"{description}, from=1-3, to=1-5]
	\arrow["{F(f,g)}"{description}, from=1-1, to=3-3]
	\arrow["{F(f,g')}"{description}, from=1-3, to=3-5]
	\arrow["{F(f',g)}"{description}, from=3-1, to=5-3]
	\arrow["{F(f',g')}"{description}, from=3-3, to=5-5]
	\arrow[""{name=0, anchor=center, inner sep=0}, "{F(f'\circ f,\id_{X'})}"', shift right=1, curve={height=30pt}, from=1-1, to=5-1]
	\arrow[""{name=1, anchor=center, inner sep=0}, "{F(\id_X,g'\circ g)}", curve={height=-30pt}, from=1-1, to=1-5]
	\arrow[""{name=2, anchor=center, inner sep=0}, "{F(f'\circ f,\id_{Z'})}", shift left=1, curve={height=-30pt}, from=1-5, to=5-5]
	\arrow[""{name=3, anchor=center, inner sep=0}, "{F(\id_Z,g'\circ g)}"', curve={height=30pt}, from=5-1, to=5-5]
	\arrow["{*}"{description, pos=0.7}, Rightarrow, draw=none, from=1, to=1-3]
	\arrow["{*}"{description, pos=0.3}, Rightarrow, draw=none, from=5-3, to=3]
	\arrow["{*}"{description, pos=0.3}, Rightarrow, draw=none, from=3-5, to=2]
	\arrow["{*}"{description, pos=0.7}, Rightarrow, draw=none, from=0, to=3-1]
\end{tikzcd}
        \end{equation*}
        We conclude from the "commutativity" of the whole diagram that $F(f',g') \circ F(f,g) = F(f'\circ f, g' \circ g)$.
        \item For any $(A,B) \in \obj{(\mathbf{C} \cattimes \mathbf{C}')}$, the "functoriality" of either component yields
    \[F(\id_{(A,B)}) = F(\id_A,\id_B) = \id_{F(A,B)}.\]
    \end{enumerate}
    
\end{proof}
\section{Solutions to Chapter \ref{chap:duality}}
\begin{proof}[Solution to Exercise \ref{exer:duality:composemor}]\label{soln:duality:composemor}
    Let us have two "morphisms" $f: X \rightarrow Y$ and $g: Y \rightarrow Z$.
    \begin{itemize}
        \item Suppose $f$ and $g$ are "monic". For any $h_1,h_2: Z \rightarrow Z'$ satisfying $h_1 \circ g \circ f = h_2 \circ g \circ f$, "monicity" of $f$ implies $h_1 \circ g = h_2 \circ g$ which in turn, by "monicity" of $g$ imply $h_1 = h_2$. Thus, $g \circ f$ is "monic".
        \item We apply "duality". Suppose $f$ and $g$ are "epic", then $\op{f}$ and $\op{g}$ are "monic" so $\op{(g\circ f)} = \op{f} \circ \op{g}$ is "monic", thus $g \circ f$ is "epic".
        \item If $f$ and $g$ are "isomorphisms@@CAT", then it is easy to check that $f^{-1} \circ g^{-1}$ is the "inverse" of $g \circ f$, implying $g \circ f$ is an "isomorphism@@CAT".
    \end{itemize}
\end{proof}
\begin{proof}[Solution to Exercise \ref{exer:duality:terminalinitial}]\label{soln:duality:terminalinitial}
    We draw the "categories" with all the "morphisms" and we let you infer the "composition"\footnote{The "categories" (a) and (b) have a uniquely determined "composition". For (c) and (d), "composing" the non-identity "endomorphism" with itself can yield either itself or $\id_Y$.} and show that they fit the requirement (by counting "morphisms").\\
    \begin{minipage}{0.49\textwidth}
        \begin{equation*}
            % https://tikzcd.yichuanshen.de/#N4Igdg9gJgpgziAXAbVABwnAlgFyxMJZABgBpiBdUkANwEMAbAVxiRAA0QBfU9TXfIRQBGclVqMWbAJrdxMKAHN4RUADMAThAC2SMiBwQ91BhAhoiwgBxk1jODHEM6AIxgMACvzwE2GrIoAFjgg1PTMrIggADrRWFAA+pw86lq6iKIGRhkmZhYiNqR2DA5Oru5e2D5CIP5BIWGSkTFxibIpIJo6xllIwlwUXEA
            \begin{tikzcd}
            X \arrow["\id_X"', loop, distance=2em, in=125, out=55] \arrow[r] & Y \arrow["\id_Y"', loop, distance=2em, in=125, out=55]
            \end{tikzcd}
            \tag{a}
        \end{equation*}
    \end{minipage}\begin{minipage}{0.49\textwidth}
        \begin{equation*}
            % https://tikzcd.yichuanshen.de/#N4Igdg9gJgpgziAXAbVABwnAlgFyxMJZABgBpiBdUkANwEMAbAVxiRAA0QBfU9TXfIRQBGclVqMWbAJrdxMKAHN4RUADMAThAC2SMiBwQ91BhAhoiwgBxk1jODHEM6AIxgMACvzwE2GrIoAFjgg1PTMrIggADrRWFAA+pw86lq6iKIGRhkmZhYiNqR2DA5Oru5e2D5CIP5BIWGSkTFxibIpIJo6xllImW5gUEgAzPpwgVhqIRkdXemZhj0DQ4ij1OOT08KzaX3Ui4hjE1NIALT9MIMjxFwUXEA
            \begin{tikzcd}
            X \arrow["\id_X"', loop, distance=2em, in=125, out=55] \arrow[r, bend left, shift right] & Y \arrow["\id_Y"', loop, distance=2em, in=125, out=55] \arrow[l, bend left, shift right] \arrow[l, bend left, shift left]
            \end{tikzcd}
            \tag{b}
        \end{equation*}
    \end{minipage}\\\begin{minipage}{0.49\textwidth}
        \begin{equation*}
            % https://tikzcd.yichuanshen.de/#N4Igdg9gJgpgziAXAbVABwnAlgFyxMJZABgBpiBdUkANwEMAbAVxiRAA0QBfU9TXfIRQBGclVqMWbAJrdxMKAHN4RUADMAThAC2SMiBwQ91BhAhoiwgBxk1jODHEM6AIxgMACvzwE2GrIoAFjgg1PTMrIggADrRWFAA+pw86lq6iKIGRhkmZhYiNqR2DA5Oru5e2D5CIP5BIWGSkTFxibIpIJo6SJmGxiBwgVhqIYjEHV3pvdmZpuZEtvaOJuWe3oJ+AcFyXEA
            \begin{tikzcd}
            X \arrow["\id_X"', loop, distance=2em, in=125, out=55] & Y \arrow["\id_Y"', loop, distance=2em, in=125, out=55] \arrow[l] \arrow[loop, distance=2em, in=305, out=235]
            \end{tikzcd}
            \tag{c}
        \end{equation*}
    \end{minipage}\begin{minipage}{0.49\textwidth}
        \begin{equation*}
            % https://tikzcd.yichuanshen.de/#N4Igdg9gJgpgziAXAbVABwnAlgFyxMJZABgBpiBdUkANwEMAbAVxiRAA0QBfU9TXfIRQBGclVqMWbAJrdxMKAHN4RUADMAThAC2SMiBwQ91BhAhoiwgBxk1jODHEM6AIxgMACvzwE2GrIoAFjgg1PTMrIggADrRWFAA+pw86lq6iKIGRhkmZhYiNqR2DA5Oru5e2D5CIP5BIWGSkTFxibIpIJo6SJmGxiBwgVhqITkgbmBQSADMxB1d6b3ZmabmRLb2jiblnt6CfgHB3Lydaf19Y4PDo5kTU4izXBRcQA
            \begin{tikzcd}
            X \arrow["\id_X"', loop, distance=2em, in=125, out=55] \arrow[r, bend left, shift right] & Y \arrow["\id_Y"', loop, distance=2em, in=125, out=55] \arrow[l, bend left, shift right] \arrow[loop, distance=2em, in=305, out=235]
            \end{tikzcd}
            \tag{d}
        \end{equation*}
    \end{minipage}
\end{proof}
\begin{proof}[Solution to Exercise \ref{exer:duality:preserving}]\label{soln:duality:preservingexer:duality:preserving}
    \begin{enumerate}
        \item Let $f:A \rightarrow B$ be the only non-identity "morphism" in $\cattwo$, it is a "monomorphism" vacuously because there is only one "morphism" with "target" $A$ ($\id_A$). Now, for any "morphism" $m: X \rightarrow Y \in \mor{\mathbf{C}}$, we can define $F: \cattwo \rightsquigarrow \mathbf{C}$ by $FA = X$, $FB = Y$ and $Ff = m$ and it will be a "functor". Thus, choosing $m$ that is not "monic" yields the required example.
        \item If $f$ is "split monic", it has a "right inverse" $f'$. This implies $Ff'$ is the "right inverse" of $Ff$ because $Ff \circ Ff' = F(f \circ f') = F(\id) = \id$. We conclude that $Ff$ is "split monic".
        \item We need to show that "functors" "preserve" "split epimorphisms". By "duality@@CAT", if $f$ is "split epic", then $\op{f}$ is "split monic", thus it is "preserved" by the "functor" $\op{F}$. And $Ff = \op{(\op{F}(\op{f}))}$ is "split epic". %TODO: write better.
        \item "Functors" "preserve" "isomorphisms@@CAT" because a "morphism" is an "isomorphism@@CAT" if and only if it is "split epic" and "split monic".\footnote{Because "split epic" is equivalent to having a "left inverse" and "split monic" is equivalent to having a "right inverse".} If $A \isoCAT B$ and $i: A \rightarrow B$ is an "isomorphism@@CAT", then $Fi: FA \rightarrow FB$ is an "isomorphism@@CAT", so $FA \isoCAT FB$.
    \end{enumerate}
\end{proof}
\begin{proof}[Solution to Exercise \ref{exer:duality:reflecting}]\label{soln:duality:reflecting}
    \begin{enumerate}
    \item Let $\mathbf{C}$ be a "category" with at least one "morphism" $f$ that is not "monic", the only "functor" $\termmorph : \mathbf{C} \rightsquigarrow \termcat$ sends $f$ to $\id_{\bullet}$ which is "monic".
    \item Suppose that $F(f)$ is "monic" and let $g$ and $h$ be such that $f \circ g = f \circ h$. By "monicity" of $F(f)$, $F(f) \circ F(g) = F(f \circ g) = F(f \circ h) = F(f) \circ F(h)$ implies $F(g) = F(h)$. Since $F$ is "faithful", $g = h$.
    \item We need to show "faithful" "functors" "reflect" "epimorphisms". %TODO: finish
    \end{enumerate}
\end{proof}
\begin{proof}[Solution to Exercise \ref{exer:duality:equivsubobj}]\label{soln:duality:equivsubobj}
    Let us have three "monomorphisms" $m: Y \hookrightarrow X$, $n: Z \hookrightarrow X$ and $o: W \hookrightarrow X$.

    \textbf{Reflexivity:} We have $m \circ \id_Y = m$ thus $m \sim m$.

    \textbf{Symmetry:} Suppose that $m \sim n$, namely, there is an "isomorphism@@CAT" $i: Y \rightarrow X$ such that $m = n \circ i$. Then, "pre-composing" with the "isomorphism@@CAT" $i^{-1}$ yields $m \circ i^{-1} = n$ which implies $n \sim m$.

    \textbf{Transitivity:} If $m\sim n$ and $n\sim o$, then there exist "isomorphisms@@CAT" $i: Y \rightarrow Z$ and $i': W \rightarrow Z$ satisfying $m = n \circ i$ and $n = o \circ i'$. Therefore, we have $m =  o \circ i' \circ i$ which implies $m \sim o$.\footnote{Recall that the "composition" of two "isomorphisms@@CAT" is an "isomorphism@@CAT".}
\end{proof}
\begin{proof}[Solution to Exercise \ref{exer:duality:posetsubobj}]\label{soln:duality:posetsubobj}
    Let us have five "monomorphisms" $m: Y \hookrightarrow X$, $m:Y' \hookrightarrow X$, $n: Z \hookrightarrow X$, $n': Z' \hookrightarrow X$ and $o: W \hookrightarrow X$.\footnote{Recall that we often use $m$ to refer to $[m]$.}

    \textbf{Well-defined:} Suppose that $m \leq n$, $m' \sim m$ and $n\sim n'$, namely, there is a "morphism" $k: Y \rightarrow Z$ and "isomorphisms@@CAT" $i: Y \circ Y'$ and $i': Z' \rightarrow Z$ such that $m = n \circ k$, $m' = m \circ i$ and $n = n' \circ i'$. Combining these equalities yields $m' = n' \circ i' \circ  k \circ i$ which witnesses $m' \leq n'$.

    \textbf{Reflexivity:} We have $m \circ \id_Y = m$ thus $m \leq m$.

    \textbf{Antisymmetry:} If $m\leq n$ and $n\leq m$, then there exist "morphisms" $k: Y \rightarrow Z$ and $k': Z \rightarrow Y$ satisfying $m = n \circ k$ and $n = m \circ k'$. Combining these two equalities yield $m = m \circ k' \circ k$ and $n = n \circ k \circ k'$. Therefore, since $m$ and $n$ are "monic", we infer that $k' \circ k = \id_Y$ and $k \circ k' = \id_Z$. This means $k$ is an "isomorphism@@CAT" and $m \sim n$ (so $[m] = [n]$).

    \textbf{Transitivity:} If $m\leq n$ and $n\leq o$, then there exist "morphisms" $k: Y \rightarrow Z$ and $k': W \rightarrow Z$ satisfying $m = n \circ k$ and $n = o \circ k'$. Therefore, we have $m =  o \circ k' \circ k$ which implies $m \leq o$.
\end{proof}
\section{Solutions to Chapter \ref{chap:limits}}
\begin{proof}[Solution to Exercise \ref{exer:limits:prodcommute}]\label{soln:limits:prodcommute}
    As we have said that "binary products" are unique up to "isomorphism@@CAT", it is enough to show that $A\product B$ satisfies the same "universal property" as $B \product A$. Let $\projection_A$ and $\projection_B$ be the "projections" of $A \product B$, we claim that $\begin{tikzcd}[cramped, sep=small]
        B & {A\product B} & A
        \arrow["{\projection_B}"', from=1-2, to=1-1]
        \arrow["{\projection_A}", from=1-2, to=1-3]
    \end{tikzcd}$ is the "product@bproduct" of $B$ and $A$. Indeed, for any $\begin{tikzcd}[cramped, sep=small]
        B & {X} & A
        \arrow["{p_B}"', from=1-2, to=1-1]
        \arrow["{p_A}", from=1-2, to=1-3]
    \end{tikzcd}$, we use the original "universal property" of $A \product B$ to find a unique "mediating morphism" $!:X \rightarrow A \product B$ such that $\projection_B \circ {!} = p_B$ and $\projection_A \circ {!} = p_A$.
\end{proof}
\begin{proof}[Solution to Exercise \ref{exer:limits:termneutralprod}]\label{soln:limits:termneutralprod}
    
\end{proof}
\begin{proof}[Solution to Exercise \ref{exer:limits:functionproduct}]\label{soln:limits:functionproduct}
    The existence and uniqueness of $\Product_{i \in I} f_i$ is given by the "universal property" of the "product" $\Product_{i \in I} Y_i$ with for each $j \in I$, the "morphism" $f_j \circ \projection_j: \Product_{i \in I} X_i \rightarrow Y_j$.
\end{proof}
\begin{proof}[Solution to Exercise \ref{exer:limits:pullbackmono}]\label{soln:limits:pullbackmono}
    ($\Rightarrow$) Suppose $f:X \rightarrow Y$ is "monic", "commutativity" of \eqref{diag:pullbackmono} is trivial. For any $\begin{tikzcd}[cramped, sep=small] X &\arrow[l, "g"']  Z\arrow[r, "h"]&X \end{tikzcd}$ satisfying $f \circ g = f \circ h$, we have $g = h$. Thus $g=h$ is the "mediating" "morphism" $!$ of \eqref{diag:solpullbackmono}, it is unique because $\id_X \circ m = g$ implies $m = g$.\begin{marginfigure}
        \begin{equation}\label{diag:solpullbackmono}
            \begin{tikzcd}
                Z \\
                & X & X \\
                & X & Y
                \arrow["{\id_X}"', from=2-2, to=3-2]
                \arrow["f"', from=3-2, to=3-3]
                \arrow["{\id_X}", from=2-2, to=2-3]
                \arrow["f", from=2-3, to=3-3]
                \arrow["\pullbackd"{anchor=center, pos=0.125}, draw=none, from=2-2, to=3-3]
                \arrow["g"', curve={height=12pt}, from=1-1, to=3-2]
                \arrow["h", curve={height=-12pt}, from=1-1, to=2-3]
                \arrow["{!}"{description}, dashed, from=1-1, to=2-2]
            \end{tikzcd}
        \end{equation}
    \end{marginfigure}
    ($\Leftarrow$) For any $g,h : Z \rightarrow X$ satisfying $f \circ g = f \circ h$, the "universal property" of the "pullback" tells us there is a unique $!:Z \rightarrow X$ making \eqref{diag:solpullbackmono} "commute". Since $!$ satisfies $g = \id_X \circ ! = h$, we conclude $g = ! = h$, thus $f$ is a "monomorphism".
    \begin{marginfigure}[1\baselineskip]
        \begin{equation}\label{diag:pushoutepic}
            \begin{tikzcd}
                X & Y \\
                Y & Y
                \arrow["{\id_X}"', from=2-1, to=2-2]
                \arrow["f"', from=1-1, to=2-1]
                \arrow["{\id_X}", from=1-2, to=2-2]
                \arrow["f", from=1-1, to=1-2]
                \arrow["\pushoutd"{anchor=center, pos=0.125, rotate=180}, draw=none, from=2-2, to=1-1]
            \end{tikzcd}
        \end{equation}
    \end{marginfigure}
    The "dual" statement is that $f: X \rightarrow Y$ is "epic" if and only if \eqref{diag:pushoutepic} is a "pushout". We leave the proof to you.
\end{proof}
\begin{proof}[Solution to Exercise \ref{exer:limits:isopullback}]\label{soln:limits:isopullback}
    Let $p_A: X \rightarrow A$ and $p_B: X \rightarrow B$ be such that \eqref{diag:pullid} "commutes". A "mediating morphism" $!: X \rightarrow A$ must satisfy $\id_A \circ ! = p_A$ and $f \circ !  = p_B$. The first equality ensures $! = p_A$ is unique and satisfies the second equality because the outer square "commuting" yields $f \circ p_A = p_B$.\\
    \begin{minipage}{0.49\textwidth}
        \begin{equation}\label{diag:pullid}
            % https://q.uiver.app/?q=WzAsNSxbMSwxLCJBIl0sWzEsMiwiQiJdLFsyLDEsIkEiXSxbMiwyLCJCIl0sWzAsMCwiWCJdLFswLDEsImYiLDJdLFswLDIsIlxcaWRfQSJdLFsyLDMsImYiXSxbMSwzLCJcXGlkX0IiLDJdLFswLDMsIiIsMSx7InN0eWxlIjp7Im5hbWUiOiJjb3JuZXIifX1dLFs0LDEsInBfQiIsMix7ImN1cnZlIjoyfV0sWzQsMiwicF9BIiwwLHsiY3VydmUiOi0yfV1d
            \begin{tikzcd}
            X \\
            & A & A \\
            & B & B
            \arrow["f"', from=2-2, to=3-2]
            \arrow["{\id_A}", from=2-2, to=2-3]
            \arrow["f", from=2-3, to=3-3]
            \arrow["{\id_B}"', from=3-2, to=3-3]
            \arrow["{p_B}"', curve={height=12pt}, from=1-1, to=3-2]
            \arrow["{p_A}", curve={height=-12pt}, from=1-1, to=2-3]
            \end{tikzcd}
        \end{equation}
    \end{minipage}\begin{minipage}{0.49\textwidth}
        \begin{equation}\label{diag:pullisomorphism}
            % https://q.uiver.app/?q=WzAsNSxbMSwxLCJBJyJdLFsxLDIsIkIiXSxbMiwxLCJBIl0sWzIsMiwiQiJdLFswLDAsIlgiXSxbMCwxLCJmXFxjaXJjIGkiLDJdLFswLDIsImkiXSxbMiwzLCJmIl0sWzEsMywiXFxpZF9CIiwyXSxbNCwxLCJwX0IiLDIseyJjdXJ2ZSI6Mn1dLFs0LDIsInBfQSIsMCx7ImN1cnZlIjotMn1dXQ==
        \begin{tikzcd}
            X \\
            & {A'} & A \\
            & B & B
            \arrow["{f\circ i}"', from=2-2, to=3-2]
            \arrow["i", from=2-2, to=2-3]
            \arrow["f", from=2-3, to=3-3]
            \arrow["{\id_B}"', from=3-2, to=3-3]
            \arrow["{p_B}"', curve={height=12pt}, from=1-1, to=3-2]
            \arrow["{p_A}", curve={height=-12pt}, from=1-1, to=2-3]
        \end{tikzcd}
        \end{equation}
    \end{minipage}\\
    Let $p_A: X \rightarrow A$ and $p_B: X \rightarrow B$ be such that \eqref{diag:pullisomorphism} "commutes". A unique "mediating morphism" $!: X \rightarrow A$ must satisfy $i \circ ! = p_A$ and $f \circ i \circ !  = p_B$. "Post-composing" the first equality by $i^{-1}$ implies $! = i^{-1} \circ p_A$ is unique and satisfies the second equality because $f \circ i \circ i^{-1} \circ p_A = f \circ p_A = p_B$.
\end{proof}
\begin{proof}[Solution to Exercise \ref{exer:limits:termpullcomplete}]\label{soln:limits:termpullcomplete}
    We will show that if $\mathbf{C}$ has all "pullbacks" and a "terminal" "object", then it has all finite "products" and "equalizers". This implies, using Remark \ref{rem:prodeqcomplete}, that $\mathbf{C}$ is "finitely complete".

    For finite "products", recall that it is enough to show that $\mathbf{C}$ has all "binary products" as it already has the empty "product" (the "terminal" "object"). We claim that the "pullback" of $\begin{tikzcd}[cramped, sep=small] A \arrow[r, "\termmorph"] & \terminal & B \arrow[l, "\termmorph"'] \end{tikzcd}$ is the "binary product" $A \product B$.%TODO: ref for first sentence.
    \begin{marginfigure}
        \begin{equation}
            % https://q.uiver.app/?q=WzAsNCxbMCwwLCJBXFxwdWxsYmFja3tcXHRlcm1pbmFsfSBCIl0sWzAsMSwiQSJdLFsxLDAsIkIiXSxbMSwxLCJcXHRlcm1pbmFsIl0sWzAsMSwiXFxwaV9BIiwyXSxbMCwyLCJcXHBpX0IiXSxbMiwzLCJcXHRlcm1tb3JwaCJdLFsxLDMsIlxcdGVybW1vcnBoIiwyXSxbMCwzLCIiLDEseyJzdHlsZSI6eyJuYW1lIjoiY29ybmVyIn19XV0=
            \begin{tikzcd}
                {A\pullback{\terminal} B} & B \\
                A & \terminal
                \arrow["{\pi_A}"', from=1-1, to=2-1]
                \arrow["{\pi_B}", from=1-1, to=1-2]
                \arrow["\termmorph", from=1-2, to=2-2]
                \arrow["\termmorph"', from=2-1, to=2-2]
                \arrow["\pullbackd"{anchor=center, pos=0.125}, draw=none, from=1-1, to=2-2]
            \end{tikzcd}
        \end{equation}
    \end{marginfigure}
    Indeed, for any $\begin{tikzcd}[cramped, sep=small] A &\arrow[l, "p_A"']  X\arrow[r, "p_B"]&B \end{tikzcd}$, we have $\termmorph \circ p_A = \termmorph \circ p_B$, thus, there is a unique "morphism" $!:X \rightarrow A\pullback{\terminal} B$ making \eqref{diag:prodfrompull} "commute". Since the "commutativity" of the squares always hold, this is equivalent to the "unviersal property" of the "binary product". Hence $A \product B \isoCAT A \pullback{\terminal} B$.
    \begin{equation}\label{diag:prodfrompull}
        % https://q.uiver.app/?q=WzAsNSxbMSwxLCJBXFxwdWxsYmFja3tcXHRlcm1pbmFsfSBCIl0sWzEsMiwiQSJdLFsyLDEsIkIiXSxbMiwyLCJcXHRlcm1pbmFsIl0sWzAsMCwiWCJdLFswLDEsIlxccGlfQSIsMl0sWzAsMiwiXFxwaV9CIl0sWzIsMywiXFx0ZXJtbW9ycGgiXSxbMSwzLCJcXHRlcm1tb3JwaCIsMl0sWzAsMywiIiwxLHsic3R5bGUiOnsibmFtZSI6ImNvcm5lciJ9fV0sWzQsMiwicF9CIiwxLHsiY3VydmUiOi0yfV0sWzQsMSwicF9BIiwxLHsiY3VydmUiOjJ9XSxbNCwwLCIhIiwxLHsic3R5bGUiOnsiYm9keSI6eyJuYW1lIjoiZGFzaGVkIn19fV1d
        \begin{tikzcd}
            X \\
            & {A\pullback{\terminal} B} & B \\
            & A & \terminal
            \arrow["{\pi_A}"', from=2-2, to=3-2]
            \arrow["{\pi_B}", from=2-2, to=2-3]
            \arrow["\termmorph", from=2-3, to=3-3]
            \arrow["\termmorph"', from=3-2, to=3-3]
            \arrow["\pullbackd"{anchor=center, pos=0.125}, draw=none, from=2-2, to=3-3]
            \arrow["{p_B}"{description}, curve={height=-12pt}, from=1-1, to=2-3]
            \arrow["{p_A}"{description}, curve={height=12pt}, from=1-1, to=3-2]
            \arrow["{!}"{description}, dashed, from=1-1, to=2-2]
        \end{tikzcd}
    \end{equation}
    %TODO: equalizers.
\end{proof}
\section{Solutions to Chapter \ref{chap:universal}}
\begin{proof}[Solution to Exercise \ref{exer:universal:prodXfunc}]\label{soln:universal:prodXfunc}
    We define $\placeholder\product X$ on "morphisms" by sending $f: Y \rightarrow Y' \in \mor{\mathbf{C}}$ to $f \productm \id_X : Y \product X \rightarrow Y'\productm X$. "Functoriality" follows from the definition of $\productm$ on "morphisms". Indeed, $\id_Y \productm \id_X$ is the only "morphism" making \eqref{diag:idprodXfunc} "commute" and $(g \circ f) \productm \id_X$ is the only "morphism" making \eqref{diag:compprodXfunc} "commute".\begin{marginfigure}Recall that if $f: A \rightarrow A'$ and $g: B \rightarrow B'$, $f\productm g: A \product B \rightarrow A'\product B'$ is the unique "morphism" making the diagram below "commute":
        \begin{equation*}
            % https://q.uiver.app/?q=WzAsNixbMSwwLCJBXFxwcm9kdWN0IEIiXSxbMCwwLCJBIl0sWzAsMSwiQSciXSxbMSwxLCJBJ1xccHJvZHVjdCBCJyJdLFsyLDAsIkIiXSxbMiwxLCJCJyJdLFswLDEsIlxccHJvamVjdGlvbl9BIiwyXSxbMSwyLCJmIiwyXSxbMCwzLCJmXFxwcm9kdWN0bSBnIiwxLHsic3R5bGUiOnsiYm9keSI6eyJuYW1lIjoiZGFzaGVkIn19fV0sWzMsMiwiXFxwcm9qZWN0aW9uX3tBJ30iXSxbMCw0LCJcXHByb2plY3Rpb25fQiJdLFs0LDUsImciXSxbMyw1LCJcXHBpX0IiLDJdXQ==
            \begin{tikzcd}
                A & {A\product B} & B \\
                {A'} & {A'\product B'} & {B'}
                \arrow["{\projection_A}"', from=1-2, to=1-1]
                \arrow["f"', from=1-1, to=2-1]
                \arrow["{f\productm g}", dashed, from=1-2, to=2-2]
                \arrow["{\projection_{A'}}", from=2-2, to=2-1]
                \arrow["{\projection_B}", from=1-2, to=1-3]
                \arrow["g", from=1-3, to=2-3]
                \arrow["{\pi_B}"', from=2-2, to=2-3]
            \end{tikzcd}
        \end{equation*}
        \end{marginfigure}
        \begin{minipage}{0.49\textwidth}
            \begin{equation}\label{diag:idprodXfunc}
                \begin{tikzcd}
                    Y & {Y\product X} \\
                    Y & {Y\product X}
                    \arrow["{\projection_Y}"', from=1-2, to=1-1]
                    \arrow["{\id_Y}"', from=1-1, to=2-1]
                    \arrow["{\id_Y\productm \id_X}", dashed, from=1-2, to=2-2]
                    \arrow["{\projection_{Y}}", from=2-2, to=2-1]
                \end{tikzcd}
            \end{equation}
        \end{minipage}\begin{minipage}{0.49\textwidth}
            \begin{equation}\label{diag:compprodXfunc}
                % https://q.uiver.app/?q=WzAsNixbMSwwLCJZXFxwcm9kdWN0IFgiXSxbMCwwLCJZIl0sWzAsMSwiWSciXSxbMSwxLCJZJ1xccHJvZHVjdCBYIl0sWzAsMiwiWScnIl0sWzEsMiwiWCJdLFswLDEsIlxccHJvamVjdGlvbl9ZIiwyXSxbMSwyLCJmIiwxXSxbMCwzLCJmXFxwcm9kdWN0bSBcXGlkX1giLDEseyJzdHlsZSI6eyJib2R5Ijp7Im5hbWUiOiJkYXNoZWQifX19XSxbMywyLCJcXHByb2plY3Rpb25fe1knfSJdLFszLDUsImdcXHByb2R1Y3RtIFxcaWRfWCIsMSx7InN0eWxlIjp7ImJvZHkiOnsibmFtZSI6ImRhc2hlZCJ9fX1dLFsyLDQsImciLDFdLFs1LDQsIlxccHJvamVjdGlvbl97WScnfSJdLFsxLDQsImcgXFxjaXJjIGYiLDIseyJjdXJ2ZSI6M31dLFswLDUsIihnXFxjaXJjIGYpIFxccHJvZHVjdG0gXFxpZF9YIiwwLHsiY3VydmUiOi00LCJzdHlsZSI6eyJib2R5Ijp7Im5hbWUiOiJkYXNoZWQifX19XV0=
        \begin{tikzcd}
            Y & {Y\product X} \\
            {Y'} & {Y'\product X} \\
            {Y''} & X
            \arrow["{\projection_Y}"', from=1-2, to=1-1]
            \arrow["f"{description}, from=1-1, to=2-1]
            \arrow["{f\productm \id_X}"{description}, dashed, from=1-2, to=2-2]
            \arrow["{\projection_{Y'}}", from=2-2, to=2-1]
            \arrow["{g\productm \id_X}"{description}, dashed, from=2-2, to=3-2]
            \arrow["g"{description}, from=2-1, to=3-1]
            \arrow["{\projection_{Y''}}", from=3-2, to=3-1]
            \arrow["{g \circ f}"', curve={height=18pt}, from=1-1, to=3-1]
            \arrow["{(g\circ f) \productm \id_X}", dashed, curve={height=-24pt}, from=1-2, to=3-2]
        \end{tikzcd}
            \end{equation}
        \end{minipage}
\end{proof}
\begin{proof}[Solution to Exercise \ref{exer:universal:subobjfunctor}]\label{soln:universal:subobjfunctor}
    First, we know that the "pullback" of the "monomorphism" $m$ "along@apull" $f$ is "monic" by Theorem \ref{thm:pullmono}. Next, for $n: I' \hookrightarrow X \in \Sub_{\mathbf{C}}(Y)$, we need to show $[m] = [n]$ implies $[\pull{f}{m}] = [\pull{f}{n}]$.\footnote{Recall that $[m] = [n]$ when there is an "isomorphism@@CAT" $i$ satifying $n = m \circ i$.} In \eqref{diag:subcfunctor}, we need to show there is an "isomorphism@@CAT" $i': J \rightarrow J'$ making everything "commute".
    \begin{equation}\label{diag:subcfunctor}
        % https://q.uiver.app/?q=WzAsNixbMCwxLCJKIl0sWzEsMiwiWCJdLFszLDIsIlkiXSxbMiwxLCJJIl0sWzMsMCwiSSciXSxbMSwwLCJKJyJdLFswLDEsIlxccHVsbHtmfXttfSIsMix7InN0eWxlIjp7InRhaWwiOnsibmFtZSI6Imhvb2siLCJzaWRlIjoidG9wIn19fV0sWzEsMiwiZiIsMl0sWzAsMywiaiIsMSx7ImxhYmVsX3Bvc2l0aW9uIjo3MH1dLFszLDIsIm0iLDEseyJzdHlsZSI6eyJ0YWlsIjp7Im5hbWUiOiJob29rIiwic2lkZSI6InRvcCJ9fX1dLFs0LDIsIm4iLDAseyJzdHlsZSI6eyJ0YWlsIjp7Im5hbWUiOiJob29rIiwic2lkZSI6InRvcCJ9fX1dLFszLDQsImkiLDFdLFs1LDEsIlxccHVsbHtmfXtufSIsMSx7ImxhYmVsX3Bvc2l0aW9uIjozMCwic3R5bGUiOnsidGFpbCI6eyJuYW1lIjoiaG9vayIsInNpZGUiOiJ0b3AifX19XSxbNSw0LCJqJyJdXQ==
\begin{tikzcd}
	& {J'} && {I'} \\
	J && I \\
	& X && Y
	\arrow["{\pull{f}{m}}"', hook, from=2-1, to=3-2]
	\arrow["f"', from=3-2, to=3-4]
	\arrow["j"{description, pos=0.7}, from=2-1, to=2-3]
	\arrow["m"{description}, hook, from=2-3, to=3-4]
	\arrow["n", hook, from=1-4, to=3-4]
	\arrow["i"{description}, from=2-3, to=1-4]
	\arrow["{\pull{f}{n}}"{description, pos=0.3}, hook, from=1-2, to=3-2]
	\arrow["{j'}", from=1-2, to=1-4]
\end{tikzcd}
    \end{equation}
    By the "pullback" property of $J'$, there is a unique "mediating morphism" $i': J \rightarrow J'$ "commuting" with \eqref{diag:subcfunctor}.\footnote{Use the fact that $n \circ i^{-1} \circ j = m \circ j = f \circ \pull{f}{m}$.} Similarly, the "pullback" property of $J$, there is a unique "mediating morphism" $i'^{-1}: J' \rightarrow J$ "commuting" with \eqref{diag:subcfunctor}.\footnote{Use the fact that $m \circ i \circ j' = n \circ j' = f \circ \pull{f}{n}$.} The fact that $i'$ and $i'^{-1}$ are inverses follows from viewing $i'^{-1} \circ  i'$ as a "mediating morphism" from the "pullback" $J$ to itself which must be the identity by uniqueness. Similarly for $i' \circ i'^{-1}$.

    For "functoriality" of $\Sub_{\mathbf{C}}$, we need to show $\pull{\id}{m} = m$ and $\pull{g}{\pull{f}{m}} = \pull{f \circ g}{m}$. The first equality follows from Exercise \ref{exer:limits:isopullback} and the second from the "pasting lemma".
\end{proof}
\begin{proof}[Solution to Exercise \ref{exer:universal:arrowcatfunctors}]\label{soln:universal:arrowcatfunctors}
    \begin{enumerate}
        \item On "morphisms", $\id$ sends $f: X \rightarrow Y$ to the "commutative" square $f: \id_X \rightarrow \id_Y$ depicted in \eqref{diag:commsquareimmageid}. Since the "identity" of $\id_X \in \obj{\arrowcat{\mathbf{C}}}$ is $\id_X:\id_X \rightarrow \id_X$ and the "composition" of "commutative" squares is done by "composing" the left part and right part independently, we conclude that $\id(f \circ g) = f \circ g = \id(f) \circ \id(g)$. Thus, $\id$ is a "functor".\begin{marginfigure}[-2\baselineskip]\begin{equation}\label{diag:commsquareimageid}
            \begin{tikzcd}
                X & X \\
                {Y} & {Y}
                \arrow["{f}"', from=1-1, to=2-1]
                \arrow["\id_X", from=1-1, to=1-2]
                \arrow["{f}", from=1-2, to=2-2]
                \arrow["\id_Y"', from=2-1, to=2-2]
            \end{tikzcd}
        \end{equation}\end{marginfigure}
        \item On "morphisms", $\source$ sends a "commutative" square $\phi: f \rightarrow g$ to the "morphism" $\source(f) \rightarrow \source(g)$ in the square, we denote it $\source(\phi)$. In other words, we send a "commutative" square to its left part. Again, since the "composition" in $\arrowcat{\mathbf{C}}$ is done independently on the left and right part, we find that $\source(\phi \circ \psi) = \source(\phi) \circ \source(\psi)$, thus $\source$ is a "functor" (see \eqref{diag:sourcetargetfunctors} for a visual aid).\begin{marginfigure}\begin{equation}\label{diag:sourcetargetfunctors}
            % https://q.uiver.app/?q=WzAsNixbMCwwLCJcXGJ1bGxldCJdLFsxLDAsIlxcYnVsbGV0Il0sWzAsMSwiXFxidWxsZXQiXSxbMSwxLCJcXGJ1bGxldCJdLFswLDIsIlxcYnVsbGV0Il0sWzEsMiwiXFxidWxsZXQiXSxbMCwxLCJmIl0sWzIsMywiZyJdLFs0LDUsImgiLDJdLFswLDIsIlxcc291cmNlKFxccHNpKSIsMl0sWzIsNCwiXFxzb3VyY2UoXFxwaGkpIiwyXSxbMSwzLCJcXHRhcmdldChcXHBzaSkiXSxbMyw1LCJcXHRhcmdldChcXHBoaSkiXV0=
            \begin{tikzcd}
                \bullet & \bullet \\
                \bullet & \bullet \\
                \bullet & \bullet
                \arrow["f", from=1-1, to=1-2]
                \arrow["g", from=2-1, to=2-2]
                \arrow["h"', from=3-1, to=3-2]
                \arrow["{\source(\psi)}"', from=1-1, to=2-1]
                \arrow["{\source(\phi)}"', from=2-1, to=3-1]
                \arrow["{\target(\psi)}", from=1-2, to=2-2]
                \arrow["{\target(\phi)}", from=2-2, to=3-2]
            \end{tikzcd}
        \end{equation}\end{marginfigure}
        \item On "morphisms", $\target$ sends a "commutative" square $\phi: f \rightarrow g$ to the "morphism" $\target(f) \rightarrow \target(g)$ in the square, we denote it $\target(\phi)$. With a similar argument to the second point, we conclude that $\target$ is a "functor".
    \end{enumerate}
\end{proof}
\begin{proof}[Solution to Exercise \ref{exer:universal:termslice}]\label{soln:universal:termslice}
    The "terminal" "object" of $\slice{\mathbf{C}}{X}$ is the "identity morphism" $\id_X : X \rightarrow X$. For any "object" of the "slice category" $f:A \rightarrow X$, we have the "commutative" triangle \eqref{diag:soltermslice} with $!=f$. Uniqueness of $!$ follows from $\id_X \circ ! = f \implies ! = f$.\begin{marginfigure}
        \begin{equation}\label{diag:soltermslice}
            \begin{tikzcd}
                A && X \\
                & X
                \arrow["f"', from=1-1, to=2-2]
                \arrow["f", dashed, from=1-1, to=1-3]
                \arrow["{\id_X}", from=1-3, to=2-2]
            \end{tikzcd}
        \end{equation}
    \end{marginfigure}
    The "dual" statement is that $\id_X$ is the "initial" "object" of $\coslice{X}{\mathbf{C}}$.
\end{proof}
\section{Solutions to Chapter \ref{chap:natural}}
\begin{proof}[Solution to Exercise \ref{exer:natural:componentwise}]\label{soln:natural:componentwise}
    ($\Rightarrow$) For any $g: Y \rightarrow Y'$, the "naturality" of $\phi$ yields this "commutative square".
    \begin{equation}\label{diag:naturalcomponentY}
        % https://q.uiver.app/?q=WzAsNCxbMCwwLCJGKFgsWSkiXSxbMCwxLCJGKFgsWScpIl0sWzEsMCwiRyhYLFkpIl0sWzEsMSwiRyhYLFknKSJdLFswLDEsIkYoXFxpZF9YLGYpIiwyXSxbMCwyLCJcXHBoaV97WCxZfSJdLFsyLDMsIkcoXFxpZF9YLGYpIl0sWzEsMywiXFxwaGlfe1gsWSd9IiwyXV0=
        \begin{tikzcd}
            {F(X,Y)} & {G(X,Y)} \\
            {F(X,Y')} & {G(X,Y')}
            \arrow["{F(X,g) = F(\id_X,g)}"', from=1-1, to=2-1]
            \arrow["{\phi_{X,Y}}", from=1-1, to=1-2]
            \arrow["{G(\id_X,g) = G(X,g)}", from=1-2, to=2-2]
            \arrow["{\phi_{X,Y'}}"', from=2-1, to=2-2]
        \end{tikzcd}
    \end{equation}
    We conclude that $\phi_{X,\placeholder}$ is a "natural transformation" $F(X,\placeholder)$. A symmetric argument works for $\phi_{\placeholder,Y}$ (see \eqref{diag:naturalcomponentX}).\begin{marginfigure}[-4\baselineskip]
        \begin{equation}\label{diag:naturalcomponentX}
            % https://q.uiver.app/?q=WzAsNCxbMCwwLCJGKFgsWSkiXSxbMSwwLCJHKFgsWSkiXSxbMCwxLCJGKFgnLFkpIl0sWzEsMSwiRyhYJyxZKSJdLFswLDEsIlxccGhpX3tYLFl9Il0sWzIsMywiXFxwaGlfe1gnLFl9IiwyXSxbMCwyLCJGKGYsXFxpZF9ZKSIsMl0sWzEsMywiRyhmLFxcaWRfWSkiXV0=
        \begin{tikzcd}
            {F(X,Y)} & {G(X,Y)} \\
            {F(X',Y)} & {G(X',Y)}
            \arrow["{\phi_{X,Y}}", from=1-1, to=1-2]
            \arrow["{\phi_{X',Y}}"', from=2-1, to=2-2]
            \arrow["{F(f,\id_Y)}"', from=1-1, to=2-1]
            \arrow["{G(f,\id_Y)}", from=1-2, to=2-2]
        \end{tikzcd}
        \end{equation}
    \end{marginfigure}

    ($\Leftarrow$) For any $(f,g): (X,Y) \rightarrow (X',Y')$, we note that, by "functoriality", $F(f,g) = F(f, \id_{Y'}) \circ F(\id_X, g)$ and similarly for $G$. Thus, we can combine the "naturality" of $\phi_{X,\placeholder}$ and $\phi_{\placeholder,Y}$ to obtain the "commutativity" of $\phi_{X,Y}$ as shown in \eqref{diag:naturalcombined}.
    \begin{equation}\label{diag:naturalcombined}
        % https://q.uiver.app/?q=WzAsNixbMCwwLCJGKFgsWSkiXSxbMCwxLCJGKFgsWScpIl0sWzEsMCwiRyhYLFkpIl0sWzEsMSwiRyhYLFknKSJdLFswLDIsIkYoWCcsWScpIl0sWzEsMiwiRyhYJyxZJykiXSxbMCwxLCJGKFxcaWRfWCxnKSJdLFswLDIsIlxccGhpX3tYLFl9Il0sWzIsMywiRyhcXGlkX1gsZykiLDJdLFsxLDMsIlxccGhpX3tYLFknfSJdLFs0LDUsIlxccGhpX3tYJyxZJ30iLDJdLFsxLDQsIkYoZixcXGlkX1kpIl0sWzMsNSwiRyhmLFxcaWRfWSkiLDJdLFswLDQsIkYoZixnKSIsMix7ImN1cnZlIjo0fV0sWzIsNSwiRyhmLGcpIiwwLHsiY3VydmUiOi00fV1d
        \begin{tikzcd}
            {F(X,Y)} & {G(X,Y)} \\
            {F(X,Y')} & {G(X,Y')} \\
            {F(X',Y')} & {G(X',Y')}
            \arrow["{F(\id_X,g)}", from=1-1, to=2-1]
            \arrow["{\phi_{X,Y}}", from=1-1, to=1-2]
            \arrow["{G(\id_X,g)}"', from=1-2, to=2-2]
            \arrow["{\phi_{X,Y'}}", from=2-1, to=2-2]
            \arrow["{\phi_{X',Y'}}"', from=3-1, to=3-2]
            \arrow["{F(f,\id_{Y'})}", from=2-1, to=3-1]
            \arrow["{G(f,\id_{Y'})}"', from=2-2, to=3-2]
            \arrow["{F(f,g)}"', curve={height=24pt}, from=1-1, to=3-1]
            \arrow["{G(f,g)}", curve={height=-24pt}, from=1-2, to=3-2]
\end{tikzcd}
    \end{equation}
\end{proof}
\begin{proof}[Solution to Exercise \ref{exer:natural:natiso}]\label{soln:natural:natiso}Let $F, G: \mathbf{C} \rightsquigarrow \mathbf{D}$ be "functors".

    ($\Rightarrow$) If $\phi: F \Rightarrow G$ is a "natural isomorphism", then it has an "inverse" $\phi^{-1}: G \Rightarrow F$ which satisfies $\phi \cdot \phi^{-1} = \one_G$ and $\phi^{-1}\cdot \phi = \one_F$. Looking at each "components", we find $\phi_X \circ (\phi^{-1})_X = \id_X$ and $(\phi^{-1})_X \circ  \phi_X = \id_X$, hence they are "isomorphisms@@CAT".

    ($\Leftarrow$) Let $\phi: F \Rightarrow G$ be a "natural transformation" such that $\phi_X$ is an "isomorphism@@CAT" for each $X \in \obj{\mathbf{C}}$. We claim that the family $\phi_X^{-1}$ is the "inverse" of $\phi$. After we show that this family is a "natural transformation" $G \Rightarrow F$, the construction implies it is the "inverse" of $\phi$. For any $f: X \rightarrow Y \in \mor{\mathbf{C}}$, the "naturality" of $\phi$ implies $\phi_Y \circ F(f) = G(f) \circ \phi_X$. "Pre-composing" with $\phi_X^{-1}$, we have $G(f) = \phi_Y \circ F(f) \circ \phi_X^{-1}$ and therefore
    \[\phi^{-1}_Y \circ G(f) = \phi^{-1}_Y \circ \phi_Y \circ F(f) \circ \phi_X^{-1} = F(f) \circ \phi_X^{-1}\]
    yields the "naturality" of $\phi^{-1}$.
\end{proof}
\begin{proof}[Solution to Exercise \ref{exer:natural:opcatfunc}]\label{soln:natural:opcatfunc}
    We have already seen in Exercise \ref{exer:duality:oppositefunc} that we can take the "dual" of a "functor" $F: \mathbf{C} \rightsquigarrow \mathbf{D}$ to obtain a "functor" $\op{F}: \op{\mathbf{C}} \rightsquigarrow \op{\mathbf{D}}$. It remains to check that a "natural transformation" $F \Rightarrow G$ can be identified with a "natural transformation" $\op{G} \Rightarrow \op{F}$. This follows from observing that the "naturality" square \eqref{diag:natsquarenotop} in $\mathbf{D}$ corresponds to the "naturality" square \eqref{diag:natsquareop} in $\op{\mathbf{D}}$.\footnote{i.e.: \eqref{diag:natsquarenotop} "commutes" if and only if \eqref{diag:natsquareop} "commutes".}\\
    \begin{minipage}{0.49\textwidth}
        \begin{equation}\label{diag:natsquarenotop}
            \begin{tikzcd}
                FX & GX \\
                FY & GY
                \arrow["Ff"', from=1-1, to=2-1]
                \arrow["Gf", from=1-2, to=2-2]
                \arrow["{\phi_X}", from=1-1, to=1-2]
                \arrow["{\phi_Y}"', from=2-1, to=2-2]
            \end{tikzcd}
        \end{equation}
    \end{minipage}
    \begin{minipage}{0.49\textwidth}
        \begin{equation}\label{diag:natsquareop}
            % https://q.uiver.app/?q=WzAsNCxbMSwxLCJcXG9we0Z9WCJdLFsxLDAsIlxcb3B7Rn1ZIl0sWzAsMSwiXFxvcHtHfVgiXSxbMCwwLCJcXG9we0d9WSJdLFsxLDAsIlxcb3B7Rn1mIl0sWzMsMiwiXFxvcHtHfWYiLDJdLFsyLDAsIlxccGhpX1giLDJdLFszLDEsIlxccGhpX1kiXV0=
        \begin{tikzcd}
            {\op{G}Y} & {\op{F}Y} \\
            {\op{G}X} & {\op{F}X}
            \arrow["{\op{F}f}", from=1-2, to=2-2]
            \arrow["{\op{G}f}"', from=1-1, to=2-1]
            \arrow["{\phi_X}"', from=2-1, to=2-2]
            \arrow["{\phi_Y}", from=1-1, to=1-2]
        \end{tikzcd}
        \end{equation}
    \end{minipage}
\end{proof}
\begin{proof}[Solution to Exercise \ref{exer:natural:compositionisfunc}]\label{soln:natural:compositionisfunc}%TODO: and say how you use it in other exercises.
    On "morphisms", this "functor" must send a pair of "natural transformations" $\eta: F \Rightarrow F'$ and $\phi: G \Rightarrow G'$ to a "natural transformation" $FG \Rightarrow F'G'$. This is exactly what "horizontal composition" does.
    
    To see that "horizontal composition" is "functorial", first note that $\one_F \horcomp \one_G = \one_{FG}$. Next, the fact that "horizontal composition" commutes with "composition" of "functors" is exactly the "interchange identity".
\end{proof}
\begin{proof}[Solution to Exercise \ref{exer:natural:equivequiv}]\label{soln:natural:equivequiv}
    We need to show that $\eqCat$ is reflexive, symmetric and transitive. Symmetry is trivial because the definition of $\mathbf{C} \eqCat \mathbf{D}$ is symmetric. Reflexivity follows from the fact that the "identity functor" on any "category" is "fully faithful" and "essentially surjective".

    For transitivity, given the "categories" and "functors" represented in \eqref{diag:composeequiv} with "natural isomorphisms" $\phi: FG\Rightarrow \id_{\mathbf{D}}$, $\psi: GF \Rightarrow \id_{\mathbf{C}}$, $\phi': F'G'\Rightarrow \id_{\mathbf{E}}$ and $\psi': G'F' \Rightarrow \id_{\mathbf{D}}$, we claim that the "composition" $G \circ G'$ is the "quasi-inverse" of $F' \circ F$.
    \begin{marginfigure}[-3\baselineskip]
        \begin{equation}\label{diag:composeequiv}
            % https://q.uiver.app/?q=WzAsMyxbMCwwLCJcXG1hdGhiZntDfSJdLFsyLDAsIlxcbWF0aGJme0R9Il0sWzQsMCwiXFxtYXRoYmZ7RX0iXSxbMCwxLCJGIiwwLHsiY3VydmUiOi0yLCJzdHlsZSI6eyJib2R5Ijp7Im5hbWUiOiJzcXVpZ2dseSJ9fX1dLFsxLDIsIkYnIiwwLHsiY3VydmUiOi0yLCJzdHlsZSI6eyJib2R5Ijp7Im5hbWUiOiJzcXVpZ2dseSJ9fX1dLFsxLDAsIkciLDAseyJjdXJ2ZSI6LTIsInN0eWxlIjp7ImJvZHkiOnsibmFtZSI6InNxdWlnZ2x5In19fV0sWzIsMSwiRyciLDAseyJjdXJ2ZSI6LTIsInN0eWxlIjp7ImJvZHkiOnsibmFtZSI6InNxdWlnZ2x5In19fV1d
            \begin{tikzcd}
            {\mathbf{C}} && {\mathbf{D}} && {\mathbf{E}}
            \arrow["F", curve={height=-12pt}, squiggly, from=1-1, to=1-3]
            \arrow["{F'}", curve={height=-12pt}, squiggly, from=1-3, to=1-5]
            \arrow["G", curve={height=-12pt}, squiggly, from=1-3, to=1-1]
            \arrow["{G'}", curve={height=-12pt}, squiggly, from=1-5, to=1-3]
            \end{tikzcd}
        \end{equation}
    \end{marginfigure}
    Since the "biaction" of "functors" preserves "natural isomorphisms",\footnote{This holds because acting on the left or right with a "functor" is a "functor", part of this is shown in the next solution and it also follows from the previous exercise.} we have two "natural isomorphisms"
    \[\phi' \cdot (F'\phi G'):F'FGG' \Rightarrow \id_{\mathbf{E}} \text{ and } \psi \cdot (G\psi' F):GG'F'F \Rightarrow \id_{\mathbf{C}},\]
    which shows $\mathbf{C} \eqCat \mathbf{E}$.
\end{proof}
\begin{proof}[Solution to Exercise \ref{exer:natural:equivfunccat}]\label{soln:natural:equivfunccat}
    We will show the following two implications
    \begin{align*}
        \forall \mathbf{D} \quad \mathbf{C} \eqCat \mathbf{C}' &\implies \catFunc{\mathbf{C}}{\mathbf{D}} \eqCat \catFunc{\mathbf{C}'}{\mathbf{D}}\\
        \forall \mathbf{C} \quad \mathbf{D} \eqCat \mathbf{D}' &\implies \catFunc{\mathbf{C}}{\mathbf{D}} \eqCat \catFunc{\mathbf{C}}{\mathbf{D}'}
    \end{align*}
    and infer that $\mathbf{C} \eqCat \mathbf{C}'$ and $\mathbf{D} \eqCat \mathbf{D}'$ implies 
    \[\catFunc{\mathbf{C}}{\mathbf{D}} \eqCat \catFunc{\mathbf{C}'}{\mathbf{D}} \eqCat  \catFunc{\mathbf{C}'}{\mathbf{D}'}.\]
    For the first implication, let $F: \mathbf{C} \rightsquigarrow \mathbf{C}'$ and $G: \mathbf{C}' \rightsquigarrow \mathbf{C}$ be "quasi-inverses". We define the "functor" $(\placeholder)F: \catFunc{\mathbf{C}'}{\mathbf{D}} \rightsquigarrow \catFunc{\mathbf{C}}{\mathbf{D}}$ that acts on "functors" by "pre-composition" and on "natural transformations" by the right action in Definition \ref{defn:rightaction}.\footnote{i.e.: $H: \mathbf{C} \rightsquigarrow \mathbf{D}$ is mapped to $HF = H \circ F$ and $\phi: H \Rightarrow H'$ is mapped to $\phi F$. "Functoriality" follows from the properties of the right action.
    
    Another way to show "functoriality" is to recall that $\phi F = \phi \horcomp \one_F$ and hence $(\placeholder)F$ is the "composition" of the "functor" \[\id_{\catFunc{\mathbf{C}'}{\mathbf{D}}} \cattimes F:\catFunc{\mathbf{C}'}{\mathbf{D}}\cattimes \termcat \rightsquigarrow \catFunc{\mathbf{C}'}{\mathbf{D}}\cattimes \catFunc{\mathbf{C}}{\mathbf{C}'}\] with the "horizontal composition" "functor" defined in Exercise \ref{exer:natural:compositionisfunc}.} Similarly, we define the "functor" $(\placeholder)G: \catFunc{\mathbf{C}}{\mathbf{D}} \rightsquigarrow \catFunc{\mathbf{C}'}{\mathbf{D}}$. We claim that $(\placeholder)F$ and $(\placeholder)G$ are "quasi-inverses".

    Let $\Phi: GF \Rightarrow \id_{\mathbf{C}}$ be a "natural isomorphism" witnessing $F$ and $G$ being "quasi-inverses", then $(\placeholder)\Phi$ is a "natural isomorphism" from $(\placeholder)GF$ to $\id_{\catFunc{\mathbf{C}}{\mathbf{D}}}$. Indeed, for any $\phi: H \Rightarrow H' \in \mor{\catFunc{\mathbf{C}}{\mathbf{D}}}$, \eqref{diag:natisoequivfunccat} "commutes" as the top path and bottom path are both equal to $\phi \horcomp \Phi$ and $H\Phi$ is an "isomorphism" because $\Phi$ is and "functors" preserve "isomorphisms".
    \begin{equation}\label{diag:natisoequivfunccat}
        \begin{tikzcd}
            HGF & H \\
            {H'GF} & {H'}
            \arrow["{\phi GF}"', from=1-1, to=2-1]
            \arrow["H\Phi", from=1-1, to=1-2]
            \arrow["\phi", from=1-2, to=2-2]
            \arrow["{H'\Phi}"', from=2-1, to=2-2]
        \end{tikzcd}
    \end{equation}
    We leave to you the symmetric argument showing $(\placeholder)FG \isoCAT \id_{\catFunc{\mathbf{C}'}{\mathbf{D}}}$ and the similar argument for the second implication.
    %TODO: footnote naturality square notation.
\end{proof}
\section{Solutions to Chapter \ref{chap:yoneda}}
\begin{proof}[Solution to Exercise \ref{exer:yoneda:initisrepr1}]\label{soln:yoneda:initisrepr1}
    ($\Rightarrow$) Suppose there is a "natural isomorphism" $\phi: \Hom_{\mathbf{C}}(X,\placeholder) \Rightarrow \terminal$, then for any "object" $Y \in \obj{\mathbf{C}}$, there is a bijection $\Hom_{\mathbf{C}}(X,Y) \isoCAT \{\star\}$. Hence, there is a unique "morphism" $X \rightarrow Y$.

    ($\Leftarrow$) Suppose that $X$ is "initial", then for any $Y \in \obj{\mathbf{C}}$, we have an "isomorphism@@CAT" $\phi_Y: \Hom_{\mathbf{C}}(X,Y) \rightarrow \terminal(Y)$ which sends the unique "morphism" $X \rightarrow Y$ to $\star$. We need to show this family is "natural" in $Y$. Let $f: Y \rightarrow Y'\in \mor{\mathbf{C}}$, \eqref{diag:initialrepr} clearly "commutes" because all sets are singletons.\begin{marginfigure}\begin{equation}\label{diag:initialrepr}
       % https://q.uiver.app/?q=WzAsNCxbMCwwLCJcXEhvbV97XFxtYXRoYmZ7Q319KFgsWSkiXSxbMCwxLCJcXEhvbV97XFxtYXRoYmZ7Q319KFgsWScpIl0sWzEsMCwiXFx0ZXJtaW5hbChZKSJdLFsxLDEsIlxcdGVybWluYWwoWScpIl0sWzAsMSwiZiBcXGNpcmMgXFxwbGFjZWhvbGRlciIsMl0sWzAsMiwiXFxwaGlfWSJdLFsyLDMsIlxcaWRfe1xcdGVybWluYWx9Il0sWzEsMywiXFxwaGlfe1knfSIsMl1d
        \begin{tikzcd}
        {\Hom_{\mathbf{C}}(X,Y)} & {\terminal(Y)} \\
        {\Hom_{\mathbf{C}}(X,Y')} & {\terminal(Y')}
        \arrow["{f \circ \placeholder}"', from=1-1, to=2-1]
        \arrow["{\phi_Y}", from=1-1, to=1-2]
        \arrow["{\id_{\terminal}}", from=1-2, to=2-2]
        \arrow["{\phi_{Y'}}"', from=2-1, to=2-2]
        \end{tikzcd}
    \end{equation}\end{marginfigure}
\end{proof}
\section{Solutions to Chapter \ref{chap:adjoints}}
\begin{proof}[Solution to Exercise \ref{exer:adjoints:chainadjCarrow}]\label{soln:adjointsexer:adjoints:chainadjCarrow}
    We will proceed by defining the "units@@ADJ" and "counits@@ADJ" because, as you will see, they are practically given and then we will verify they satisfy the "triangle identities". We denote $(\phi_X,\phi_Y)$ for a "commutative" square with $\sourcearr(\phi_X,\phi_Y) = \phi_X$ and $\targetarr(\phi_X,\phi_Y) = \phi_Y$

    ($\targetarr \adjoint \idarr$) The "component" of the "unit@@ADJ" at $f \in \obj{\arrowcat{\mathbf{C}}}$ is a "commutative" square from $f$ to $\idarr(\targetarr(f)) = \id_{\target(f)}$. You should convince yourself that \eqref{diag:unittargetadj} is the only such square that is guaranteed to exist no matter what $\mathbf{C}$ is, we have $\eta_f = (f,\id_{\target(f)})$. The "component" of the "counit@@ADJ" at $X \in \obj{\mathbf{C}}$ is a "morphism" from $\target(\id_X) = X$ to $X$. Again, the only possible choice is $\varepsilon_X = \id_X$. We check in the following derivations that the "triangle identities" hold.
    \begin{gather*}
        \varepsilon_{\targetarr(f)} \circ \targetarr(\eta_f) = \id_{\target(f)} \circ \id_{\target(f)} = \id_{\targetarr(f)}\\
        \idarr(\varepsilon_X) \circ \eta_{\idarr(X)} = (\id_X,\id_X) \circ (\id_X,\id_X) = (\id_X,\id_X) = \id_{\idarr(X)}.
    \end{gather*}
    \begin{marginfigure}[-16\baselineskip]\begin{equation}\label{diag:unittargetadj}
        % https://q.uiver.app/?q=WzAsNCxbMCwwLCJcXHNvdXJjZShmKSJdLFswLDEsIlxcdGFyZ2V0KGYpIl0sWzEsMCwiXFx0YXJnZXQoZikiXSxbMSwxLCJcXHRhcmdldChmKSJdLFswLDEsImYiLDJdLFsyLDMsIlxcaWRfe1xcdGFyZ2V0KGYpfSJdLFsxLDMsIlxcaWRfe1xcdGFyZ2V0KGYpfSIsMl0sWzAsMiwiZiJdXQ==
        \begin{tikzcd}
            {\source(f)} & {\target(f)} \\
            {\target(f)} & {\target(f)}
            \arrow["f"', from=1-1, to=2-1]
            \arrow["{\id_{\target(f)}}", from=1-2, to=2-2]
            \arrow["{\id_{\target(f)}}"', from=2-1, to=2-2]
            \arrow["f", from=1-1, to=1-2]
        \end{tikzcd}
    \end{equation}\end{marginfigure}

    ($\idarr \adjoint \sourcearr$) The "component" of the "unit@@ADJ" at $X \in \obj{\mathbf{C}}$ is a "morphism" from $X$ to $\sourcearr(\idarr(X))= X$, thus $\eta_X = \id_X$. The "component" of the "counit@@ADJ" at $f \in \obj{\arrowcat{\mathbf{C}}}$ is a "commutative" square from $\idarr(\sourcearr(f)) = \id_{\source(f)}$ to $f$. Again, there is only once choice: $\varepsilon_f = (\id_{\source(f)},f)$ depicted in \eqref{diag:counitsourceadj}. The following derivations show the "triangle identities" hold.
    \begin{gather*}
        \varepsilon_{\idarr(X)} \circ \idarr(\eta_X) = (\id_X,\id_X) \circ (\id_X,\id_X) = (\id_X,\id_X) = \id_{\idarr(X)}\\
        \sourcearr(\varepsilon_f) \circ \eta_{\sourcearr(f)} = \id_{\source(f)} \circ \id_{\source(f)} = \id_{\sourcearr(f)}.
    \end{gather*}
    \begin{marginfigure}[-12\baselineskip]\begin{equation}\label{diag:counitsourceadj}
        % https://q.uiver.app/?q=WzAsNCxbMCwxLCJcXHNvdXJjZShmKSJdLFswLDAsIlxcc291cmNlKGYpIl0sWzEsMSwiXFx0YXJnZXQoZikiXSxbMSwwLCJcXHNvdXJjZShmKSJdLFsxLDAsIlxcaWRfe1xcc291cmNlKGYpfSIsMl0sWzMsMiwiZiJdLFsxLDMsIlxcaWRfe1xcc291cmNlKGYpfSJdLFswLDIsImYiLDJdXQ==
        \begin{tikzcd}
            {\source(f)} & {\source(f)} \\
            {\source(f)} & {\target(f)}
            \arrow["{\id_{\source(f)}}"', from=1-1, to=2-1]
            \arrow["f", from=1-2, to=2-2]
            \arrow["{\id_{\source(f)}}", from=1-1, to=1-2]
            \arrow["f"', from=2-1, to=2-2]
        \end{tikzcd}
    \end{equation}\end{marginfigure}
    ($\mathsf{?}\adjoint \targetarr$) If $\targetarr$ has a "left adjoint" $\mathsf{?}$, then there is a "isomorphism@@CAT" $\Hom_{\arrowcat{\mathbf{C}}}(\mathsf{?}X, f) \isoCAT \Hom_{\mathbf{C}}(X,\target(f))$ that is "natural" in $X$ and $f$.
\end{proof}
\begin{proof}[Solution to Exercise \ref{exer:adjoints:alllimitspreserved}]\label{soln:adjoints:alllimitspreserved}
    Using Theorem \ref{thm:limitadj}, Theorem \ref{thm:adjcomp} and Proposition \ref{prop:adjcompisadj}, we can obtain two chains of "adjunctions".
    \[% https://q.uiver.app/?q=WzAsMyxbMCwwLCJcXG1hdGhiZntDfSJdLFsxLDAsIlxcbWF0aGJme0R9Il0sWzIsMCwiXFxjYXRGdW5je1xcbWF0aGJme0p9fXtcXG1hdGhiZntEfX0iXSxbMCwxLCJMIiwwLHsib2Zmc2V0IjotMn1dLFsxLDIsIlxcZ2RpYWdGdW5jX3tcXG1hdGhiZntEfX1ee1xcbWF0aGJme0p9fSIsMCx7Im9mZnNldCI6LTJ9XSxbMiwxLCJcXGxpbV97XFxtYXRoYmZ7Sn19IiwwLHsib2Zmc2V0IjotMn1dLFsxLDAsIlIiLDAseyJvZmZzZXQiOi0yfV0sWzMsNiwiIiwwLHsibGV2ZWwiOjEsInN0eWxlIjp7Im5hbWUiOiJhZGp1bmN0aW9uIn19XSxbNCw1LCIiLDAseyJsZXZlbCI6MSwic3R5bGUiOnsibmFtZSI6ImFkanVuY3Rpb24ifX1dXQ==
    \begin{tikzcd}
        {\mathbf{C}} & {\mathbf{D}} & {\catFunc{\mathbf{J}}{\mathbf{D}}}
        \arrow[""{name=0, anchor=center, inner sep=0}, "L", shift left=2, from=1-1, to=1-2]
        \arrow[""{name=1, anchor=center, inner sep=0}, "{\gdiagFunc_{\mathbf{D}}^{\mathbf{J}}}", shift left=2, from=1-2, to=1-3]
        \arrow[""{name=2, anchor=center, inner sep=0}, "{\lim_{\mathbf{J}}}", shift left=2, from=1-3, to=1-2]
        \arrow[""{name=3, anchor=center, inner sep=0}, "R", shift left=2, from=1-2, to=1-1]
        \arrow["\dashv"{anchor=center, rotate=-90}, draw=none, from=0, to=3]
        \arrow["\dashv"{anchor=center, rotate=-90}, draw=none, from=1, to=2]
    \end{tikzcd}\hspace{4em}
    % https://q.uiver.app/?q=WzAsMyxbMCwwLCJcXG1hdGhiZntDfSJdLFsxLDAsIlxcY2F0RnVuY3tcXG1hdGhiZntKfX17XFxtYXRoYmZ7Q319Il0sWzIsMCwiXFxjYXRGdW5je1xcbWF0aGJme0p9fXtcXG1hdGhiZntEfX0iXSxbMCwxLCJcXGdkaWFnRnVuY197XFxtYXRoYmZ7Q319XntcXG1hdGhiZntKfX0iLDAseyJvZmZzZXQiOi0yfV0sWzEsMiwiTFxccGxhY2Vob2xkZXIiLDAseyJvZmZzZXQiOi0yfV0sWzIsMSwiUlxccGxhY2Vob2xkZXIiLDAseyJvZmZzZXQiOi0yfV0sWzEsMCwiXFxsaW1fe1xcbWF0aGJme0p9fSIsMCx7Im9mZnNldCI6LTJ9XSxbMyw2LCIiLDAseyJsZXZlbCI6MSwic3R5bGUiOnsibmFtZSI6ImFkanVuY3Rpb24ifX1dLFs0LDUsIiIsMCx7ImxldmVsIjoxLCJzdHlsZSI6eyJuYW1lIjoiYWRqdW5jdGlvbiJ9fV1d
    \begin{tikzcd}
        {\mathbf{C}} & {\catFunc{\mathbf{J}}{\mathbf{C}}} & {\catFunc{\mathbf{J}}{\mathbf{D}}}
        \arrow[""{name=0, anchor=center, inner sep=0}, "{\gdiagFunc_{\mathbf{C}}^{\mathbf{J}}}", shift left=2, from=1-1, to=1-2]
        \arrow[""{name=1, anchor=center, inner sep=0}, "L\placeholder", shift left=2, from=1-2, to=1-3]
        \arrow[""{name=2, anchor=center, inner sep=0}, "R\placeholder", shift left=2, from=1-3, to=1-2]
        \arrow[""{name=3, anchor=center, inner sep=0}, "{\lim_{\mathbf{J}}}", shift left=2, from=1-2, to=1-1]
        \arrow["\dashv"{anchor=center, rotate=-90}, draw=none, from=0, to=3]
        \arrow["\dashv"{anchor=center, rotate=-90}, draw=none, from=1, to=2]
    \end{tikzcd}\]
    Then, observing that both "composite" "left adjoints" are equal,\footnote{Both $\gdiagFunc_{\mathbf{D}}^{\mathbf{J}} \circ L$ and $L\gdiagFunc_{\mathbf{C}}^{\mathbf{J}}$ send $X \in \obj{\mathbf{C}}$ to the "constant functor" at $LX$.} we conclude by Corollary \ref{cor:rightadjunique} that $R\lim_{\mathbf{J}} \isoCAT \lim_{\mathbf{J}}(R\placeholder)$. %TODO: explain the end.
\end{proof}
\section{Solutions to Chapter \ref{chap:monads}}
\begin{proof}[Solution to Exercise \ref{exer:monad:adjmorphcommute}]\label{soln:monad:adjmorphcommute}
    By the "universal property" of $\eta'$ and one of the "triangle identities", $\varepsilon'_{KA}$ is the unique "morphism" such that $R'\varepsilon'_{KA} \circ \eta'_{R'KA} = \id_{R'KA}$ (see \eqref{diag:univetavareps}).\begin{marginfigure}\begin{equation}\label{diag:univetavareps}
        % https://q.uiver.app/?q=WzAsNyxbMSwwLCJSJ0wnUidLQSJdLFswLDAsIlInS0EiXSxbMSwxLCJSJ0tBIl0sWzMsMSwiS0EiXSxbMywwLCJMJ1InS0EiXSxbMiwwXSxbNCwwXSxbMSwwLCJcXGV0YSdfe1InS0F9Il0sWzEsMiwiXFxpZF97UidLQX0iLDJdLFs0LDMsIlxcdmFyZXBzaWxvbidfe0tBfSIsMCx7InN0eWxlIjp7ImJvZHkiOnsibmFtZSI6ImRhc2hlZCJ9fX1dLFswLDIsIlInXFx2YXJlcHNpbG9uJ197S0F9IiwwLHsic3R5bGUiOnsiYm9keSI6eyJuYW1lIjoiZGFzaGVkIn19fV0sWzksMTAsIlIiLDIseyJsYWJlbF9wb3NpdGlvbiI6NDAsInNob3J0ZW4iOnsic291cmNlIjoxMCwidGFyZ2V0IjozMH0sImxldmVsIjoxfV1d
        \begin{tikzcd}
            {R'KA} & {R'L'R'KA} & {} & {L'R'KA} & {} \\
            & {R'KA} && KA
            \arrow["{\eta'_{R'KA}}", from=1-1, to=1-2]
            \arrow["{\id_{R'KA}}"', from=1-1, to=2-2]
            \arrow[""{name=0, anchor=center, inner sep=0}, "{\varepsilon'_{KA}}", dashed, from=1-4, to=2-4]
            \arrow[""{name=1, anchor=center, inner sep=0}, "{R'\varepsilon'_{KA}}", dashed, from=1-2, to=2-2]
            \arrow["R"'{pos=0.4}, shorten <=8pt, shorten >=24pt, from=0, to=1]
        \end{tikzcd}
    \end{equation}\end{marginfigure}
    We claim that $K\varepsilon_A$ also fits in the place of $\varepsilon'_{KA}$ in \eqref{diag:univetavareps} which means they are equal by uniqueness. We need to show $R'K\varepsilon_A \circ \eta'_{R'KA} = \id_{R'KA}$. Recalling that $\eta' = \eta$ and $R'K = R$, we rewrite the equality as $R\varepsilon_A \circ \eta_{RA} = \id_{RA}$ which holds by a "triangle identity".
\end{proof}
\end{document}